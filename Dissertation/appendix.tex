%\input{Dissertation/appendixsetup}   % Предварительные настройки для правильного подключения Приложений
%\chapter{\MakeUppercase{список публікацій за темою дисертації}} \label{AppendixA}
\chapter*{\MakeUppercase{Додатки}}						% Заголовок
\addcontentsline{toc}{chapter}{\MakeUppercase{Додатки}}
\section*{Додаток А. Список публікацій за темою дисертації та відомості про апробацію результатів}


\begin{center}%
\emph{Наукові праці, в яких опубліковано основні наукові результати дисертації}
\end{center}%
%\subsection*{Наукові праці, в яких опубліковано основні наукові результати дисертації}
\begin{enumerate}[label=\arabic*.,leftmargin=1em,itemindent=1em]
%\begin{enumerate}[label=\arabic*.,leftmargin=0em,itemindent=2em]
%\setcounter{enumi}{25}
\item
Acousto--defect interaction in irradiated and non--irradiated silicon
  $n^+$--$p$ structure~/ O.~Ya.~Olikh, A.~M.~Gorb, R.~G.~Chupryna,
  O.~V.~Pristay-Fenenkov~// \emph{J. Appl. Phys.} --- 2018. --- Apr. ---
 Vol. 123, no.~16. --- P.~161573--1--161573--12.

\item
\emph{Olikh,~O.Ya.} Acoustically driven degradation in single crystalline
  silicon solar cell~/ O.Ya.~Olikh~// \emph{Superlattices Microstruct.} ---
  2018. --- May. ---
  Vol. 117. ---
  P.~173--188.

\item
\emph{Olikh,~Oleg}. On the mechanism of ultrasonic loading effect in
  silicon--based {S}chottky diodes~/ Oleg~Olikh, Katerina~Voytenko~//
  \emph{Ultrasonics}. ---
  2016. --- Mar. ---
  Vol.~66, no.~1. ---
  P.~1--3.

\item
Effect of ultrasound on reverse leakage current of silicon {S}chottky barrier
  structure~/ O.~Ya.~Olikh, K.~V.~Voytenko, R.~M.~Burbelo, Ja.~M.~Olikh~//
  \emph{Journal of Semiconductors}. ---
  2016. --- Dec. ---
  Vol.~37, no.~12. ---
  P.~122002--1--122002--7.

\item
\emph{Olikh,~O.~Ya.} Review and test of methods for determination of the
  {S}chottky diode parameters~/ O.~Ya.~Olikh~// \emph{J. Appl. Phys.} ---
  2015. --- Jul. ---
  Vol. 118, no.~2. ---
  P.~024502--1--024502--14.

\item
\emph{Olikh,~O.~Ya.} Ultrasound influence on {I}--{V}--{T} characteristics
  of silicon {S}chottky barrier structure~/ O.~Ya.~Olikh, K.~V.~Voytenko,
  R.~M.~Burbelo~// \emph{J. Appl. Phys.} ---
  2015. --- Jan. ---
  Vol. 117, no.~4. ---
  P.~044505--1--044505--7.

\item
\emph{Olikh,~Oleg}. Reversible influence of ultrasound on
  $\gamma-$irradiated {M}o/n-{S}i {S}chottky barrier structure~/ Oleg~Olikh~//
  \emph{Ultrasonics}. ---
  2015. --- Feb. ---
  Vol.~56. ---
  P.~545--550.

\item
Особливості дислокаційного поглинання
  ультразвуку в безсубблочних кристалах
  {C}d$_{0,2}${H}g$_{0,8}${T}e~/ І.~О.~Лисюк, Я.~М.~Оліх,
  О.~Я.~Оліх, Г.~В.~Бекетов~// \emph{УФЖ}. ---
  2014. ---
  Т.~59, {№}~1. ---
  {С.}~50--57.

\item
\emph{Olikh,~O.~Ya.} Non-Monotonic $\gamma-$Ray Influence on {M}o/n-{S}i
  {S}chottky Barrier Structure Properties~/ O.~Ya.~Olikh~// \emph{IEEE
  Trans. Nucl. Sci.} ---
  2013. --- Feb. ---
  Vol.~60, no.~1. ---
  P.~394--401.

\item
\emph{Оліх,~О.~Я.} Особливості впливу
  ультразвуку на перенесення заряду в
  кремнієвих структурах з бар’єром {Ш}отки
  залежно від дози $\gamma$--опромінення~/
  О.~Я.~Оліх~// \emph{Сенсорна електроніка і
  мікросистемні технології}. ---
  2013. ---
  Т.~10, {№}~1. ---
  {С.}~47--55.

\item
\emph{Олих,~О.~Я.} Влияние ультразвукового
  нагружения на протекание тока в
  структурах {M}o/n--n$^+$--{S}i c барьером {Ш}оттки~/
  О.~Я.~Олих~// \emph{Физика и техника
  полупроводников}. ---
  2013. ---
  Т.~47, {№}~7. ---
  {С.}~979--984.

\item
\emph{Оліх,~О.~Я.} Особливості перенесення
  заряду в структурах {M}o/n--{S}i з бар’єром
  {Ш}отки~/ О.~Я.~Оліх~// \emph{УФЖ}. ---
  2013. ---
  Т.~58, {№}~2. ---
  {С.}~126--134.

\item
\emph{Олих,~О.~Я.} Особенности динамических
  акустоиндуцированных изменений
  фотоэлектрических параметров кремниевых
  солнечных элементов~/ О.~Я.~Олих~//
  \emph{Физика и техника полупроводников}. ---
  2011. ---
  Т.~45, {№}~6. ---
  {С.}~816--822.

\item
\emph{Оліх,~Я.~М.} Інформаційний чинник
  акустичної дії на структуру дефектних
  комплексів у напівпровідниках~/ Я.~М.~Оліх,
  О.~Я.~Оліх~// \emph{Сенсорна електроніка і
  мікросистемні технології}. ---
  2011. ---
  Т. 2(8), {№}~2. ---
  {С.}~5--12.

\item
\emph{Оліх,~О.~Я.} Особливості впливу
  нейтронного опромінення на динамічну
  акустодефектну взаємодію у кремнієвих
  сонячних елементах~/ О.~Я.~Оліх~// \emph{УФЖ}.
  ---
  2010. ---
  Т.~55, {№}~7. ---
  {С.}~770--776.


\item
Ultrasonically Recovered Performance of $\gamma-$Irradiated Metal-Silicon
  Structures~/ A.M.~Gorb, O.A.~Korotchenkov, O.Ya~Olikh, A.O.~Podolian~//
  \emph{IEEE Trans. Nucl. Sci.} ---
  2010. --- June. ---
  Vol.~57, no.~3. ---
  P.~1632--1639.

\item
\emph{Олих,~О.~Я.} Изменение активности
  рекомбинационных центров в кремниевых
  p--n--структурах в условиях акустического
  нагружения~/ О.~Я.~Олих~// \emph{Физика и
  техника полупроводников}. ---
  2009. ---
  Т.~43, {№}~6. ---
  {С.}~774--779.

\item
\emph{Оліх,~О.~Я.} Робота кремнієвих сонячних
  елементів в умовах акустичного
  навантаження мегагерцового діапазону~/
  О.~Я.~Оліх, Р.~М.~Бурбело, М.~К.~Хіндерс~//
  \emph{Сенсорна електроніка і
  мікросистемні технології}. ---
  2007. ---
  Т.~4, {№}~3. ---
  {С.}~40--45.

\item
\emph{Olikh,~O.Ya.} The Dynamic Ultrasound Influence on Diffusion and Drift
  of the Charge Carriers in Silicon p--n Structures~/ O.Ya.~Olikh, R.~Burbelo,
  M.~Hinders~// Semiconductor Defect Engineering --- Materials, Synthetic,
  Structures and Devices II~/ Ed. by S.~Ashok, P.~Kiesel, J.~Chevallier,
  T.~Ogino. ---
  Vol.~994 of \emph{Materials Research Society Symposium
  Proceedings}. ---
  Warrendale, PA: 2007. ---
  P.~269--274.

\item
\emph{Олих,~О.~Я.} Акустостимулированные
  коррекции вольт--амперных характеристик
  арсенид--галлиевых структур с контактом
  {Ш}оттки~/ О.~Я.~Олих, Т.~Н.~Пинчук~//
  \emph{Письма в Журнал Технической Физики}.
  ---
  2006. ---
  Т.~32, {№}~12. ---
  {С.}~22--27.

\item
\emph{Конакова,~Р.В.} Влияние микроволновой
  обработки на уровень остаточной
  деформации и параметры глубоких уровней
  монокристаллах карбида кремния~/
  Р.В.~Конакова, П.М.~Литвин, О.Я.~Олих~//
  \emph{Физика и химия обработки материалов}.
  ---
  2005. ---
  {№}~2. ---
  {С.}~19--22.


\item
\emph{Конакова,~Р.В.} Влияние микроволновой
  обработки на глубокие уровни
  монокристаллов {G}a{A}s и {S}i{C}~/ Р.В.~Конакова,
  П.М.~Литвин, О.Я.~Олих~// \emph{Петербургский
  журнал электроники}. ---
  2004. ---
  {№}~1. ---
  {С.}~20--24.

\item
\emph{Olikh,~Ja.~М.} Active ultrasound effects in the future usage in
  sensor electronics~/ Ja.~М.~Olikh, O.Ya.~Olikh~// \emph{Сенсорна
  електроніка і мікросистемні технології}.
  ---
  2004. ---
  Т.~1, {№}~1. ---
  {С.}~19--29.

\item
\emph{Olikh,~O.Ya.} Acoustoelectric transient spectroscopy of microwave
  treated {G}a{A}s--based structures~/ O.Ya.~Olikh~// \emph{Semiconductor
  Physics, Quantum Electronics \& Optoelectronics}. ---
  2003. ---
  Vol.~6, no.~4. ---
  P.~450--453.

\item
\emph{Оліх,~О.Я.} Акустостимульовані
  динамічні ефекти в сонячних елементах на
  основі кремнію~/ О.Я.~Оліх~// \emph{Вісник
  Київського ун-ту, Сер.: Фізико-математичні
  науки}. ---
  2003. ---
  {№}~4. ---
  {С.}~408--414.
\end{enumerate}

\begin{center}%
\emph{Наукові праці, які засвідчують апробацію матеріалів дисертації}
\end{center}%
\begin{enumerate}[label=\arabic*.,leftmargin=1em,itemindent=1em]
\setcounter{enumi}{25}
\item
\emph{Оліх,~О.~Я.} Ефекти активного
  ультразвуку в напівпровідникових
  кристалах~/ О.~Я.~Оліх~// 1--а {У}країнська
  наукова конференція з фізики
  напівпровідників, {О}деса, {У}країна. ---
  Т.~1. ---
  Одеса: 2002. ---
  {С.}~80.

\item
Влияние {СВЧ} облучения на остаточный
  уровень внутренних механических
  напряжений и параметры глубоких уровней в
  эпитак-сиальных структурах {G}a{A}s~/
  Р.~В.~Конакова, А.~Б.~Камалов, О.~Я.~Олих
  {и~др.}~// Труды {III} международной
  конференции <<{Р}адиационно--термические
  эффекты и процессы в неорганических
  материалах>>, {Т}омск, {Р}оссия. ---
  Томск: 2002. ---
  {С.}~338--339.

\item
\emph{Оліх,~О.~Я.} Про роль теплових і
  деформаційних механізмів дії ультразвуку
  на роботу кремнієвих сонячних елементів~/
  О.~Я.~Оліх~// Міжнародна науково--технічна
  конференція <<{С}енсорна електроніка і
  мікросистемні технології {СЕМСТ}--1>>,
  {О}деса, {У}країна. Тези доповідей. ---
  Одеса: 2004. ---
  {С.}~163.

\item
\emph{Olikh,~O.} Investigation of microwave treated epitaxial {G}a{A}s
  structures by acoustoelectric method~/ O.~Olikh~// 2004 {IEEE}
  {I}nternational {U}ltrasonics, {F}erroelectrics and {F}requency {C}ontrol
  {J}oint 50$^{th}$ {A}nniversary {C}onference. Montreal, {C}anada. Abstracts.
  ---
  Montreal: 2004. ---
  Pp.~230--231.

\item
\emph{Олих,~О.~Я.} Влияние {СВЧ} облучения на
  остаточный уровень внутренних
  механических напряжений и параметры
  глубоких уровней в эпитак-сиальных
  структурах {G}a{A}s~/ О.~Я.~Олих~// Труды девятой
  международной научно--технической
  конференции <<{А}ктуальные проблемы
  твердотельной электроники и
  микроэлектроники>>, {Д}ивноморское,
  {Р}оссия. ---
  Дивноморское: 2004. ---
  {С.}~278--279.

\item
Influence of acoustic wave on forming and characteristics of silicon p--n
  junction~/ J.~Olikh, A.~Evtukh, B.~Romanyuk, O.~Olikh~// 2005 {IEEE}
  {I}nternational {U}ltrasonics {S}ymposium and {S}hort {C}ourses. Rotterdam,
  {N}etherlands. Abstracts. ---
  Rotterdam: 2005. ---
  P.~542.

\item
\emph{Olikh,~O.} Dynamic ultrasound effects in silicon solar sell~/
  O.~Olikh, R.~Burbelo, Hinders~M.~// 2007 {I}nternational {C}ongress on
  {U}ltrasonics. {P}rogram and {B}ook of {A}bstracts. {V}ienna, {A}ustria. ---
  Vienna: 2007. ---
  P.~94.

\item
\emph{Olikh,~O.} Influence of the ultrasound treatment on
  {A}u-{T}i{B}--n--n$^+$--{G}a{A}s structure electrical properties~/
  O.~Olikh~// 2007 {I}nternational {C}ongress on {U}ltrasonics. {P}rogram and
  {B}ook of {A}bstracts. {V}ienna, {A}ustria. ---
  Vienna: 2007. ---
  P.~94.

\item
\emph{Olikh,~O.} The Dynamic Ultrasound In-fluence on Diffusion and Drift of
  the Charge Carriers in Silicon p--n Structures~/ O.~Olikh, R.~Burbelo,
  M.~Hinders~// {MRS} 2007 {S}pring {M}eeting, {S}ymposium {F}: {S}emiconductor
  {D}efect {E}ngineering --- {M}aterials, {S}ynthetic {S}tructures, and
  {D}evices {II}. San {F}rancisco, {USA}. ---
  San {F}rancisco: 2007. ---
  P.~3.11.

\item
\emph{Оліх,~О.~Я.} Робота кремнієвих сонячних
  елементів в умовах акустичного
  навантаження мегагерцового діапазону~/
  О.~Я.~Оліх~// {ІІІ} {У}країнська наукова
  конференція з фізики напівпровідників
  {УНКФН}--3, {О}деса, {У}країна. Тези доповідей.
  ---
  Одеса: 2007. ---
  {С.}~322.

\item
\emph{Оліх,~О.~Я.} Вплив ультразвукової
  обробки на вольт--амперні характеристики
  опромінених кремнієвих структур~/
  О.~Я.~Оліх, А.~М.~Горб~// {VІ} {М}іжнародна
  школа--конференція <<Актуальні проблеми
  фізики напівпровідників>>, {Д}рогобич,
  {У}країна. Тези доповідей. ---
  Дрогобич: 2008. ---
  {С.}~114.

\item
\emph{Оліх,~О.~Я.} Акустичні збурення
  дефектної підсистеми кремнієвих
  p--n--структур~/ О.~Я.~Оліх~// {VІ} {М}іжнародна
  школа--конференція <<Актуальні проблеми
  фізики напівпровідників>>, {Д}рогобич,
  {У}країна. Тези доповідей. ---
  Дрогобич: 2008. ---
  {С.}~174.


\item
\emph{Оліх,~О.~Я.} Особливості механізму
  ультразвукового впливу на
  фото--електричний струм у
  нейтронно--опромінених {S}i--p--n--структурах~/
  О.~Я.~Оліх~// {IV} {У}країнська наукова
  конференція з фізики напівпровідників,
  {З}апоріжжя, {У}країна. Тези доповідей. ---
  Т.~2. ---
  {З}апоріжжя: 2009. ---
  {С.}~59.

  \item
\emph{Olikh,~O.} Ultrasound influence on the recombination centers in
  silicon p-n--structures~/ O.~Olikh~// 13th International Conference on
  Defects --- Recognition, Imaging and Physics in Semiconductors. Wheeling,
  {USA}. Final program. ---
Wheeling: 2009. ---
Pp.~9--10.


\item
\emph{Оліх,~Я.~М.} Про можливості практично-го
  застосування ультразвуку для керування
  характеристиками перетворювачів
  сонячної енергії~/ Я.~М.~Оліх, О.~Я.~Оліх~//
  Четверта міжнародна науково--практична
  конференція <<Матеріали електронної
  техніки та сучасні інформаційні
  технології>>, {К}ременчук, {У}країна. Тези
  доповідей. ---
  {К}ременчук: 2010. ---
  {С.}~147--148.

\item
\emph{Оліх,~О.~Я.} Немонотонний вплив
  $\gamma$--опромінення на електричні
  властивості кремнієвих структур з
  бар’єром {Ш}отки~/ О.~Я.~Оліх, С.~В.~Онисюк~//
  {VІI} {М}іжнародна школа--конференція
  <<Актуальні проблеми фізики
  напівпровідників>>, {Д}рогобич, {У}країна.
  Тези доповідей. ---
  Дрогобич: 2010. ---
  {С.}~171--172.

\item
\emph{Оліх,~О.~Я.} Особливості динамічного
  ультразвукового впливу на
  $\gamma$--опромінені кремнієві $m-s-$структури~/
  О.~Я.~Оліх, С.~В.~Онисюк~// Збірник тез {V}
  {У}країнської наукової конференції з
  фізики напівпровідників {УНКФН}--5,
  Ужгород, {У}країна. ---
  Ужгород: 2011. ---
  {С.}~339--340.

\item
\emph{Оліх,~О.~Я.} Вплив ультразвуку на
  термоемісійні процеси в Mo/n--n$^+$--Si
  структурах~/ О.~Я.~Оліх~// Матеріали
  {В}сеукраїнської наукової конференції
  <<Актуальні проблеми теоретичної,
  експериментальної та прикладної фізики>>,
  {Т}ернопіль, {У}країна. ---
  Тернопіль: 2012. ---
  {С.}~101--103.

\item
\emph{Olikh,~O.~Ya.} Reversible Alteration of Reverse Current in Mo/n--Si
  Structures Under Ultrasound Loading~/ O.~Ya.~Olikh, Ya.~M.~Olikh~//
  Фізика і технологія тонких плівок та
  наносистем. {М}атеріали {ХІV} Міжнародної
  конференції~/ {Під ред. }Д.М.~Фреїкa. ---
  Івано--Франківськ: Видавництво
  {П}рикарпатського національного
  університету імені {В}асиля {С}тефаника,
  2013. ---
  {С.}~322.

\item
\emph{Olikh,~O.~Ya.} Modification of reverse current in the Mo/n--Si
  structures under conditions of ultrasonic loading~/ O.~Ya.~Olikh,
  K.~V.~Voytenko~// {VІІI} {I}nternational school--conference <<Actual
  problems of semiconductor physics>>, {D}rohobych, {U}kraine. Abstract book.
  ---
  Drohobych: 2013. ---
  Pp.~101--102.

\item
\emph{Olikh,~Ya.~M.} About acoustical--stimulated a self--organization
  defect structures in semiconductor during ion implantation~/ Ya.~M.~Olikh,
  O.~Ya.~Olikh~// International research and practice conference
  <<Nanotechnology and nanomaterials>>, {B}ukovel, {U}kraine. Abstract book.
  ---
  Bukovel: 2013. ---
  P.~240.

\item
\emph{Оліх,~О.~Я.} Вплив $\gamma$--опромінення на
  механізм перенесення заряду в структурах
  Mo/n--Si~/ О.~Я.~Оліх~// {VІ} {У}країнська наукова
  конференція з фізики напівпровідників
  {УНКФН}--6. Чернівці, {У}країна. Тези
  доповідей. ---
  Чернівці: 2013. ---
  {С.}~121--122.

\item
\emph{Olikh,~Ya.} New approach to ultrasonic absorption in subgrain--free
  {C}d$_{0,2}${H}g$_{0,8}${T}e crystals~/ Ya.~Olikh, I.~Lysyuk, O.~Olikh~//
  2014 {IEEE} {I}nternational {U}ltrasonics {S}ymposium. Chicago, {I}llinois,
  {USA}. Abstract book. ---
  Chicago: 2014. ---
  Pp.~439--440.

\item
\emph{Olikh,~O.} Ultrasonically induced effects in {S}chottky barrier
  structure depending on a $\gamma$--irradiation~/ O.~Olikh~// 2014 {IEEE}
  {I}nternational {U}ltrasonics {S}ymposium. Chicago, {I}llinois, {USA}.
  Abstract book. ---
  Chicago: 2014. ---
  Pp.~645--646.

\item
\emph{Оліх,~О.~Я.} Характеризація
  $\gamma$--опромінених кремнієвих p--n--структур
  методом диференційних коефіцієнтів~/
  О.~Я.~Оліх, О.~В.~Пристай~// 6--та Міжнародна
  науково--технічна конференція <<{С}енсорна
  електроніка і мікросистемні технології>>,
  {О}деса, {У}країна. Тези доповідей. ---
  Одеса: 2014. ---
  {С.}~193.

\item
\emph{Olikh,~O.Ya}. Ultrasonic Loading Effects on Silicon--based Schottky
  Diodes~/ O.Ya~Olikh, K.~V.~Voytenko~// 2015 {I}nternational {C}ongress on
  {U}ltrasonics. Metz, {F}rance. Abstract book. ---
  Metz: 2015. ---
  P.~225.

\item
\emph{Оліх,~О.~Я.} Порівняння ефективності
  методів визначення параметрів діодів
  {Ш}отки~/ О.~Я.~Оліх~// Сучасні проблеми
  фізики конденсованого стану: {П}раці {IV}--ї
  міжнародної конференції. {К}иїв, {У}країна.
  ---
  Київ: 2015. ---
  {С.}~32--34.

\item
Ультразвукова модифікація стимульованого
  фононами тунелювання у кремнієвих діодах
  Шотки~/ О.~Я.~Оліх, К.~В.~Войтенко,
  Р.~М.~Бурбело, Я.~М.~Оліх~// {VІI} {У}країнська
  наукова конференція з фізики
  напівпровідників {УНКФН}--7. Дніпро,
  {У}країна. Тези доповідей. ---
  Дніпро: 2016. ---
  {С.}~190--191.

\item
\emph{Оліх,~О.~Я.} Акусто--керована
  модифікація властивостей кремнієвих
  фотоелектроперетворювачів~/ О.~Я.~Оліх~//
  Перспективні напрямки сучасної
  електроніки, інформаційних і
  комп’ютерних систем. Тези доповідей на
  {ІІ} Всеукраїнській науково--практичній
  конференції {МЕІСS}--2017. Дніпро, {У}країна. ---
  Дніпро: 2017. ---
  {С.}~302--303.
\end{enumerate}


\begin{center}%
\emph{Апробація результатів дисертації}
\end{center}%
\begin{enumerate}[label=\arabic*.,leftmargin=1em,itemindent=1em]

\item
1--а Українська наукова конференція з фізики напівпровідників, Одеса, Україна, 2002~р., очна форма участі.

\item
III международная конференция <<Радиационно--термические эффекты и процессы в неорганических материалах>>, Томск, Россия, 2002~р., заочна форма участі.

\item
Міжнародна науково-технічна конференція <<Сенсорна електроніка і мікросистемні технології СЕМСТ--1>>, Одеса, Україна, 2004~р., очна форма участі.

\item
2004 IEEE International Ultrasonics, Ferroelectrics and Frequency Control Joint 50th Anniversary Conference, Montreal, Canada, 2004~р., очна форма участі.

\item
Девятая международная научно-техническая конференция <<Актуальные проблемы твердотельной электроники и микроэлектроники>>, Дивноморское, Россия, 2004~р., заочна форма участі.

\item
2005 IEEE International Ultrasonics Symposium and Short Courses, Rotterdam, Netherlands, 2005~р., заочна форма участі.

\item
2007 International Congress on Ultrasonics, Vienna, Austria, 2007~р., очна форма участі.

\item
MRS 2007 Spring Meeting, Symposium F: Semiconductor Defect Engineering --- Materials, Synthetic Structures, and Devices II. San Francisco, USA, 2007~р., очна форма участі.

\item
ІІІ Українська наукова конференція з фізики напівпровідників УНКФН--3, Одеса, Україна, 2007~р., заочна форма участі.

\item
VІ Міжнародна школа--конференція <<Актуальні проблеми фізики напівпровідників>>, Дрогобич, Україна, 2008~р., очна форма участі.

\item
ХІІ Міжнародна конференція <<Фізика і технологія тонких плівок та наносистем>>, Івано--Франківськ, Україна, 2009~р., заочна форма участі.

\item
IV Українська наукова конференція з фізики напівпровідників, Запоріжжя, Україна, 2009~р., очна форма участі.

\item
13th International Conference on Defects --- Recognition, Imaging and Physics in Semiconductors, Wheeling, USA, 13--17 вересня, 2009~р., заочна форма участі.

\item
Четверта міжнародна науково--практична конференція <<Матеріали електронної техніки та сучасні інформаційні технології>>, Кременчук, Україна, 2010~р., заочна форма участі.

\item
VІІ Міжнародна школа-конференція <<Актуальні проблеми фізики напівпровідників>>, Дрогобич, Україна, 2010~р., заочна форма участі.

\item
V Українська наукова конференція з фізики напівпровідників УНКФН--6, Ужгород, Україна, 9--15 жовтня, 2011~р., заочна форма участі.

\item
Всеукраїнська наукова конференція <<Актуальні проблеми теоретичної, експериментальної та прикладної фізики>>, Тернопіль, Україна, 2012~р., заочна форма участі.

\item
ХІV Міжнародна конференція <<Фізика і технологія тонких плівок та наносистем>>, Буковель, Україна, 20--25 травня, 2013~р., заочна форма участі.

\item
VІІI Міжнародна школа--конференція <<Актуальні проблеми фізики напівпровідників>>, Дрогобич, Україна, 2013~р., заочна форма участі.

\item
International research and practice conference <<Nanotechnology and nanomaterials>>, Bukovel, Ukraine, 2013~р., заочна форма участі.

\item
VІ Українська наукова конференція з фізики напівпровідників УНКФН--6, Чернівці, Україна, 30 вересня -- 4 жовтня, 2013~р., очна форма участі.

\item
2014 IEEE International Ultrasonics Symposium, Chicago, USA, 2014~р., заочна форма участі.

\item
6--та Міжнародна науково--технічна конференція <<Сенсорна електроніка і мікросистемні технології>>, Одеса, Україна, 2014~р., заочна форма участі.

\item
2015 International Congress on Ultrasonics. Metz, France, 2015~р., очна форма участі.

\item
IV міжнародна конференція <<Сучасні проблеми фізики конденсованого стану>>, Київ, Україна, 2015~р., очна форма участі.

\item
VІІ Українська наукова конференція з фізики напівпровідників УНКФН--7, Дніпро, Україна, 26--30 вересня 2016~р., заочна форма участі.


\item
ІІ Всеукраїнська науково-практична конференція <<Перспективні напрямки 
сучасної електроніки, 
інформаційних і комп'ютерних 
систем>> МЕІСS-2017, Дніпро, Україна, 22--24 листопада 2017~р., заочна форма участі.

\end{enumerate}
%печатаемых сообщениях), он представлен на листинге~\ref{list:hwbeauty}.
%\begin{ListingEnv}[!h]% настройки floating аналогичны окружению figure
%    \captiondelim{ } % разделитель идентификатора с номером от наименования
%    \caption{Программа ,,Hello, world`` на \protect\cpp}
%    % далее метка для ссылки:
%    \label{list:hwbeauty}
%    % окружение учитывает пробелы и табуляции и применяет их в сответсвии с настройками
%    \begin{lstlisting}[language={[ISO]C++}]
%	#include <iostream>
%	using namespace std;
%
%	int main() //кириллица в комментариях при xelatex и lualatex имеет проблемы с пробелами
%	{
%		cout << "Hello, world" << endl; //latin letters in commentaries
%		system("pause");
%		return 0;
%	}
%    \end{lstlisting}
%\end{ListingEnv}%
%Второй не~такой красивый, но без ограничений (см.~листинг~\ref{list:hwplain}).
%\begin{ListingEnv}[!h]
%    \captiondelim{ } % разделитель идентификатора с номером от наименования
%    \caption{Программа ,,Hello, world`` без подсветки}
%    \label{list:hwplain}
%    \begin{Verb}
%
%        #include <iostream>
%        using namespace std;
%
%        int main() //кириллица в комментариях
%        {
%            cout << "Привет, мир" << endl;
%        }
%    \end{Verb}
%\end{ListingEnv}
%
%Можно использовать первый для вставки небольших фрагментов
%внутри текста, а второй для вставки полного
%кода в приложении, если таковое имеется.
%
%Если нужно вставить совсем короткий пример кода (одна или две строки),
%то~выделение  линейками и нумерация может смотреться чересчур громоздко.
%В таких случаях можно использовать окружения \texttt{lstlisting} или
%\texttt{Verb} без \texttt{ListingEnv}. Приведём такой пример
%с указанием языка программирования, отличного от~заданного по умолчанию:
%\begin{lstlisting}[language=Haskell]
%fibs = 0 : 1 : zipWith (+) fibs (tail fibs)
%\end{lstlisting}
%Такое решение~--- со вставкой нумерованных листингов покрупнее
%и вставок без выделения для маленьких фрагментов~--- выбрано,
%например, в книге Эндрю Таненбаума и Тодда Остина по архитектуре
%%компьютера~\autocite{TanAus2013} (см.~рис.~\ref{fig:tan-aus}).
%
%Наконец, для оформления идентификаторов внутри строк
%(функция \lstinline{main} и~тому подобное) используется
%\texttt{lstinline} или, самое простое, моноширинный текст
%(\texttt{\textbackslash texttt}).
%
%
%Пример~\ref{list:internal3}, иллюстрирующий подключение переопределённого языка. Может быть полезным, если подсветка кода работает криво. Без дополнительного окружения, с подписью и ссылкой, реализованной встроенным средством.
%\begingroup
%\captiondelim{ } % разделитель идентификатора с номером от наименования
%\begin{lstlisting}[language={Renhanced},caption={Пример листинга c подписью собственными средствами},label={list:internal3}]
%## Caching the Inverse of a Matrix
%
%## Matrix inversion is usually a costly computation and there may be some
%## benefit to caching the inverse of a matrix rather than compute it repeatedly
%## This is a pair of functions that cache the inverse of a matrix.
%
%## makeCacheMatrix creates a special "matrix" object that can cache its inverse
%
%makeCacheMatrix <- function(x = matrix()) {#кириллица в комментариях при xelatex и lualatex имеет проблемы с пробелами
%    i <- NULL
%    set <- function(y) {
%        x <<- y
%        i <<- NULL
%    }
%    get <- function() x
%    setSolved <- function(solve) i <<- solve
%    getSolved <- function() i
%    list(set = set, get = get,
%    setSolved = setSolved,
%    getSolved = getSolved)
%
%}
%
%
%## cacheSolve computes the inverse of the special "matrix" returned by
%## makeCacheMatrix above. If the inverse has already been calculated (and the
%## matrix has not changed), then the cachesolve should retrieve the inverse from
%## the cache.
%
%cacheSolve <- function(x, ...) {
%    ## Return a matrix that is the inverse of 'x'
%    i <- x$getSolved()
%    if(!is.null(i)) {
%        message("getting cached data")
%        return(i)
%    }
%    data <- x$get()
%    i <- solve(data, ...)
%    x$setSolved(i)
%    i
%}
%\end{lstlisting} %$ %Комментарий для корректной подсветки синтаксиса
%                 %вне листинга
%\endgroup
%
%Листинг~\ref{list:external1} подгружается из внешнего файла. Приходится загружать без окружения дополнительного. Иначе по страницам не переносится.
%\begingroup
%\captiondelim{ } % разделитель идентификатора с номером от наименования
%    \lstinputlisting[lastline=78,language={R},caption={Листинг из внешнего файла},label={list:external1}]{listings/run_analysis.R}
%\endgroup
%
%
%
%
%
%\chapter{Очень длинное название второго приложения, в~котором продемонстрирована работа с~длинными таблицами} \label{AppendixB}
%
% \section{Подраздел приложения}\label{AppendixB1}
%Вот размещается длинная таблица:
%\fontsize{10pt}{10pt}\selectfont
%\begin{longtable*}[c]{|l|c|l|l|} %longtable* появляется из пакета ltcaption и даёт ненумерованную таблицу
%% \caption{Описание входных файлов модели}\label{Namelists}
%%\\
% \hline
% %\multicolumn{4}{|c|}{\textbf{Файл puma\_namelist}}        \\ \hline
% Параметр & Умолч. & Тип & Описание               \\ \hline
%                                              \endfirsthead   \hline
% \multicolumn{4}{|c|}{\small\slshape (продолжение)}        \\ \hline
% Параметр & Умолч. & Тип & Описание               \\ \hline
%                                              \endhead        \hline
%% \multicolumn{4}{|c|}{\small\slshape (окончание)}        \\ \hline
%% Параметр & Умолч. & Тип & Описание               \\ \hline
%%                                             \endlasthead        \hline
% \multicolumn{4}{|r|}{\small\slshape продолжение следует}  \\ \hline
%                                              \endfoot        \hline
%                                              \endlastfoot
% \multicolumn{4}{|l|}{\&INP}        \\ \hline
% kick & 1 & int & 0: инициализация без шума ($p_s = const$) \\
%      &   &     & 1: генерация белого шума                  \\
%      &   &     & 2: генерация белого шума симметрично относительно \\
%  & & & экватора    \\
% mars & 0 & int & 1: инициализация модели для планеты Марс     \\
% kick & 1 & int & 0: инициализация без шума ($p_s = const$) \\
%      &   &     & 1: генерация белого шума                  \\
%      &   &     & 2: генерация белого шума симметрично относительно \\
%  & & & экватора    \\
% mars & 0 & int & 1: инициализация модели для планеты Марс     \\
%kick & 1 & int & 0: инициализация без шума ($p_s = const$) \\
%      &   &     & 1: генерация белого шума                  \\
%      &   &     & 2: генерация белого шума симметрично относительно \\
%  & & & экватора    \\
% mars & 0 & int & 1: инициализация модели для планеты Марс     \\
%kick & 1 & int & 0: инициализация без шума ($p_s = const$) \\
%      &   &     & 1: генерация белого шума                  \\
%      &   &     & 2: генерация белого шума симметрично относительно \\
%  & & & экватора    \\
% mars & 0 & int & 1: инициализация модели для планеты Марс     \\
%kick & 1 & int & 0: инициализация без шума ($p_s = const$) \\
%      &   &     & 1: генерация белого шума                  \\
%      &   &     & 2: генерация белого шума симметрично относительно \\
%  & & & экватора    \\
% mars & 0 & int & 1: инициализация модели для планеты Марс     \\
%kick & 1 & int & 0: инициализация без шума ($p_s = const$) \\
%      &   &     & 1: генерация белого шума                  \\
%      &   &     & 2: генерация белого шума симметрично относительно \\
%  & & & экватора    \\
% mars & 0 & int & 1: инициализация модели для планеты Марс     \\
%kick & 1 & int & 0: инициализация без шума ($p_s = const$) \\
%      &   &     & 1: генерация белого шума                  \\
%      &   &     & 2: генерация белого шума симметрично относительно \\
%  & & & экватора    \\
% mars & 0 & int & 1: инициализация модели для планеты Марс     \\
%kick & 1 & int & 0: инициализация без шума ($p_s = const$) \\
%      &   &     & 1: генерация белого шума                  \\
%      &   &     & 2: генерация белого шума симметрично относительно \\
%  & & & экватора    \\
% mars & 0 & int & 1: инициализация модели для планеты Марс     \\
%kick & 1 & int & 0: инициализация без шума ($p_s = const$) \\
%      &   &     & 1: генерация белого шума                  \\
%      &   &     & 2: генерация белого шума симметрично относительно \\
%  & & & экватора    \\
% mars & 0 & int & 1: инициализация модели для планеты Марс     \\
%kick & 1 & int & 0: инициализация без шума ($p_s = const$) \\
%      &   &     & 1: генерация белого шума                  \\
%      &   &     & 2: генерация белого шума симметрично относительно \\
%  & & & экватора    \\
% mars & 0 & int & 1: инициализация модели для планеты Марс     \\
%kick & 1 & int & 0: инициализация без шума ($p_s = const$) \\
%      &   &     & 1: генерация белого шума                  \\
%      &   &     & 2: генерация белого шума симметрично относительно \\
%  & & & экватора    \\
% mars & 0 & int & 1: инициализация модели для планеты Марс     \\
%kick & 1 & int & 0: инициализация без шума ($p_s = const$) \\
%      &   &     & 1: генерация белого шума                  \\
%      &   &     & 2: генерация белого шума симметрично относительно \\
%  & & & экватора    \\
% mars & 0 & int & 1: инициализация модели для планеты Марс     \\
%kick & 1 & int & 0: инициализация без шума ($p_s = const$) \\
%      &   &     & 1: генерация белого шума                  \\
%      &   &     & 2: генерация белого шума симметрично относительно \\
%  & & & экватора    \\
% mars & 0 & int & 1: инициализация модели для планеты Марс     \\
%kick & 1 & int & 0: инициализация без шума ($p_s = const$) \\
%      &   &     & 1: генерация белого шума                  \\
%      &   &     & 2: генерация белого шума симметрично относительно \\
%  & & & экватора    \\
% mars & 0 & int & 1: инициализация модели для планеты Марс     \\
%kick & 1 & int & 0: инициализация без шума ($p_s = const$) \\
%      &   &     & 1: генерация белого шума                  \\
%      &   &     & 2: генерация белого шума симметрично относительно \\
%  & & & экватора    \\
% mars & 0 & int & 1: инициализация модели для планеты Марс     \\
% \hline
%  %& & & $\:$ \\
% \multicolumn{4}{|l|}{\&SURFPAR}        \\ \hline
%kick & 1 & int & 0: инициализация без шума ($p_s = const$) \\
%      &   &     & 1: генерация белого шума                  \\
%      &   &     & 2: генерация белого шума симметрично относительно \\
%  & & & экватора    \\
% mars & 0 & int & 1: инициализация модели для планеты Марс     \\
%kick & 1 & int & 0: инициализация без шума ($p_s = const$) \\
%      &   &     & 1: генерация белого шума                  \\
%      &   &     & 2: генерация белого шума симметрично относительно \\
%  & & & экватора    \\
% mars & 0 & int & 1: инициализация модели для планеты Марс     \\
%kick & 1 & int & 0: инициализация без шума ($p_s = const$) \\
%      &   &     & 1: генерация белого шума                  \\
%      &   &     & 2: генерация белого шума симметрично относительно \\
%  & & & экватора    \\
% mars & 0 & int & 1: инициализация модели для планеты Марс     \\
%kick & 1 & int & 0: инициализация без шума ($p_s = const$) \\
%      &   &     & 1: генерация белого шума                  \\
%      &   &     & 2: генерация белого шума симметрично относительно \\
%  & & & экватора    \\
% mars & 0 & int & 1: инициализация модели для планеты Марс     \\
%kick & 1 & int & 0: инициализация без шума ($p_s = const$) \\
%      &   &     & 1: генерация белого шума                  \\
%      &   &     & 2: генерация белого шума симметрично относительно \\
%  & & & экватора    \\
% mars & 0 & int & 1: инициализация модели для планеты Марс     \\
%kick & 1 & int & 0: инициализация без шума ($p_s = const$) \\
%      &   &     & 1: генерация белого шума                  \\
%      &   &     & 2: генерация белого шума симметрично относительно \\
%  & & & экватора    \\
% mars & 0 & int & 1: инициализация модели для планеты Марс     \\
%kick & 1 & int & 0: инициализация без шума ($p_s = const$) \\
%      &   &     & 1: генерация белого шума                  \\
%      &   &     & 2: генерация белого шума симметрично относительно \\
%  & & & экватора    \\
% mars & 0 & int & 1: инициализация модели для планеты Марс     \\
%kick & 1 & int & 0: инициализация без шума ($p_s = const$) \\
%      &   &     & 1: генерация белого шума                  \\
%      &   &     & 2: генерация белого шума симметрично относительно \\
%  & & & экватора    \\
% mars & 0 & int & 1: инициализация модели для планеты Марс     \\
%kick & 1 & int & 0: инициализация без шума ($p_s = const$) \\
%      &   &     & 1: генерация белого шума                  \\
%      &   &     & 2: генерация белого шума симметрично относительно \\
%  & & & экватора    \\
% mars & 0 & int & 1: инициализация модели для планеты Марс     \\
% \hline
%\end{longtable*}
%
%\normalsize% возвращаем шрифт к нормальному
%\section{Ещё один подраздел приложения} \label{AppendixB2}
%
%Нужно больше подразделов приложения!
%Конвынёры витюпырата но нам, тебиквюэ мэнтётюм позтюлант ед про. Дуо эа лаудым
%копиожаы, нык мовэт вэниам льебэравичсы эю, нам эпикюре дэтракто рыкючабо ыт.
%
%Пример длинной таблицы с записью продолжения по ГОСТ 2.105:
%
%\begingroup
%    \centering
%    \small
%    \begin{longtable}[c]{|l|c|l|l|}
%    \caption{Наименование таблицы средней длины}%
%    \label{tbl:test5}% label всегда желательно идти после caption
%    \\[-0.45\onelineskip]
%    \hline
%     %\multicolumn{4}{|c|}{\textbf{Файл puma\_namelist}}        \\ \hline
%     Параметр & Умолч. & Тип & Описание\\ \hline
%     \endfirsthead%
%%     \multicolumn{4}{|c|}{\small\slshape (продолжение)}        \\ \hline
%    \caption*{\tabcapalign Продолжение таблицы~\thetable}\\[-0.45\onelineskip]
%    \hline
%     Параметр & Умолч. & Тип & Описание\\ \hline
%      \endhead
%      \hline
%%     \multicolumn{4}{|r|}{\small\slshape продолжение следует}  \\
%%\hline
%     \endfoot
%         \hline
%     \endlastfoot
%     \multicolumn{4}{|l|}{\&INP}        \\ \hline
%     kick & 1 & int & 0: инициализация без шума ($p_s = const$) \\
%          &   &     & 1: генерация белого шума                  \\
%          &   &     & 2: генерация белого шума симметрично относительно \\
%      & & & экватора    \\
%     mars & 0 & int & 1: инициализация модели для планеты Марс     \\
%     kick & 1 & int & 0: инициализация без шума ($p_s = const$) \\
%          &   &     & 1: генерация белого шума                  \\
%          &   &     & 2: генерация белого шума симметрично относительно \\
%      & & & экватора    \\
%     mars & 0 & int & 1: инициализация модели для планеты Марс     \\
%    kick & 1 & int & 0: инициализация без шума ($p_s = const$) \\
%          &   &     & 1: генерация белого шума                  \\
%          &   &     & 2: генерация белого шума симметрично относительно \\
%      & & & экватора    \\
%     mars & 0 & int & 1: инициализация модели для планеты Марс     \\
%    kick & 1 & int & 0: инициализация без шума ($p_s = const$) \\
%          &   &     & 1: генерация белого шума                  \\
%          &   &     & 2: генерация белого шума симметрично относительно \\
%      & & & экватора    \\
%     mars & 0 & int & 1: инициализация модели для планеты Марс     \\
%    kick & 1 & int & 0: инициализация без шума ($p_s = const$) \\
%          &   &     & 1: генерация белого шума                  \\
%          &   &     & 2: генерация белого шума симметрично относительно \\
%      & & & экватора    \\
%     mars & 0 & int & 1: инициализация модели для планеты Марс     \\
%    kick & 1 & int & 0: инициализация без шума ($p_s = const$) \\
%          &   &     & 1: генерация белого шума                  \\
%          &   &     & 2: генерация белого шума симметрично относительно \\
%      & & & экватора    \\
%     mars & 0 & int & 1: инициализация модели для планеты Марс     \\
%    kick & 1 & int & 0: инициализация без шума ($p_s = const$) \\
%          &   &     & 1: генерация белого шума                  \\
%          &   &     & 2: генерация белого шума симметрично относительно \\
%      & & & экватора    \\
%     mars & 0 & int & 1: инициализация модели для планеты Марс     \\
%    kick & 1 & int & 0: инициализация без шума ($p_s = const$) \\
%          &   &     & 1: генерация белого шума                  \\
%          &   &     & 2: генерация белого шума симметрично относительно \\
%      & & & экватора    \\
%     mars & 0 & int & 1: инициализация модели для планеты Марс     \\
%    kick & 1 & int & 0: инициализация без шума ($p_s = const$) \\
%          &   &     & 1: генерация белого шума                  \\
%          &   &     & 2: генерация белого шума симметрично относительно \\
%      & & & экватора    \\
%     mars & 0 & int & 1: инициализация модели для планеты Марс     \\
%    kick & 1 & int & 0: инициализация без шума ($p_s = const$) \\
%          &   &     & 1: генерация белого шума                  \\
%          &   &     & 2: генерация белого шума симметрично относительно \\
%      & & & экватора    \\
%     mars & 0 & int & 1: инициализация модели для планеты Марс     \\
%    kick & 1 & int & 0: инициализация без шума ($p_s = const$) \\
%          &   &     & 1: генерация белого шума                  \\
%          &   &     & 2: генерация белого шума симметрично относительно \\
%      & & & экватора    \\
%     mars & 0 & int & 1: инициализация модели для планеты Марс     \\
%    kick & 1 & int & 0: инициализация без шума ($p_s = const$) \\
%          &   &     & 1: генерация белого шума                  \\
%          &   &     & 2: генерация белого шума симметрично относительно \\
%      & & & экватора    \\
%     mars & 0 & int & 1: инициализация модели для планеты Марс     \\
%    kick & 1 & int & 0: инициализация без шума ($p_s = const$) \\
%          &   &     & 1: генерация белого шума                  \\
%          &   &     & 2: генерация белого шума симметрично относительно \\
%      & & & экватора    \\
%     mars & 0 & int & 1: инициализация модели для планеты Марс     \\
%    kick & 1 & int & 0: инициализация без шума ($p_s = const$) \\
%          &   &     & 1: генерация белого шума                  \\
%          &   &     & 2: генерация белого шума симметрично относительно \\
%      & & & экватора    \\
%     mars & 0 & int & 1: инициализация модели для планеты Марс     \\
%    kick & 1 & int & 0: инициализация без шума ($p_s = const$) \\
%          &   &     & 1: генерация белого шума                  \\
%          &   &     & 2: генерация белого шума симметрично относительно \\
%      & & & экватора    \\
%     mars & 0 & int & 1: инициализация модели для планеты Марс     \\
%     \hline
%      %& & & $\:$ \\
%     \multicolumn{4}{|l|}{\&SURFPAR}        \\ \hline
%    kick & 1 & int & 0: инициализация без шума ($p_s = const$) \\
%          &   &     & 1: генерация белого шума                  \\
%          &   &     & 2: генерация белого шума симметрично относительно \\
%      & & & экватора    \\
%     mars & 0 & int & 1: инициализация модели для планеты Марс     \\
%    kick & 1 & int & 0: инициализация без шума ($p_s = const$) \\
%          &   &     & 1: генерация белого шума                  \\
%          &   &     & 2: генерация белого шума симметрично относительно \\
%      & & & экватора    \\
%     mars & 0 & int & 1: инициализация модели для планеты Марс     \\
%    kick & 1 & int & 0: инициализация без шума ($p_s = const$) \\
%          &   &     & 1: генерация белого шума                  \\
%          &   &     & 2: генерация белого шума симметрично относительно \\
%      & & & экватора    \\
%     mars & 0 & int & 1: инициализация модели для планеты Марс     \\
%    kick & 1 & int & 0: инициализация без шума ($p_s = const$) \\
%          &   &     & 1: генерация белого шума                  \\
%          &   &     & 2: генерация белого шума симметрично относительно \\
%      & & & экватора    \\
%     mars & 0 & int & 1: инициализация модели для планеты Марс     \\
%    kick & 1 & int & 0: инициализация без шума ($p_s = const$) \\
%          &   &     & 1: генерация белого шума                  \\
%          &   &     & 2: генерация белого шума симметрично относительно \\
%      & & & экватора    \\
%     mars & 0 & int & 1: инициализация модели для планеты Марс     \\
%    kick & 1 & int & 0: инициализация без шума ($p_s = const$) \\
%          &   &     & 1: генерация белого шума                  \\
%          &   &     & 2: генерация белого шума симметрично относительно \\
%      & & & экватора    \\
%     mars & 0 & int & 1: инициализация модели для планеты Марс     \\
%    kick & 1 & int & 0: инициализация без шума ($p_s = const$) \\
%          &   &     & 1: генерация белого шума                  \\
%          &   &     & 2: генерация белого шума симметрично относительно \\
%      & & & экватора    \\
%     mars & 0 & int & 1: инициализация модели для планеты Марс     \\
%    kick & 1 & int & 0: инициализация без шума ($p_s = const$) \\
%          &   &     & 1: генерация белого шума                  \\
%          &   &     & 2: генерация белого шума симметрично относительно \\
%      & & & экватора    \\
%     mars & 0 & int & 1: инициализация модели для планеты Марс     \\
%    kick & 1 & int & 0: инициализация без шума ($p_s = const$) \\
%          &   &     & 1: генерация белого шума                  \\
%          &   &     & 2: генерация белого шума симметрично относительно \\
%      & & & экватора    \\
%     mars & 0 & int & 1: инициализация модели для планеты Марс     \\
%%     \hline
%    \end{longtable}
%\normalsize% возвращаем шрифт к нормальному
%\endgroup
%\section{Использование длинных таблиц с окружением \textit{longtabu}} \label{AppendixB2a}
%
%В таблице~\ref{tbl:test-functions} более книжный вариант
%длинной таблицы, используя окружение \verb!longtabu! и разнообразные
%\verb!toprule! \verb!midrule! \verb!bottomrule! из пакета
%\verb!booktabs!. Чтобы визуально таблица смотрелась лучше, можно
%использовать следующие параметры: в самом начале задаётся расстояние
%между строчками с~помощью \verb!arraystretch!. Таблица задаётся на
%всю ширину, \verb!longtabu! позволяет делить ширину колонок
%пропорционально "--- тут три колонки в пропорции 1.1:1:4 "--- для каждой
%колонки первый параметр в описании \verb!X[]!. Кроме того, в~таблице
%убраны отступы слева и справа с помощью \verb!@{}! в
%преамбуле таблицы. К первому и~второму столбцу применяется
%модификатор
%
%\verb!>{\setlength{\baselineskip}{0.7\baselineskip}}!,
%
%\noindent который уменьшает межстрочный интервал в для текста таблиц (иначе
%заголовок второго столбца значительно шире, а двухстрочное имя
%сливается с~окружающими). Для первой и второй колонки текст в ячейках
%выравниваются по~центру как по вертикали, так и по горизонтали "---
%задаётся буквами \verb!m!~и~\verb!c!~в~описании столбца \verb!X[]!.
%
%Так как формулы большие "--- используется окружение \verb!alignedat!,
%чтобы отступ был одинаковый у всех формул "--- он сделан для всех, хотя
%для большей части можно было и не использовать.  Чтобы формулы
%занимали поменьше места в~каждом столбце формулы (где надо)
%используется \verb!\textstyle! "--- он~делает дроби меньше, у знаков
%суммы и произведения "--- индексы сбоку. Иногда формулы слишком большая,
%сливается со следующей, поэтому после неё ставится небольшой
%дополнительный отступ \verb!\vspace*{2ex}!  Для штрафных функций "---
%размер фигурных скобок задан вручную \verb!\Big\{!, т.к. не умеет
%\verb!alignedat! работать с~\verb!\left! и \verb!\right! через
%несколько строк/колонок.
%
%
%В примечании к таблице наоборот, окружение \verb!cases! даёт слишком
%большие промежутки между вариантами, чтобы их уменьшить, в конце
%каждой строчки окружения использовался отрицательный дополнительный
%отступ \verb!\\[-0.5em]!.
%
%
%
%\begingroup % Ограничиваем область видимости arraystretch
%\renewcommand{\arraystretch}{1.6}%% Увеличение расстояния между рядами, для улучшения восприятия.
%\begin{longtabu} to \textwidth
%{%
%@{}>{\setlength{\baselineskip}{0.7\baselineskip}}X[1.1mc]%
%>{\setlength{\baselineskip}{0.7\baselineskip}}X[mc]%
%X[4]@{}%
%}
%        \caption{Тестовые функции для оптимизации, $D$ "---
%          размерность. Для всех функций значение в точке глобального
%          минимума равно нулю.\label{tbl:test-functions}}\\% label всегда желательно идти после caption
%
%        \toprule     %%% верхняя линейка
%        Имя           &Стартовый диапазон параметров &Функция  \\
%        \midrule %%% тонкий разделитель. Отделяет названия столбцов. Обязателен по ГОСТ 2.105 пункт 4.4.5
%        \endfirsthead
%
%        \multicolumn{3}{c}{\small\slshape (продолжение)}        \\
%        \toprule     %%% верхняя линейка
%        Имя           &Стартовый диапазон параметров &Функция  \\
%        \midrule %%% тонкий разделитель. Отделяет названия столбцов. Обязателен по ГОСТ 2.105 пункт 4.4.5
%        \endhead
%
%        \multicolumn{3}{c}{\small\slshape (окончание)}        \\
%        \toprule     %%% верхняя линейка
%        Имя           &Стартовый диапазон параметров &Функция  \\
%        \midrule %%% тонкий разделитель. Отделяет названия столбцов. Обязателен по ГОСТ 2.105 пункт 4.4.5
%        \endlasthead
%
%        \bottomrule %%% нижняя линейка
%        \multicolumn{3}{r}{\small\slshape продолжение следует}  \\
%        \endfoot
%        \endlastfoot
%
%        сфера         &$\left[-100,\,100\right]^D$   &
%        $\begin{aligned}\textstyle f_1(x)=\sum_{i=1}^Dx_i^2\end{aligned}$                                                        \\
%        Schwefel 2.22 &$\left[-10,\,10\right]^D$     &
%        $\begin{aligned}\textstyle f_2(x)=\sum_{i=1}^D|x_i|+\prod_{i=1}^D|x_i|\end{aligned}$                                     \\
%        Schwefel 1.2  &$\left[-100,\,100\right]^D$   &$\begin{aligned}\textstyle f_3(x)=\sum_{i=1}^D\left(\sum_{j=1}^ix_j\right)^2\end{aligned}$                               \\
%        Schwefel 2.21 &$\left[-100,\,100\right]^D$   &$\begin{aligned}\textstyle f_4(x)=\max_i\!\left\{\left|x_i\right|\right\}\end{aligned}$                             \\
%        Rosenbrock    &$\left[-30,\,30\right]^D$     &$\begin{aligned}\textstyle f_5(x)=\sum_{i=1}^{D-1}\left[100\!\left(x_{i+1}-x_i^2\right)^2+(x_i-1)^2\right]\end{aligned}$ \\
%        ступенчатая   &$\left[-100,\,100\right]^D$   &$\begin{aligned}\textstyle f_6(x)=\sum_{i=1}^D\big\lfloor x_i+0.5\big\rfloor^2\end{aligned}$                             \\
%зашумлённая квартическая  &$\left[-1.28,\,1.28\right]^D$ &$\begin{aligned}\textstyle f_7(x)=\sum_{i=1}^Dix_i^4+rand[0,1)\end{aligned}$\vspace*{2ex}\\
%        Schwefel 2.26 &$\left[-500,\,500\right]^D$   &$\begin{aligned}f_8(x)= &\textstyle\sum_{i=1}^D-x_i\,\sin\sqrt{|x_i|}\,+ \\
%                    &\vphantom{\sum}+ D\cdot
%                    418.98288727243369 \end{aligned}$\\
%        Rastrigin     &$\left[-5.12,\,5.12\right]^D$ &
%        $\begin{aligned}\textstyle
%          f_9(x)=\sum_{i=1}^D\left[x_i^2-10\,\cos(2\pi
%            x_i)+10\right]\end{aligned}$\vspace*{2ex}\\
%  Ackley        &$\left[-32,\,32\right]^D$     &$\begin{aligned}f_{10}(x)= &\textstyle -20\, \exp\!\left(-0.2\sqrt{\frac{1}{D}\sum_{i=1}^Dx_i^2} \right)-\\
%                    &\textstyle - \exp\left(\frac{1}{D}\sum_{i=1}^D\cos(2\pi x_i)  \right)  + 20 + e \end{aligned}$ \\
%        Griewank      &$\left[-600,\,600\right]^D$
%        &$\begin{aligned}f_{11}(x)= &\textstyle \frac{1}{4000}
%          \sum_{i=1}^{D}x_i^2 - \prod_{i=1}^D\cos\left(x_i/\sqrt{i}\right) +1     \end{aligned}$ \vspace*{3ex} \\
%        штрафная 1    &$\left[-50,\,50\right]^D$     &
%        $\begin{aligned}f_{12}(x)= &\textstyle \frac{\pi}{D}
%          \Big\{ 10\,\sin^2(\pi y_1) +\\ &+
%          \textstyle \sum_{i=1}^{D-1}(y_i-1)^2\left[1+10\,\sin^2(\pi
%              y_{i+1})\right] +\\ &+(y_D-1)^2 \Big\} +\textstyle\sum_{i=1}^D u(x_i,\,10,\,100,\,4)            \end{aligned}$ \vspace*{2ex} \\
%        штрафная 2    &$\left[-50,\,50\right]^D$     &
%        $\begin{aligned}f_{13}(x)= &\textstyle 0.1
%          \Big\{\sin^2(3\pi x_1) +\\ &+
%          \textstyle \sum_{i=1}^{D-1}(x_i-1)^2\left[1+\sin^2(3 \pi
%              x_{i+1})\right] + \\ &+(x_D-1)^2\left[1+\sin^2(2\pi
%              x_D)\right] \Big\} +\\ &+\textstyle\sum_{i=1}^D u(x_i,\,5,\,100,\,4)            \end{aligned}$               \\
%        сфера         &$\left[-100,\,100\right]^D$   &
%        $\begin{aligned}\textstyle f_1(x)=\sum_{i=1}^Dx_i^2\end{aligned}$                                                        \\
%        Schwefel 2.22 &$\left[-10,\,10\right]^D$     &
%        $\begin{aligned}\textstyle f_2(x)=\sum_{i=1}^D|x_i|+\prod_{i=1}^D|x_i|\end{aligned}$                                     \\
%        Schwefel 1.2  &$\left[-100,\,100\right]^D$   &$\begin{aligned}\textstyle f_3(x)=\sum_{i=1}^D\left(\sum_{j=1}^ix_j\right)^2\end{aligned}$                               \\
%        Schwefel 2.21 &$\left[-100,\,100\right]^D$   &$\begin{aligned}\textstyle f_4(x)=\max_i\!\left\{\left|x_i\right|\right\}\end{aligned}$                             \\
%        Rosenbrock    &$\left[-30,\,30\right]^D$     &$\begin{aligned}\textstyle f_5(x)=\sum_{i=1}^{D-1}\left[100\!\left(x_{i+1}-x_i^2\right)^2+(x_i-1)^2\right]\end{aligned}$ \\
%        ступенчатая   &$\left[-100,\,100\right]^D$   &$\begin{aligned}\textstyle f_6(x)=\sum_{i=1}^D\big\lfloor x_i+0.5\big\rfloor^2\end{aligned}$                             \\
%зашумлённая квартическая  &$\left[-1.28,\,1.28\right]^D$ &$\begin{aligned}\textstyle f_7(x)=\sum_{i=1}^Dix_i^4+rand[0,1)\end{aligned}$\vspace*{2ex}\\
%        Schwefel 2.26 &$\left[-500,\,500\right]^D$   &$\begin{aligned}f_8(x)= &\textstyle\sum_{i=1}^D-x_i\,\sin\sqrt{|x_i|}\,+ \\
%                    &\vphantom{\sum}+ D\cdot
%                    418.98288727243369 \end{aligned}$\\
%        Rastrigin     &$\left[-5.12,\,5.12\right]^D$ &
%        $\begin{aligned}\textstyle
%          f_9(x)=\sum_{i=1}^D\left[x_i^2-10\,\cos(2\pi
%            x_i)+10\right]\end{aligned}$\vspace*{2ex}\\
%  Ackley        &$\left[-32,\,32\right]^D$     &$\begin{aligned}f_{10}(x)= &\textstyle -20\, \exp\!\left(-0.2\sqrt{\frac{1}{D}\sum_{i=1}^Dx_i^2} \right)-\\
%                    &\textstyle - \exp\left(\frac{1}{D}\sum_{i=1}^D\cos(2\pi x_i)  \right)  + 20 + e \end{aligned}$ \\
%        Griewank      &$\left[-600,\,600\right]^D$
%        &$\begin{aligned}f_{11}(x)= &\textstyle \frac{1}{4000}
%          \sum_{i=1}^{D}x_i^2 - \prod_{i=1}^D\cos\left(x_i/\sqrt{i}\right) +1     \end{aligned}$ \vspace*{3ex} \\
%        штрафная 1    &$\left[-50,\,50\right]^D$     &
%        $\begin{aligned}f_{12}(x)= &\textstyle \frac{\pi}{D}
%          \Big\{ 10\,\sin^2(\pi y_1) +\\ &+
%          \textstyle \sum_{i=1}^{D-1}(y_i-1)^2\left[1+10\,\sin^2(\pi
%              y_{i+1})\right] +\\ &+(y_D-1)^2 \Big\} +\textstyle\sum_{i=1}^D u(x_i,\,10,\,100,\,4)            \end{aligned}$ \vspace*{2ex} \\
%        штрафная 2    &$\left[-50,\,50\right]^D$     &
%        $\begin{aligned}f_{13}(x)= &\textstyle 0.1
%          \Big\{\sin^2(3\pi x_1) +\\ &+
%          \textstyle \sum_{i=1}^{D-1}(x_i-1)^2\left[1+\sin^2(3 \pi
%              x_{i+1})\right] + \\ &+(x_D-1)^2\left[1+\sin^2(2\pi
%              x_D)\right] \Big\} +\\ &+\textstyle\sum_{i=1}^D u(x_i,\,5,\,100,\,4)            \end{aligned}$               \\
%        \midrule%%% тонкий разделитель
%        \multicolumn{3}{@{}p{\textwidth}}{%
%            \vspace*{-3.5ex}% этим подтягиваем повыше
%            \hspace*{2.5em}% абзацный отступ - требование ГОСТ 2.105
%            Примечание "---  Для функций $f_{12}$ и $f_{13}$
%            используется $y_i = 1 + \frac{1}{4}(x_i+1)$
%            и~$u(x_i,\,a,\,k,\,m)=\begin{cases}
%k(x_i-a)^m,\quad &x_i >a\\[-0.5em]
%0,\quad &-a\leq x_i \leq a\\[-0.5em]
%k(-x_i-a)^m,\quad &x_i <-a
%\end{cases}$  }   \\        \bottomrule %%% нижняя линейка
%\end{longtabu}
%\endgroup
%
%
%\section{Форматирование внутри таблиц} \label{AppendixB3}
%
%В таблице~\ref{tbl:other-row} пример с чересстрочным
%форматированием. В~файле \verb+userstyles.tex+  задаётся счётчик
%\verb+\newcounter{rowcnt}+ который увеличивается на 1 после каждой
%строчки (как указано в преамбуле таблицы). Кроме того, задаётся
%условный макрос \verb+\altshape+ который выдаёт одно
%из~двух типов форматирования в~зависимости от чётности счётчика.
%
%В таблице~\ref{tbl:other-row} каждая чётная строчка "--- синяя,
%нечётная "--- с наклоном и~слегка поднята вверх. Визуально это приводит
%к тому, что среднее значение и~среднеквадратичное изменение
%группируются и хорошо выделяются взглядом в~таблице. Сохраняется
%возможность отдельные значения в таблице выделить цветом или
%шрифтом. К первому и второму столбцу форматирование не применяется
%по~сути таблицы, к шестому общее форматирование не применяетсся для
%наглядности.
%
%Так как заголовок таблицы тоже считается за строчку, то перед ним (для
%первого, промежуточного и финального варианта) счётчик обнуляется,
%а~в~\verb+\altshape+ для нулевого значения счётчика форматирования
%не~применяется.
%
%
%\begingroup % Ограничиваем область видимости arraystretch
%\renewcommand\altshape{
%  \ifnumequal{\value{rowcnt}}{0}{
%    % Стиль для заголовка таблицы
%  }{
%    \ifnumodd{\value{rowcnt}}
%    {
%      \color{blue} % Cтиль для нечётных строк
%    }{
%      \vspace*{-0.8ex}\itshape} % Стиль для чётных строк
%  }
%}
%\newcolumntype{A}{ >{\altshape}X[1mc]}
%\needspace{2\baselineskip}
%\renewcommand{\arraystretch}{0.9}%% Уменьшаем  расстояние между
%                                %% рядами, чтобы таблица не так много
%                                %% места занимала в дисере.
%\begin{longtabu} to \textwidth {@{}X[0.2ml]X[0.9mc]AAAX[0.99mc]>{\setlength{\baselineskip}{0.7\baselineskip}}AA<{\stepcounter{rowcnt}}@{}}
%% \begin{longtabu} to \textwidth {@{}X[0.2ml]X[1mc]X[1mc]X[1mc]X[1mc]X[1mc]>{\setlength{\baselineskip}{0.7\baselineskip}}X[1mc]X[1mc]@{}}
%  \caption{Длинная таблица с примером чересстрочного форматирования\label{tbl:other-row}}\vspace*{1ex}\\% label всегда желательно идти после caption
%  % \vspace*{1ex}     \\
%
%  \toprule %%% верхняя линейка
%\setcounter{rowcnt}{0} &Итерации & JADE\texttt{++} & JADE & jDE & SaDE
%& DE/rand /1/bin & PSO \\
% \midrule %%% тонкий разделитель. Отделяет названия столбцов. Обязателен по ГОСТ 2.105 пункт 4.4.5
% \endfirsthead
%
% \multicolumn{8}{c}{\small\slshape (продолжение)} \\
% \toprule %%% верхняя линейка
%\setcounter{rowcnt}{0} &Итерации & JADE\texttt{++} & JADE & jDE & SaDE
%& DE/rand /1/bin & PSO \\
% \midrule %%% тонкий разделитель. Отделяет названия столбцов. Обязателен по ГОСТ 2.105 пункт 4.4.5
% \endhead
%
% \multicolumn{8}{c}{\small\slshape (окончание)} \\
% \toprule %%% верхняя линейка
%\setcounter{rowcnt}{0} &Итерации & JADE\texttt{++} & JADE & jDE & SaDE
%& DE/rand /1/bin & PSO \\
% \midrule %%% тонкий разделитель. Отделяет названия столбцов. Обязателен по ГОСТ 2.105 пункт 4.4.5
% \endlasthead
%
% \bottomrule %%% нижняя линейка
% \multicolumn{8}{r}{\small\slshape продолжение следует}     \\
% \endfoot
% \endlastfoot
%
%f1  & 1500 & \textbf{1.8E-60}   & 1.3E-54   & 2.5E-28   & 4.5E-20   & 9.8E-14   & 9.6E-42   \\\nopagebreak
%    &      & (8.4E-60) & (9.2E-54) & \color{red}(3.5E-28) & (6.9E-20) & (8.4E-14) & (2.7E-41) \\
%f2  & 2000 & 1.8E-25   & 3.9E-22   & 1.5E-23   & 1.9E-14   & 1.6E-09   & 9.3E-21   \\\nopagebreak
%    &      & (8.8E-25) & (2.7E-21) & (1.0E-23) & (1.1E-14) & (1.1E-09) & (6.3E-20) \\
%f3  & 5000 & 5.7E-61   & 6.0E-87   & 5.2E-14   & \color{green}9.0E-37   & 6.6E-11   & 2.5E-19   \\\nopagebreak
%    &      & (2.7E-60) & (1.9E-86) & (1.1E-13) & (5.4E-36) & (8.8E-11) & (3.9E-19) \\
%f4  & 5000 & 8.2E-24   & 4.3E-66   & 1.4E-15   & 7.4E-11   & 4.2E-01   & 4.4E-14   \\\nopagebreak
%    &      & (4.0E-23) & (1.2E-65) & (1.0E-15) & (1.8E-10) & (1.1E+00) & (9.3E-14) \\
%f5  & 3000 & 8.0E-02   & 3.2E-01   & 1.3E+01   & 2.1E+01   & 2.1E+00   & 2.5E+01   \\\nopagebreak
%    &      & (5.6E-01) & (1.1E+00) & (1.4E+01) & (7.8E+00) & (1.5E+00) & (3.2E+01) \\
%f6  & 100  & 2.9E+00   & 5.6E+00   & 1.0E+03   & 9.3E+02   & 4.7E+03   & 4.5E+01   \\\nopagebreak
%    &      & (1.2E+00) & (1.6E+00) & (2.2E+02) & (1.8E+02) & (1.1E+03) & (2.4E+01) \\
%f7  & 3000 & 6.4E-04   & 6.8E-04   & 3.3E-03   & 4.8E-03   & 4.7E-03   & 2.5E-03   \\\nopagebreak
%    &      & (2.5E-04) & (2.5E-04) & (8.5E-04) & (1.2E-03) & (1.2E-03) & (1.4E-03) \\
%f8  & 1000 & 3.3E-05   & 7.1E+00   & 7.9E-11   & 4.7E+00   & 5.9E+03   & 2.4E+03   \\\nopagebreak
%    &      & (2.3E-05) & (2.8E+01) & (1.3E-10) & (3.3E+01) & (1.1E+03) & (6.7E+02) \\
%f9  & 1000 & 1.0E-04   & 1.4E-04   & 1.5E-04   & 1.2E-03   & 1.8E+02   & 5.2E+01   \\\nopagebreak
%    &      & (6.0E-05) & (6.5E-05) & (2.0E-04) & (6.5E-04) & (1.3E+01) & (1.6E+01) \\
%f10 & 500  & 8.2E-10   & 3.0E-09   & 3.5E-04   & 2.7E-03   & 1.1E-01   & 4.6E-01   \\\nopagebreak
%    &      & (6.9E-10) & (2.2E-09) & (1.0E-04) & (5.1E-04) & (3.9E-02) & (6.6E-01) \\
%f11 & 500  & 9.9E-08   & 2.0E-04   & 1.9E-05   & 7.8E-04)  & 2.0E-01   & 1.3E-02   \\\nopagebreak
%    &      & (6.0E-07) & (1.4E-03) & (5.8E-05) & (1.2E-03  & (1.1E-01) & (1.7E-02) \\
%f12 & 500  & 4.6E-17   & 3.8E-16   & 1.6E-07   & 1.9E-05   & 1.2E-02   & 1.9E-01   \\\nopagebreak
%    &      & (1.9E-16) & (8.3E-16) & (1.5E-07) & (9.2E-06) & (1.0E-02) & (3.9E-01) \\
%f13 & 500  & 2.0E-16   & 1.2E-15   & 1.5E-06   & 6.1E-05   & 7.5E-02   & 2.9E-03   \\\nopagebreak
%    &      & (6.5E-16) & (2.8E-15) & (9.8E-07) & (2.0E-05) & (3.8E-02) & (4.8E-03) \\
%f1  & 1500 & \textbf{1.8E-60}   & 1.3E-54   & 2.5E-28   & 4.5E-20   & 9.8E-14   & 9.6E-42   \\\nopagebreak
%    &      & (8.4E-60) & (9.2E-54) & \color{red}(3.5E-28) & (6.9E-20) & (8.4E-14) & (2.7E-41) \\
%f2  & 2000 & 1.8E-25   & 3.9E-22   & 1.5E-23   & 1.9E-14   & 1.6E-09   & 9.3E-21   \\\nopagebreak
%    &      & (8.8E-25) & (2.7E-21) & (1.0E-23) & (1.1E-14) & (1.1E-09) & (6.3E-20) \\
%f3  & 5000 & 5.7E-61   & 6.0E-87   & 5.2E-14   & 9.0E-37   & 6.6E-11   & 2.5E-19   \\\nopagebreak
%    &      & (2.7E-60) & (1.9E-86) & (1.1E-13) & (5.4E-36) & (8.8E-11) & (3.9E-19) \\
%f4  & 5000 & 8.2E-24   & 4.3E-66   & 1.4E-15   & 7.4E-11   & 4.2E-01   & 4.4E-14   \\\nopagebreak
%    &      & (4.0E-23) & (1.2E-65) & (1.0E-15) & (1.8E-10) & (1.1E+00) & (9.3E-14) \\
%f5  & 3000 & 8.0E-02   & 3.2E-01   & 1.3E+01   & 2.1E+01   & 2.1E+00   & 2.5E+01   \\\nopagebreak
%    &      & (5.6E-01) & (1.1E+00) & (1.4E+01) & (7.8E+00) & (1.5E+00) & (3.2E+01) \\
%f6  & 100  & 2.9E+00   & 5.6E+00   & 1.0E+03   & 9.3E+02   & 4.7E+03   & 4.5E+01   \\\nopagebreak
%    &      & (1.2E+00) & (1.6E+00) & (2.2E+02) & (1.8E+02) & (1.1E+03) & (2.4E+01) \\
%f7  & 3000 & 6.4E-04   & 6.8E-04   & 3.3E-03   & 4.8E-03   & 4.7E-03   & 2.5E-03   \\\nopagebreak
%    &      & (2.5E-04) & (2.5E-04) & (8.5E-04) & (1.2E-03) & (1.2E-03) & (1.4E-03) \\
%f8  & 1000 & 3.3E-05   & 7.1E+00   & 7.9E-11   & 4.7E+00   & 5.9E+03   & 2.4E+03   \\\nopagebreak
%    &      & (2.3E-05) & (2.8E+01) & (1.3E-10) & (3.3E+01) & (1.1E+03) & (6.7E+02) \\
%f9  & 1000 & 1.0E-04   & 1.4E-04   & 1.5E-04   & 1.2E-03   & 1.8E+02   & 5.2E+01   \\\nopagebreak
%    &      & (6.0E-05) & (6.5E-05) & (2.0E-04) & (6.5E-04) & (1.3E+01) & (1.6E+01) \\
%f10 & 500  & 8.2E-10   & 3.0E-09   & 3.5E-04   & 2.7E-03   & 1.1E-01   & 4.6E-01   \\\nopagebreak
%    &      & (6.9E-10) & (2.2E-09) & (1.0E-04) & (5.1E-04) & (3.9E-02) & (6.6E-01) \\
%f11 & 500  & 9.9E-08   & 2.0E-04   & 1.9E-05   & 7.8E-04)  & 2.0E-01   & 1.3E-02   \\\nopagebreak
%    &      & (6.0E-07) & (1.4E-03) & (5.8E-05) & (1.2E-03  & (1.1E-01) & (1.7E-02) \\
%f12 & 500  & 4.6E-17   & 3.8E-16   & 1.6E-07   & 1.9E-05   & 1.2E-02   & 1.9E-01   \\\nopagebreak
%    &      & (1.9E-16) & (8.3E-16) & (1.5E-07) & (9.2E-06) & (1.0E-02) & (3.9E-01) \\
%f13 & 500  & 2.0E-16   & 1.2E-15   & 1.5E-06   & 6.1E-05   & 7.5E-02   & 2.9E-03   \\\nopagebreak
%    &      & (6.5E-16) & (2.8E-15) & (9.8E-07) & (2.0E-05) & (3.8E-02) & (4.8E-03) \\
%
%    % \vspace*{1ex}     \\
%%         \midrule%%% тонкий разделитель
%%         \multicolumn{3}{@{}p{\textwidth}}{%
%%             % \vspace*{-4ex}% этим подтягиваем повыше
%%             % \hspace*{2.5em}% абзацный отступ - требование ГОСТ 2.105
%%             Примечание "---  Для функций $f_{12}$ и $f_{13}$
%%             используется $y_i = 1 + \frac{1}{4}(x_i+1)$ и
%%             $u(x_i,\,a,\,k,\,m)=\begin{cases}
%% k(x_i-a)^m,\quad  & x_i >a     \\[-0.5em]
%% 0,\quad           & -a\leq x_i \leq a        \\[-0.5em]
%% k(-x_i-a)^m,\quad & x_i <-a
%% \end{cases}$  }     \\
%\bottomrule %%% нижняя линейка
%\end{longtabu} \endgroup
%
%\section{Очередной подраздел приложения} \label{AppendixB4}
%
%Нужно больше подразделов приложения!
%
%\section{И ещё один подраздел приложения} \label{AppendixB5}
%
%Нужно больше подразделов приложения!
%
