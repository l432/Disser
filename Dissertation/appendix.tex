%\appendix
%%% Оформление заголовков приложений ближе к ГОСТ:
\setlength{\midchapskip}{20pt}
\renewcommand*{\afterchapternum}{\par\nobreak\vskip \midchapskip}
\renewcommand\thechapter{\Asbuk{chapter}} % Чтобы приложения русскими буквами нумеровались
   % Предварительные настройки для правильного подключения Приложений
%\chapter{\MakeUppercase{список публікацій за темою дисертації}} \label{AppendixA}
\chapter*{\MakeUppercase{Додатки}}						% Заголовок
\addcontentsline{toc}{chapter}{\MakeUppercase{Додатки}}
\section*{Додаток А. Список публікацій за темою дисертації та відомості про апробацію результатів}


\begin{center}%
\emph{Наукові праці, в яких опубліковано основні наукові результати дисертації}
\end{center}%
%\subsection*{Наукові праці, в яких опубліковано основні наукові результати дисертації}
\begin{enumerate}[label=\arabic*.,leftmargin=1em,itemindent=1em]
%\begin{enumerate}[label=\arabic*.,leftmargin=0em,itemindent=2em]
%\setcounter{enumi}{25}
\item
Acousto--defect interaction in irradiated and non--irradiated silicon
  $n^+$--$p$ structure~/ O.~Ya.~Olikh, A.~M.~Gorb, R.~G.~Chupryna,
  O.~V.~Pristay-Fenenkov~// \emph{J. Appl. Phys.} --- 2018. --- Apr. ---
 Vol. 123, no.~16. --- P.~161573--1--161573--12.

\item
\emph{Olikh,~O.Ya.} Acoustically driven degradation in single crystalline
  silicon solar cell~/ O.Ya.~Olikh~// \emph{Superlattices Microstruct.} ---
  2018. --- May. ---
  Vol. 117. ---
  P.~173--188.

\item
\emph{Olikh,~Oleg}. On the mechanism of ultrasonic loading effect in
  silicon--based {S}chottky diodes~/ Oleg~Olikh, Katerina~Voytenko~//
  \emph{Ultrasonics}. ---
  2016. --- Mar. ---
  Vol.~66, no.~1. ---
  P.~1--3.

\item
Effect of ultrasound on reverse leakage current of silicon {S}chottky barrier
  structure~/ O.~Ya.~Olikh, K.~V.~Voytenko, R.~M.~Burbelo, Ja.~M.~Olikh~//
  \emph{Journal of Semiconductors}. ---
  2016. --- Dec. ---
  Vol.~37, no.~12. ---
  P.~122002--1--122002--7.

\item
\emph{Olikh,~O.~Ya.} Review and test of methods for determination of the
  {S}chottky diode parameters~/ O.~Ya.~Olikh~// \emph{J. Appl. Phys.} ---
  2015. --- Jul. ---
  Vol. 118, no.~2. ---
  P.~024502--1--024502--14.

\item
\emph{Olikh,~O.~Ya.} Ultrasound influence on {I}--{V}--{T} characteristics
  of silicon {S}chottky barrier structure~/ O.~Ya.~Olikh, K.~V.~Voytenko,
  R.~M.~Burbelo~// \emph{J. Appl. Phys.} ---
  2015. --- Jan. ---
  Vol. 117, no.~4. ---
  P.~044505--1--044505--7.

\item
\emph{Olikh,~Oleg}. Reversible influence of ultrasound on
  $\gamma-$irradiated {M}o/n-{S}i {S}chottky barrier structure~/ Oleg~Olikh~//
  \emph{Ultrasonics}. ---
  2015. --- Feb. ---
  Vol.~56. ---
  P.~545--550.

\item
Особливості дислокаційного поглинання
  ультразвуку в безсубблочних кристалах
  {C}d$_{0,2}${H}g$_{0,8}${T}e~/ І.~О.~Лисюк, Я.~М.~Оліх,
  О.~Я.~Оліх, Г.~В.~Бекетов~// \emph{УФЖ}. ---
  2014. ---
  Т.~59, {№}~1. ---
  {С.}~50--57.

\item
\emph{Olikh,~O.~Ya.} Non-Monotonic $\gamma-$Ray Influence on {M}o/n-{S}i
  {S}chottky Barrier Structure Properties~/ O.~Ya.~Olikh~// \emph{IEEE
  Trans. Nucl. Sci.} ---
  2013. --- Feb. ---
  Vol.~60, no.~1. ---
  P.~394--401.

\item
\emph{Оліх,~О.~Я.} Особливості впливу
  ультразвуку на перенесення заряду в
  кремнієвих структурах з бар’єром {Ш}отки
  залежно від дози $\gamma$--опромінення~/
  О.~Я.~Оліх~// \emph{Сенсорна електроніка і
  мікросистемні технології}. ---
  2013. ---
  Т.~10, {№}~1. ---
  {С.}~47--55.

\item
\emph{Олих,~О.~Я.} Влияние ультразвукового
  нагружения на протекание тока в
  структурах {M}o/n--n$^+$--{S}i c барьером {Ш}оттки~/
  О.~Я.~Олих~// \emph{Физика и техника
  полупроводников}. ---
  2013. ---
  Т.~47, {№}~7. ---
  {С.}~979--984.

\item
\emph{Оліх,~О.~Я.} Особливості перенесення
  заряду в структурах {M}o/n--{S}i з бар’єром
  {Ш}отки~/ О.~Я.~Оліх~// \emph{УФЖ}. ---
  2013. ---
  Т.~58, {№}~2. ---
  {С.}~126--134.

\item
\emph{Олих,~О.~Я.} Особенности динамических
  акустоиндуцированных изменений
  фотоэлектрических параметров кремниевых
  солнечных элементов~/ О.~Я.~Олих~//
  \emph{Физика и техника полупроводников}. ---
  2011. ---
  Т.~45, {№}~6. ---
  {С.}~816--822.

\item
\emph{Оліх,~Я.~М.} Інформаційний чинник
  акустичної дії на структуру дефектних
  комплексів у напівпровідниках~/ Я.~М.~Оліх,
  О.~Я.~Оліх~// \emph{Сенсорна електроніка і
  мікросистемні технології}. ---
  2011. ---
  Т. 2(8), {№}~2. ---
  {С.}~5--12.

\item
\emph{Оліх,~О.~Я.} Особливості впливу
  нейтронного опромінення на динамічну
  акустодефектну взаємодію у кремнієвих
  сонячних елементах~/ О.~Я.~Оліх~// \emph{УФЖ}.
  ---
  2010. ---
  Т.~55, {№}~7. ---
  {С.}~770--776.


\item
Ultrasonically Recovered Performance of $\gamma-$Irradiated Metal-Silicon
  Structures~/ A.M.~Gorb, O.A.~Korotchenkov, O.Ya~Olikh, A.O.~Podolian~//
  \emph{IEEE Trans. Nucl. Sci.} ---
  2010. --- June. ---
  Vol.~57, no.~3. ---
  P.~1632--1639.

\item
\emph{Олих,~О.~Я.} Изменение активности
  рекомбинационных центров в кремниевых
  p--n--структурах в условиях акустического
  нагружения~/ О.~Я.~Олих~// \emph{Физика и
  техника полупроводников}. ---
  2009. ---
  Т.~43, {№}~6. ---
  {С.}~774--779.

\item
\emph{Оліх,~О.~Я.} Робота кремнієвих сонячних
  елементів в умовах акустичного
  навантаження мегагерцового діапазону~/
  О.~Я.~Оліх, Р.~М.~Бурбело, М.~К.~Хіндерс~//
  \emph{Сенсорна електроніка і
  мікросистемні технології}. ---
  2007. ---
  Т.~4, {№}~3. ---
  {С.}~40--45.

\item
\emph{Olikh,~O.Ya.} The Dynamic Ultrasound Influence on Diffusion and Drift
  of the Charge Carriers in Silicon p--n Structures~/ O.Ya.~Olikh, R.~Burbelo,
  M.~Hinders~// Semiconductor Defect Engineering --- Materials, Synthetic,
  Structures and Devices II~/ Ed. by S.~Ashok, P.~Kiesel, J.~Chevallier,
  T.~Ogino. ---
  Vol.~994 of \emph{Materials Research Society Symposium
  Proceedings}. ---
  Warrendale, PA: 2007. ---
  P.~269--274.

\item
\emph{Олих,~О.~Я.} Акустостимулированные
  коррекции вольт--амперных характеристик
  арсенид--галлиевых структур с контактом
  {Ш}оттки~/ О.~Я.~Олих, Т.~Н.~Пинчук~//
  \emph{Письма в Журнал Технической Физики}.
  ---
  2006. ---
  Т.~32, {№}~12. ---
  {С.}~22--27.

\item
\emph{Конакова,~Р.В.} Влияние микроволновой
  обработки на уровень остаточной
  деформации и параметры глубоких уровней
  монокристаллах карбида кремния~/
  Р.В.~Конакова, П.М.~Литвин, О.Я.~Олих~//
  \emph{Физика и химия обработки материалов}.
  ---
  2005. ---
  {№}~2. ---
  {С.}~19--22.


\item
\emph{Конакова,~Р.В.} Влияние микроволновой
  обработки на глубокие уровни
  монокристаллов {G}a{A}s и {S}i{C}~/ Р.В.~Конакова,
  П.М.~Литвин, О.Я.~Олих~// \emph{Петербургский
  журнал электроники}. ---
  2004. ---
  {№}~1. ---
  {С.}~20--24.

\item
\emph{Olikh,~Ja.~М.} Active ultrasound effects in the future usage in
  sensor electronics~/ Ja.~М.~Olikh, O.Ya.~Olikh~// \emph{Сенсорна
  електроніка і мікросистемні технології}.
  ---
  2004. ---
  Т.~1, {№}~1. ---
  {С.}~19--29.

\item
\emph{Olikh,~O.Ya.} Acoustoelectric transient spectroscopy of microwave
  treated {G}a{A}s--based structures~/ O.Ya.~Olikh~// \emph{Semiconductor
  Physics, Quantum Electronics \& Optoelectronics}. ---
  2003. ---
  Vol.~6, no.~4. ---
  P.~450--453.

\item
\emph{Оліх,~О.Я.} Акустостимульовані
  динамічні ефекти в сонячних елементах на
  основі кремнію~/ О.Я.~Оліх~// \emph{Вісник
  Київського ун-ту, Сер.: Фізико-математичні
  науки}. ---
  2003. ---
  {№}~4. ---
  {С.}~408--414.
\end{enumerate}

\begin{center}%
\emph{Наукові праці, які засвідчують апробацію матеріалів дисертації}
\end{center}%
\begin{enumerate}[label=\arabic*.,leftmargin=1em,itemindent=1em]
\setcounter{enumi}{25}
\item
\emph{Оліх,~О.~Я.} Ефекти активного
  ультразвуку в напівпровідникових
  кристалах~/ О.~Я.~Оліх~// 1--а {У}країнська
  наукова конференція з фізики
  напівпровідників, {О}деса, {У}країна. ---
  Т.~1. ---
  Одеса: 2002. ---
  {С.}~80.

\item
Влияние {СВЧ} облучения на остаточный
  уровень внутренних механических
  напряжений и параметры глубоких уровней в
  эпитак-сиальных структурах {G}a{A}s~/
  Р.~В.~Конакова, А.~Б.~Камалов, О.~Я.~Олих
  {и~др.}~// Труды {III} международной
  конференции <<{Р}адиационно--термические
  эффекты и процессы в неорганических
  материалах>>, {Т}омск, {Р}оссия. ---
  Томск: 2002. ---
  {С.}~338--339.

\item
\emph{Оліх,~О.~Я.} Про роль теплових і
  деформаційних механізмів дії ультразвуку
  на роботу кремнієвих сонячних елементів~/
  О.~Я.~Оліх~// Міжнародна науково--технічна
  конференція <<{С}енсорна електроніка і
  мікросистемні технології {СЕМСТ}--1>>,
  {О}деса, {У}країна. Тези доповідей. ---
  Одеса: 2004. ---
  {С.}~163.

\item
\emph{Olikh,~O.} Investigation of microwave treated epitaxial {G}a{A}s
  structures by acoustoelectric method~/ O.~Olikh~// 2004 {IEEE}
  {I}nternational {U}ltrasonics, {F}erroelectrics and {F}requency {C}ontrol
  {J}oint 50$^{th}$ {A}nniversary {C}onference. Montreal, {C}anada. Abstracts.
  ---
  Montreal: 2004. ---
  Pp.~230--231.

\item
\emph{Олих,~О.~Я.} Влияние {СВЧ} облучения на
  остаточный уровень внутренних
  механических напряжений и параметры
  глубоких уровней в эпитак-сиальных
  структурах {G}a{A}s~/ О.~Я.~Олих~// Труды девятой
  международной научно--технической
  конференции <<{А}ктуальные проблемы
  твердотельной электроники и
  микроэлектроники>>, {Д}ивноморское,
  {Р}оссия. ---
  Дивноморское: 2004. ---
  {С.}~278--279.

\item
Influence of acoustic wave on forming and characteristics of silicon p--n
  junction~/ J.~Olikh, A.~Evtukh, B.~Romanyuk, O.~Olikh~// 2005 {IEEE}
  {I}nternational {U}ltrasonics {S}ymposium and {S}hort {C}ourses. Rotterdam,
  {N}etherlands. Abstracts. ---
  Rotterdam: 2005. ---
  P.~542.

\item
\emph{Olikh,~O.} Dynamic ultrasound effects in silicon solar sell~/
  O.~Olikh, R.~Burbelo, Hinders~M.~// 2007 {I}nternational {C}ongress on
  {U}ltrasonics. {P}rogram and {B}ook of {A}bstracts. {V}ienna, {A}ustria. ---
  Vienna: 2007. ---
  P.~94.

\item
\emph{Olikh,~O.} Influence of the ultrasound treatment on
  {A}u-{T}i{B}--n--n$^+$--{G}a{A}s structure electrical properties~/
  O.~Olikh~// 2007 {I}nternational {C}ongress on {U}ltrasonics. {P}rogram and
  {B}ook of {A}bstracts. {V}ienna, {A}ustria. ---
  Vienna: 2007. ---
  P.~94.

\item
\emph{Olikh,~O.} The Dynamic Ultrasound In-fluence on Diffusion and Drift of
  the Charge Carriers in Silicon p--n Structures~/ O.~Olikh, R.~Burbelo,
  M.~Hinders~// {MRS} 2007 {S}pring {M}eeting, {S}ymposium {F}: {S}emiconductor
  {D}efect {E}ngineering --- {M}aterials, {S}ynthetic {S}tructures, and
  {D}evices {II}. San {F}rancisco, {USA}. ---
  San {F}rancisco: 2007. ---
  P.~3.11.

\item
\emph{Оліх,~О.~Я.} Робота кремнієвих сонячних
  елементів в умовах акустичного
  навантаження мегагерцового діапазону~/
  О.~Я.~Оліх~// {ІІІ} {У}країнська наукова
  конференція з фізики напівпровідників
  {УНКФН}--3, {О}деса, {У}країна. Тези доповідей.
  ---
  Одеса: 2007. ---
  {С.}~322.

\item
\emph{Оліх,~О.~Я.} Вплив ультразвукової
  обробки на вольт--амперні характеристики
  опромінених кремнієвих структур~/
  О.~Я.~Оліх, А.~М.~Горб~// {VІ} {М}іжнародна
  школа--конференція <<Актуальні проблеми
  фізики напівпровідників>>, {Д}рогобич,
  {У}країна. Тези доповідей. ---
  Дрогобич: 2008. ---
  {С.}~114.

\item
\emph{Оліх,~О.~Я.} Акустичні збурення
  дефектної підсистеми кремнієвих
  p--n--структур~/ О.~Я.~Оліх~// {VІ} {М}іжнародна
  школа--конференція <<Актуальні проблеми
  фізики напівпровідників>>, {Д}рогобич,
  {У}країна. Тези доповідей. ---
  Дрогобич: 2008. ---
  {С.}~174.


\item
\emph{Оліх,~О.~Я.} Особливості механізму
  ультразвукового впливу на
  фото--електричний струм у
  нейтронно--опромінених {S}i--p--n--структурах~/
  О.~Я.~Оліх~// {IV} {У}країнська наукова
  конференція з фізики напівпровідників,
  {З}апоріжжя, {У}країна. Тези доповідей. ---
  Т.~2. ---
  {З}апоріжжя: 2009. ---
  {С.}~59.

  \item
\emph{Olikh,~O.} Ultrasound influence on the recombination centers in
  silicon p-n--structures~/ O.~Olikh~// 13th International Conference on
  Defects --- Recognition, Imaging and Physics in Semiconductors. Wheeling,
  {USA}. Final program. ---
Wheeling: 2009. ---
Pp.~9--10.


\item
\emph{Оліх,~Я.~М.} Про можливості практично-го
  застосування ультразвуку для керування
  характеристиками перетворювачів
  сонячної енергії~/ Я.~М.~Оліх, О.~Я.~Оліх~//
  Четверта міжнародна науково--практична
  конференція <<Матеріали електронної
  техніки та сучасні інформаційні
  технології>>, {К}ременчук, {У}країна. Тези
  доповідей. ---
  {К}ременчук: 2010. ---
  {С.}~147--148.

\item
\emph{Оліх,~О.~Я.} Немонотонний вплив
  $\gamma$--опромінення на електричні
  властивості кремнієвих структур з
  бар’єром {Ш}отки~/ О.~Я.~Оліх, С.~В.~Онисюк~//
  {VІI} {М}іжнародна школа--конференція
  <<Актуальні проблеми фізики
  напівпровідників>>, {Д}рогобич, {У}країна.
  Тези доповідей. ---
  Дрогобич: 2010. ---
  {С.}~171--172.

\item
\emph{Оліх,~О.~Я.} Особливості динамічного
  ультразвукового впливу на
  $\gamma$--опромінені кремнієві $m-s-$структури~/
  О.~Я.~Оліх, С.~В.~Онисюк~// Збірник тез {V}
  {У}країнської наукової конференції з
  фізики напівпровідників {УНКФН}--5,
  Ужгород, {У}країна. ---
  Ужгород: 2011. ---
  {С.}~339--340.

\item
\emph{Оліх,~О.~Я.} Вплив ультразвуку на
  термоемісійні процеси в Mo/n--n$^+$--Si
  структурах~/ О.~Я.~Оліх~// Матеріали
  {В}сеукраїнської наукової конференції
  <<Актуальні проблеми теоретичної,
  експериментальної та прикладної фізики>>,
  {Т}ернопіль, {У}країна. ---
  Тернопіль: 2012. ---
  {С.}~101--103.

\item
\emph{Olikh,~O.~Ya.} Reversible Alteration of Reverse Current in Mo/n--Si
  Structures Under Ultrasound Loading~/ O.~Ya.~Olikh, Ya.~M.~Olikh~//
  Фізика і технологія тонких плівок та
  наносистем. {М}атеріали {ХІV} Міжнародної
  конференції~/ {Під ред. }Д.М.~Фреїкa. ---
  Івано--Франківськ: Видавництво
  {П}рикарпатського національного
  університету імені {В}асиля {С}тефаника,
  2013. ---
  {С.}~322.

\item
\emph{Olikh,~O.~Ya.} Modification of reverse current in the Mo/n--Si
  structures under conditions of ultrasonic loading~/ O.~Ya.~Olikh,
  K.~V.~Voytenko~// {VІІI} {I}nternational school--conference <<Actual
  problems of semiconductor physics>>, {D}rohobych, {U}kraine. Abstract book.
  ---
  Drohobych: 2013. ---
  Pp.~101--102.

\item
\emph{Olikh,~Ya.~M.} About acoustical--stimulated a self--organization
  defect structures in semiconductor during ion implantation~/ Ya.~M.~Olikh,
  O.~Ya.~Olikh~// International research and practice conference
  <<Nanotechnology and nanomaterials>>, {B}ukovel, {U}kraine. Abstract book.
  ---
  Bukovel: 2013. ---
  P.~240.

\item
\emph{Оліх,~О.~Я.} Вплив $\gamma$--опромінення на
  механізм перенесення заряду в структурах
  Mo/n--Si~/ О.~Я.~Оліх~// {VІ} {У}країнська наукова
  конференція з фізики напівпровідників
  {УНКФН}--6. Чернівці, {У}країна. Тези
  доповідей. ---
  Чернівці: 2013. ---
  {С.}~121--122.

\item
\emph{Olikh,~Ya.} New approach to ultrasonic absorption in subgrain--free
  {C}d$_{0,2}${H}g$_{0,8}${T}e crystals~/ Ya.~Olikh, I.~Lysyuk, O.~Olikh~//
  2014 {IEEE} {I}nternational {U}ltrasonics {S}ymposium. Chicago, {I}llinois,
  {USA}. Abstract book. ---
  Chicago: 2014. ---
  Pp.~439--440.

\item
\emph{Olikh,~O.} Ultrasonically induced effects in {S}chottky barrier
  structure depending on a $\gamma$--irradiation~/ O.~Olikh~// 2014 {IEEE}
  {I}nternational {U}ltrasonics {S}ymposium. Chicago, {I}llinois, {USA}.
  Abstract book. ---
  Chicago: 2014. ---
  Pp.~645--646.

\item
\emph{Оліх,~О.~Я.} Характеризація
  $\gamma$--опромінених кремнієвих p--n--структур
  методом диференційних коефіцієнтів~/
  О.~Я.~Оліх, О.~В.~Пристай~// 6--та Міжнародна
  науково--технічна конференція <<{С}енсорна
  електроніка і мікросистемні технології>>,
  {О}деса, {У}країна. Тези доповідей. ---
  Одеса: 2014. ---
  {С.}~193.

\item
\emph{Olikh,~O.Ya}. Ultrasonic Loading Effects on Silicon--based Schottky
  Diodes~/ O.Ya~Olikh, K.~V.~Voytenko~// 2015 {I}nternational {C}ongress on
  {U}ltrasonics. Metz, {F}rance. Abstract book. ---
  Metz: 2015. ---
  P.~225.

\item
\emph{Оліх,~О.~Я.} Порівняння ефективності
  методів визначення параметрів діодів
  {Ш}отки~/ О.~Я.~Оліх~// Сучасні проблеми
  фізики конденсованого стану: {П}раці {IV}--ї
  міжнародної конференції. {К}иїв, {У}країна.
  ---
  Київ: 2015. ---
  {С.}~32--34.

\item
Ультразвукова модифікація стимульованого
  фононами тунелювання у кремнієвих діодах
  Шотки~/ О.~Я.~Оліх, К.~В.~Войтенко,
  Р.~М.~Бурбело, Я.~М.~Оліх~// {VІI} {У}країнська
  наукова конференція з фізики
  напівпровідників {УНКФН}--7. Дніпро,
  {У}країна. Тези доповідей. ---
  Дніпро: 2016. ---
  {С.}~190--191.

\item
\emph{Оліх,~О.~Я.} Акусто--керована
  модифікація властивостей кремнієвих
  фотоелектроперетворювачів~/ О.~Я.~Оліх~//
  Перспективні напрямки сучасної
  електроніки, інформаційних і
  комп’ютерних систем. Тези доповідей на
  {ІІ} Всеукраїнській науково--практичній
  конференції {МЕІСS}--2017. Дніпро, {У}країна. ---
  Дніпро: 2017. ---
  {С.}~302--303.
\end{enumerate}


\begin{center}%
\emph{Апробація результатів дисертації}
\end{center}%
\begin{enumerate}[label=\arabic*.,leftmargin=1em,itemindent=1em]

\item
1--а Українська наукова конференція з фізики напівпровідників, Одеса, Україна, 10--14 вересня, 2002~р., стендова доповідь.

\item
III международная конференция <<Радиационно--термические эффекты и процессы в неорганических материалах>>, Томск, Россия, 29 липня -- 3 серпня, 2002~р., стендова доповідь.

\item
Міжнародна науково-технічна конференція <<Сенсорна електроніка і мікросистемні технології СЕМСТ--1>>, Одеса, Україна, 1--5 червня, 2004~р., усна доповідь.

\item
2004 IEEE International Ultrasonics, Ferroelectrics and Frequency Control Joint 50th Anniversary Conference, Montreal, Canada, 23--27 серпня, 2004~р., стендова доповідь.

\item
Девятая международная научно-техническая конференция <<Актуальные проблемы твердотельной электроники и микроэлектроники>>, Дивноморское, Россия, 12--17 вересня, 2004~р., стендова доповідь.

\item
2005 IEEE International Ultrasonics Symposium and Short Courses, Rotterdam, Netherlands, 19--21 вересня, 2005~р., стендова доповідь.

\item
2007 International Congress on Ultrasonics, Vienna, Austria, 9--12 квітня, 2007~р., стендові доповіді.

\item
MRS 2007 Spring Meeting, Symposium F: Semiconductor Defect Engineering --- Materials, Synthetic Structures, and Devices II. San Francisco, USA, 9--13 квітня, 2007~р., стендова доповідь.

\item
ІІІ Українська наукова конференція з фізики напівпровідників УНКФН--3, Одеса, Україна, 17--22 червня, 2007~р., стендова доповідь.

\item
VІ Міжнародна школа--конференція <<Актуальні проблеми фізики напівпровідників>>, Дрогобич, Україна, 23--26 вересня, 2008~р., стендова доповідь.

\item
IV Українська наукова конференція з фізики напівпровідників, Запоріжжя, Україна, 15--19 вересня, 2009~р., усна доповідь.

\item
13th International Conference on Defects --- Recognition, Imaging and Physics in Semiconductors, Wheeling, USA, 13--17 вересня, 2009~р., стендова доповідь.

\item
Четверта міжнародна науково--практична конференція <<Матеріали електронної техніки та сучасні інформаційні технології>>, Кременчук, Україна, 19--21 травня, 2010~р., стендова доповідь.

\item
VІІ Міжнародна школа-конференція <<Актуальні проблеми фізики напівпровідників>>, Дрогобич, Україна, 28 вересня -- 1 жовтня, 2010~р., усна доповідь.

\item
V Українська наукова конференція з фізики напівпровідників УНКФН--5, Ужгород, Україна, 9--15 жовтня, 2011~р., стендова доповідь.

\item
Всеукраїнська наукова конференція <<Актуальні проблеми теоретичної, експериментальної та прикладної фізики>>, Тернопіль, Україна, 20--22 вересня, 2012~р., стендова доповідь.

\item
ХІV Міжнародна конференція <<Фізика і технологія тонких плівок та наносистем>>, Буковель, Україна, 20--25 травня, 2013~р., стендова доповідь.

\item
VІІI Міжнародна школа--конференція <<Актуальні проблеми фізики напівпровідників>>, Дрогобич, Україна, 25--28 червня, 2013~р., стендова доповідь.

\item
International research and practice conference <<Nanotechnology and nanomaterials>>, Bukovel, Ukraine, 25 серпня -- 1~вересня, 2013~р., стендова доповідь.

\item
VІ Українська наукова конференція з фізики напівпровідників УНКФН--6, Чернівці, Україна, 30 вересня -- 4 жовтня, 2013~р., усна доповідь.

\item
2014 IEEE International Ultrasonics Symposium, Chicago, USA, 3--6 вересня, 2014~р., стендові доповіді.

\item
6--та Міжнародна науково--технічна конференція <<Сенсорна електроніка і мікросистемні технології>>, Одеса, Україна, 29 вересня -- 3 жовтня, 2014~р., стендова доповідь.

\item
2015 International Congress on Ultrasonics. Metz, France, 11--14 травня, 2015~р., стендова доповідь.

\item
IV міжнародна конференція <<Сучасні проблеми фізики конденсованого стану>>, Київ, Україна, 7--10 жовтня, 2015~р., стендова доповідь.

\item
VІІ Українська наукова конференція з фізики напівпровідників УНКФН--7, Дніпро, Україна, 26--30 вересня 2016~р., стендова доповідь.


\item
ІІ Всеукраїнська науково-практична конференція <<Перспективні напрямки
сучасної електроніки,
інформаційних і комп'ютерних
систем>> МЕІСS-2017, Дніпро, Україна, 22--24 листопада 2017~р., стендова доповідь.

\end{enumerate}

%\begin{center}%
%\emph{Апробація результатів дисертації}
%\end{center}%
%\begin{enumerate}[label=\arabic*.,leftmargin=1em,itemindent=1em]
%
%\item
%1--а Українська наукова конференція з фізики напівпровідників, Одеса, Україна, 10--14 вересня, 2002~р., очна форма участі.
%
%\item
%III международная конференция <<Радиационно--термические эффекты и процессы в неорганических материалах>>, Томск, Россия, 29 липня -- 3 серпня, 2002~р., заочна форма участі.
%
%\item
%Міжнародна науково-технічна конференція <<Сенсорна електроніка і мікросистемні технології СЕМСТ--1>>, Одеса, Україна, 1--5 червня, 2004~р., очна форма участі.
%
%\item
%2004 IEEE International Ultrasonics, Ferroelectrics and Frequency Control Joint 50th Anniversary Conference, Montreal, Canada, 23--27 серпня, 2004~р., очна форма участі.
%
%\item
%Девятая международная научно-техническая конференция <<Актуальные проблемы твердотельной электроники и микроэлектроники>>, Дивноморское, Россия, 12--17 вересня, 2004~р., заочна форма участі.
%
%\item
%2005 IEEE International Ultrasonics Symposium and Short Courses, Rotterdam, Netherlands, 19--21 вересня, 2005~р., заочна форма участі.
%
%\item
%2007 International Congress on Ultrasonics, Vienna, Austria, 9--12 квітня, 2007~р., очна форма участі.
%
%\item
%MRS 2007 Spring Meeting, Symposium F: Semiconductor Defect Engineering --- Materials, Synthetic Structures, and Devices II. San Francisco, USA, 9--13 квітня, 2007~р., очна форма участі.
%
%\item
%ІІІ Українська наукова конференція з фізики напівпровідників УНКФН--3, Одеса, Україна, 17--22 червня, 2007~р., заочна форма участі.
%
%\item
%VІ Міжнародна школа--конференція <<Актуальні проблеми фізики напівпровідників>>, Дрогобич, Україна, 23--26 вересня, 2008~р., очна форма участі.
%
%\item
%IV Українська наукова конференція з фізики напівпровідників, Запоріжжя, Україна, 15--19 вересня, 2009~р., очна форма участі.
%
%\item
%13th International Conference on Defects --- Recognition, Imaging and Physics in Semiconductors, Wheeling, USA, 13--17 вересня, 2009~р., заочна форма участі.
%
%\item
%Четверта міжнародна науково--практична конференція <<Матеріали електронної техніки та сучасні інформаційні технології>>, Кременчук, Україна, 19--21 травня, 2010~р., заочна форма участі.
%
%\item
%VІІ Міжнародна школа-конференція <<Актуальні проблеми фізики напівпровідників>>, Дрогобич, Україна, 28 вересня -- 1 жовтня, 2010~р., заочна форма участі.
%
%\item
%V Українська наукова конференція з фізики напівпровідників УНКФН--5, Ужгород, Україна, 9--15 жовтня, 2011~р., заочна форма участі.
%
%\item
%Всеукраїнська наукова конференція <<Актуальні проблеми теоретичної, експериментальної та прикладної фізики>>, Тернопіль, Україна, 20--22 вересня, 2012~р., заочна форма участі.
%
%\item
%ХІV Міжнародна конференція <<Фізика і технологія тонких плівок та наносистем>>, Буковель, Україна, 20--25 травня, 2013~р., заочна форма участі.
%
%\item
%VІІI Міжнародна школа--конференція <<Актуальні проблеми фізики напівпровідників>>, Дрогобич, Україна, 25--28 червня, 2013~р., заочна форма участі.
%
%\item
%International research and practice conference <<Nanotechnology and nanomaterials>>, Bukovel, Ukraine, 25 серпня -- 1~вересня, 2013~р., заочна форма участі.
%
%\item
%VІ Українська наукова конференція з фізики напівпровідників УНКФН--6, Чернівці, Україна, 30 вересня -- 4 жовтня, 2013~р., очна форма участі.
%
%\item
%2014 IEEE International Ultrasonics Symposium, Chicago, USA, 3--6 вересня, 2014~р., заочна форма участі.
%
%\item
%6--та Міжнародна науково--технічна конференція <<Сенсорна електроніка і мікросистемні технології>>, Одеса, Україна, 29 вересня -- 3 жовтня, 2014~р., заочна форма участі.
%
%\item
%2015 International Congress on Ultrasonics. Metz, France, 11--14 травня, 2015~р., очна форма участі.
%
%\item
%IV міжнародна конференція <<Сучасні проблеми фізики конденсованого стану>>, Київ, Україна, 7--10 жовтня, 2015~р., очна форма участі.
%
%\item
%VІІ Українська наукова конференція з фізики напівпровідників УНКФН--7, Дніпро, Україна, 26--30 вересня 2016~р., заочна форма участі.
%
%
%\item
%ІІ Всеукраїнська науково-практична конференція <<Перспективні напрямки
%сучасної електроніки,
%інформаційних і комп'ютерних
%систем>> МЕІСS-2017, Дніпро, Україна, 22--24 листопада 2017~р., заочна форма участі.
%
%\end{enumerate}
