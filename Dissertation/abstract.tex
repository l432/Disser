\noindent
АНОТАЦІЯ						

\vspace{0.7cm}
\noindent
\thesisAuthorFIO~\thesisTitle. -- Кваліфікаційна наукова праця на правах рукопису.

\vspace{0.7cm}
\noindent
%\abstractBegin
Дисертація на здобуття наукового ступеня \thesisDegree~
за спеціальністю \thesisSpecialtyNumber~<<\thesisSpecialtyTitle>>~
(\thesisKnowledgeNumber~ --- \thesisKnowledgeTitle). --- \thesisOrganization, \thesisCity, \thesisYear.

\vspace{0.7cm}

Напівпровідникові поверхнево--бар'єрні структури на основі кремнію та арсеніду галію є основою мікроелектроніки та сонячної енергетики --- галузей, розвиток яких на сучасному етапі багато в чому визначає загальний прогрес.
Щонайповніше розуміння фізичних процесів, які відбуваються у подібних структурах за різноманітних умов, зокрема при опроміненні чи поширенні пружних деформацій, є необхідною умовою їх ефективного використання та важливою задачею матеріалознавства.
Дисертаційна робота присвячена дослідження фізичних  закономірностей  та встановлення механізмів акусто-- та радіаційноіндукованих ефектів у поверхнево--бар'єрних напівпровідникових структурах, що визначає її  актуальність як з наукової, так і практичної точок зору.


У  роботі представлені результати експериментальних досліджень вперше виявлених оборотних акустоіндукованих ефектів у опромінених та вихідних кремнієвих структурах з  $p$--$n$ переходом (сонячних елементах).
Зокрема виявлено, що при ультразвуковому навантаженні (УЗН, частота ультразвуку $f_\mathtt{US}=4\div8$~МГц, інтенсивність $W_\mathtt{US}\leq0,4$~Вт/см$^2$, температура $290\div340$~К) у вихідних кремнієвих сонячних елементах  відбувається  зменшення густини струму короткого замикання (до 10\%), напруги холостого ходу  (до 15\%) та фактору форми вольт--амперних характеристик (ВАХ) (до 5\%).
Зміни оборотні, величини параметрів  після припинення УЗН  та витримки зразків при кімнатній температурі протягом декількох десятків хвилин повертаються до своїх вихідних значень.
Величини акустоіндукованих змін слабко залежать від температури, водночас при використанні поперечних хвиль зменшення параметрів більш суттєві, ніж у випадку повздовжніх хвиль з тією ж інтенсивністю.
З метою встановлення фізичного механізму виявлених ефектів проведено дослідження впливу УЗН на такі електрофізичні параметри як
ефективний час життя носіїв заряду в області просторового заряду $\tau_{g}$,
фактор неідеальності $n_\mathrm{id}$,
час життя неосновних носіїв в базі діода $\tau_n$
та шунтуючий опір $R_{sh}$.
Встановлено, що УЗН викликає оборотне зростання $n_\mathrm{id}$  (до 0,04) та зменшення $\tau_g$ (до 70\% вихідного значення) і
$\tau_n$ (до 10\% початкової величини).
Шляхом дослідження  впливів інтенсивного ($2$~кВт/м$^2$) довготривалого (до  $15$~год) освітлення
та відпалу (при 200~$^\circ$C) на параметри сонячних елементів,
а також за допомогою
методу диференційним коефіцієнтів вольт--амперних характеристик показано, що
у неопромінених кремнієвих сонячних елементах
дефектами, які приймають участь як у рекомбінаційних процесах, так і у акусто--дефектнiй взаємодiї є кисневмiснi преципiтати (переважно) та
пари Fe$_i$B$_s$ (частково).
Виявлено оборотне акустоіндуковане зменшення величини шунтуючого опору (до 30\%) та показано,
використовуючи модель дислокаційно--індукованого імпеданс, що ефект викликаний зростанням ефективності захоплення електронів лінійними дефектами, розташованими в області $p$--$n$ переходу.


Для пояснення виявлених ефектів запропонована модель акустоактивного комплексного рекомбінаційного центру,
для якого при УЗН змінюється відстань між компонентами.
В рамках запропонованої моделі проведено розрахунки очікуваних акустоіндукованих змін поперечного перерізу захоплення $\sigma_{n}$ та параметру зв'язку,
які визначають темп рекомбінації у наближеннях електронних переходів у системі рівнів спарених дефектів та  Шоклі--Ріда--Хола.
Зокрема
а)~розглянуто ефективність впливу УЗН при збудженні поперечних та повздовжніх хвиль із врахуванням наявності просторово орієнтованих дислокацій та показано, що найбільші зміни очікуються у випадку, коли комплекс складається з компонент міжвузольного та вакансійного типу  в умовах поперечних коливань;
б)~показано, що збільшення $\sigma_{n}$ та зменшення параметра зв'язку має викликати зменшення $\tau_g$ та зростання $n_\mathrm{id}$, що спостерігається на експерименті;
в)~при УЗН має бути справедливим співвідношення $\tau_{n}^{-1}\sim u_{\mathtt{US}}^2$ (де $u_\mathtt{US}$ --- амплітуда зміщень атомів при поширенні ультразвуку), що також спостерігається на експерименті.

Представлені результати досліджень впливу УЗН на властивості кремнієвих сонячних елементів,
опромінених $\gamma$--квантами $^{60}$Co (дози $D$ $10^6$ і $10^7$~рад) та реакторними нейтронами (флюєнс $4\cdot10^{11}$~см$^{-2}$).
Спираючись на температурні залежності акустоіндукованих змін $n_\mathrm{id}$ та $\tau_{g}$ у $\gamma$--опромінених структурах, виявлено, що за умов УЗН відбувається  перебудова метастабільного дефекту (VO$_i$).
Використовуючи залежності $\tau_n(W_\mathtt{US})$,  визначено коефіцієнти, які характеризують взаємодію акустичних хвиль з радіаційними дефектами (для C$_i$O$_i$ $K_\mathtt{US}^\mathtt{CO}=0$, тобто цей дефект не є акустоактивним,
для дивакансії $K_\mathtt{US}=(42\pm15)$~см$^2$~Вт$^{-1}$)
та кисневмiсними преципiтатами ($K_\mathtt{US}>5$~см$^2$~Вт$^{-1}$).

Проведено порівняльний аналіз та оптимізації методів розрахунку параметрів (струму насичення  $I_s$, висоти бар'єру Шотткі  $\Phi_b$), фактора неідеальності та послідовного опору) структур метал--напівпровідник з вольт--амперних характеристик.
Були розглянуті 10 аналітичних методів (використовують інтегрування ВАХ (метод Kaminski І), побудову різноманітних допоміжних функцій (чи їх масиву) та лінійну (методи Chung, Lee та Kaminski ІІ) чи нелінійну (Gromov) апроксимацію або пошук екстремумів (Cibils);
також для побудови функцій застосовують додаткові параметри (методи Norde та Bohlin) або диференційні коефіцієнти першого (Werner) або вищого порядків (Mikhelashvili))
2 чисельних методи (метод найменших квадратів зі статичними ваговими коефіцієнтами застосовувався безпосередньо до рівняння ВАХ та до його розв'язку, вираженого через $W$--функцію Ламберта) та
4 еволюційних алгоритми (диференційної еволюції (DE),
оптимізації зграї частинок (PSO),
модифікованої штучної бджолиної сім'ї (MABC) та
оптимізованого викладання та навчання (TLBO)).
Для методів Norde та Bohlin визначені  оптимальні (для кремнієвих діодів Шотткі при вимірюваннях в діапазоні температур $130\div330$~К) величини додаткових параметрів (1,8 для Norde та 1,6 і 3,5 для Bohlin).
Запропоновано модифікацію методу Mikhelashvili, яка дозволяє застосовувати його в автоматичному режимі до множини ВАХ;
вона полягає у послідовному використанні медіанного фільтру та процедури згладжування функції $\alpha(V)=d(\ln I)/d(\ln V)$ перед визначенням положення її максимуму;
показано доцільність застосування запропонованої процедури при опрацюванні реальних ВАХ для підвищення точності методу.
Запропоновано адаптивну процедуру вибору діапазону ВАХ, який використовується для побудови допоміжних функцій при застосуванні аналітичних методів визначення параметрів та показано, що вона дозволяє підвищити точність визначення параметрів (приблизно на порядок при кімнатних температурах у випадку низького рівня похибок вимірювання) і не викликає критичного збільшення часу розрахунку.
Проведено порівняльний аналіз точності  та швидкодії  визначення параметрів різними методами.
Показано, що найбільша точність досягається при використанні еволюцiйних алгоритмів, чисельних методів, методу Gromov з адаптивною процедурою та методу Lee.
Показано, що використання функції Ламберта при застосуванні чисельних методів дозволяє зменшити помилки визначення параметрів.
Визначено вплив абсолютних величин кожного з параметрів на точність визначення $R_s$, $\Phi_b$ та $n_\mathrm{id}$.
Зокрема показано, що еволюційні алгоритми дозволяють отримати найбільш коректні результати при малих (декілька Ом) значеннях $R_s$ або високих температурах, а найбільш стійкими до величин параметрів є точності чисельних методів.

У  роботі представлені результати досліджень
впливу $\gamma$--квантами $^{60}$Co на структури Al$-n-n^+$--Si---Al з контактом Шотткі та
вперше виявлених динамічних акустоіндукованих ефектів в цих структурах при кімнатних температурах.
Виявлено, що при
опроміненні  з дозами $10^6$~рад та $10^7$~рад відбувається
немонотонні зміни висоти бар'єра Шотткі та вперше показано взаємозв'язок характеру немонотонності та ступеня неоднорідності контакту.
Встановлено, що зміна електрофізичних параметрів структур при дозі опромінення $10^6$~рад
пов'язана з накопичення на інтерфейсній границі радіаційних дефектів акцепторного типу та радіаційно--підсиленим дислокаційним ковзанням, що викликає перегрупування патчів.
Вперше виявлено оборотні зменшення висоти бар'єру та збільшення зворотного струму  структур Al$-n-n^+$--Si---Al під дією УЗН при $T=305$~К.
Показано, що в неопромінених структурах акустоіндуковане зменшення висоти бар'єру пов'язане зі зміною рівня нейтральності інтерфейсних станів
внаслідок іонізації дефектів на границі розділу, викликане коливаннями дислокаційних відрізків у акустичному полі.
Опромінення викликає
а)~закріплення сегментів лінійних дефектів внаслідок гетерування точкових дефектів;
б)~появу акустоактивних точкових радіаційних дефектів (А--центри, дивакансії)
що спричинює зміну механізму акусто--дефектної взаємодії.
Показано, що акустоіндуковані зміни зворотного струму пов'язані з впливом пружних хвиль лише на ТЕ складову,
тоді як незмінність при УЗН тунельних струму свідчить, що відповідні дефекти (зокрема, міжвузольні атоми вуглецю) не є акустоактивними.


Досліджено оборотні акустоіндуковані ($f_\mathtt{US}=4,1$, 8,4 та 27,8~МГц) зміни параметрів діодів Шотткі Mo$/n-n^+$--Si в інтервалі температур $130\div330$~К.
Використовуючи для пояснення отриманих експериментальних даних модель неоднорідного контакту з подвійним розподілом Гауса
показано, що ультразвук викликає оборотні збільшення фактора неідеальності та зміни $\Phi_{b}$,
величина і знак яких залежить від температури,
та зростання ефективної густини патчів.
Встановлено, що температурні та частотні залежності акустоіндукованих змін в структурах
 Mo$/n-n^+$--Si  можуть бути пояснені в рамках моделі Брейсфолда,
яка передбачає дифузію дислокаційних перегинів в ультразвуковому полі.
Виявлено акустоіндуковане зростання зворотного струму та показано, що причиною цього ефекту є викликане ультразвуком підсилення емісії електронів з пасток на границі розділу (для термоемісійної складової струму) та зміна розміру дефектних кластерів (для компоненти, пов'язаної з тунелюванням, стимульованим фононами).

 Досліджено вплив надвисокочастотного випромінювання (частота 2,45 ГГц, питома потужність  $1,5$~Вт/см$^2$, час обробки --- до 80~c) на параметри глибоких центрів, розташованих у приповерхневій області монокристалів $n$--6$H$--SiC та $n$--GaAs, а також арсенід галієвих епітаксійних структур за допомогою методу акустоелектричної релаксаційної спектроскопії.
Виявлено, що внаслідок мікрохвильового опромінення біля поверхні збільшується концентрація міжвузольних атомів та відбуваються перетворення в дефектній підсистемі внаслідок їх взаємодії з вихідними дефектами вакансійного типу.
Отримані результати корелюють з вимірами радіуса кривизни структур та деформації в приповерхневому шарі.
Вперше експериментально показано, що ультразвукова обробка структур
Au--TiB$_x$--$n$--$n^+$--GaAs, виготовлених
за технологією з інтегральним тепловідведенням, викликає зменшення розкиду висоти бар'єру, фактора неідеальності та величини зворотного струму для окремих діодів Шотткі.
Досліджено можливість впливу ультразвуковго навантаження при близьких до кімнатних температурах на радіаційні дефекти у $\gamma$--опромінених структурах Si--SiO$_2$--Au.
Показано можливість низькотемпературного акустовідпалу $P_b$--центрів (ненасичених зв'язків на границі Si--SiO$_2$)
та $E'$--центрів (вакансій кисню в діелектричному шарі) внаслідок стимульованої ультразвуком дифузії атомів водню та кисню, відповідно.
Виявлено, що ефект  пасивації атомами водню ненасичених зв'язків залежить
від рівня механічних напруг в околі дефекту.





\vspace{0.7cm}
\noindent
%Ключові слова: \keywords.
\keywords

\vspace{2cm}
%Список публікацій здобувача



%\clearpage
