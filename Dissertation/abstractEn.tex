\noindent
ABSTRACT						
					
\vspace{0.7cm}
\noindent
\thesisAuthorFIOen~\thesisTitleEn. --  Qualification scientific work.

\vspace{0.7cm}
\noindent
%\abstractBegin
Doctor of Science Thesis. Speciality 01.04.07 <<Solid State Physics>> (10 --- Nature science).  --- Taras
Shevchenko National University of Kyiv, Kyiv, \thesisYear.


\vspace{0.7cm}
Semiconductor surface barrier structures, which are originated from silicon and gallium arsenide, are the basis of microelectronics and solar energy, whose development mainly determines overall progress to date.
The more complete understanding of physical processes in similar structures under various conditions, including irradiation or elastic deformation propagation, is both prerequisite of their effective use and important task of material science.
The thesis is devoted to study of physical patterns and to establish of mechanisms of acoustically  and radiation induced phenomena in surface barrier semiconductor structures ant it determines work relevance both from a scientific and practical point of view.

The paper presents the results of experimental studies of the first discovered reversible acoustically induced effects in irradiated and non--irradiated silicon structures with a $p$---$n$--transition (solar cells).
In particular, it was found that ultrasound loading (USL, frequency $f_\mathtt{US}=4\div8$~MHz, intensity $W_\mathtt{US}\leq0,4$~W/cm$^2$, temperature $290\div340$~К ) leads to decrease in the short--circuit current density (down to 10 \%), open--circuit voltage (down to 15\%) and fill factor of voltage--current characteristic (VCC) (down to 5\%) in non-irradiated silicon solar cell.
The changes are reversible, the parameters return to their original values after the USL stopping and sample store at room temperature for several tens of minutes.
The values of acoustically induced changes are slightly dependent on temperature;
at the same time, the transverse waves using leads to more significant parameters reduction as compared to the case of longitudinal waves with the same intensity.
The USL affect on a generation time $\tau_{g}$,
an ideality factor $n_\mathrm{id}$,
an recombination time $\tau_n$
and shunt resistance $R_{sh}$ was carried out for the purpose of determination of physical mechanism of the detected effects.
It is established that the USL causes the reversible both growth of $n_\mathrm{id}$  (up to 0,04) and
decrease of $\tau_g$ (down to 70\% of initial value) and $\tau_n$ (down to 10\%).
It is shown by using intensive ($2$~kW/m$^2$) long--term (up to $15$~hour) illumination,
annealing (200~$^\circ$C) and
CVC differential coefficients  method,
that defects, which take part both in the recombination processes and in the acousto-defect interaction, are oxygen precipitates  (mainly) and
pairs Fe$_i$B$_s$ (partly) in non--irradiated silicon solar cells.
A reversible acoustically induced reduction in the shunt resistance (down to 70\% of initial value) was found;
it was shown by  using the model of the dislocation-induced impedance that effect is caused by the increase in the electron capture efficiency of linear defects located in the $p$---$n$--transition region.


To explain the revealed effects, it was proposed the model of acoustoactive complex recombination center.
The model stipulates the change of distance between complex component under USL action.
The model is used to calculate the expected acoustically induced variation in capture cross section $\sigma_{n}$ and coupled parameter, which determine recombination rate in Shockley–-Read-–Hall model and coupled defect level recombination model.
In particular,
i)~it is considered the efficiency of USL influence in the case both transverse and longitudinal waves and presence of spatially oriented dislocations is taking into account;
it is shown that the largest effect is expected for the complex of interstitial--type and and vacancy--type component under transverse oscillations conditions;
ii)~it is shown that an $\sigma_{n}$ increase and a coupled parameter decrease must cause a decrease in $\tau_g$ and an increase in $n_\mathrm{id}$; such effects are observed experimentally;
iii)~the relation $\tau_{n}^{-1}\sim u_{\mathtt{US}}^2$
(where $u_\mathtt{US}$ is the atom displacement in ultrasound wave)
must be true under USL condition and observed in experiment as well.




The results of investigation of ultrasound influence on the silicon solar cells,
irradiated by $\gamma$--rays (doses $D$ $10^6$ or $10^7$~rad) or reactor neutrons (fluence $4\cdot10^{11}$~cm$^{-2}$),
are presented too.
By using the temperature dependencies of acoustically induced change of $n_\mathrm{id}$ and $\tau_{g}$ in $\gamma$--irradiated structures, it is revealed rebuilding of metastable defect (VO$_i$) under USL action.
The coefficients, which characterizes the ultrasound interaction with the radiation defect and oxygen precipitates,  are determined by using dependencies $\tau_n(W_\mathtt{US})$:
for C$_i$O$_i$ $K_\mathtt{US}^\mathtt{CO}=0$ (defect is non--acoustically active),
for divacancy $K_\mathtt{US}=(42\pm15)$~cm$^2$~W$^{-1}$,
for oxygen precipitates $K_\mathtt{US}>5$~cm$^2$~W$^{-1}$.


The comparative analysis and optimization of the methods for metal--semiconductor structure parameters (saturation current $I_s$, Schottky barrier height $\Phi_b$, ideality factor and series resistance) calculating from VCC are done.
Ten analytical method was under consideration.
They based on VCC integration (Kaminski І method),
single (or set of) auxiliary function and linear (Chung, Lee and Kaminski ІІ methods) or non--linear (Gromov) approximation or extreme search  (Cibils);
 auxiliary function with (Norde and Bohlin methods) or differential coefficient of the first (Werner) or higher order (Mikhelashvili).
Besides, two numerical methods were used:
the standard method of least squares with statistical weights was applied to both VCC equation and solution, expressed by Lambert $W$--function.
At last, four evolutionary algorithms were under consideration: the penalty based differential evolution (DE), the particle swarm optimization (PSO), the modified artificial bee colony (MABC), and the teaching learning
based optimization (TLBO).
The optimal value (Schottky silicon diodes, temperature range $130\div330$~K) of adding parameters were determined
for Norde and Bohlin methods (1,8 and (1,6 and 3,5), respectively).
The Mikhelashvili method modification, which  allows to be used in automatic mode for a set of VCCs, is proposed;
this modification is consecutive using of the median filter and the smoothing procedure for
function $\alpha(V)=d(\ln I)/d(\ln V)$ before maximum position determination;
it is shown the expediency of proposed procedure using for real VCCs to improve the method accuracy.
An adaptive procedure of choosing of VCC range, which is used to auxiliary functions construct in the parameter determination analytical methods, is proposed;
it is shown that it allows to increase the accuracy of the parameters determination (approximately at an order of magnitude at room temperatures in the case of a low level of measurement errors) and does not cause a critical increase in the calculation time.
A comparative analysis of the accuracy and speed of parameters determination by different methods is carried out.
It is shown that the high accuracy is achieved by using the evolutionary algorithms, the numerical methods, the Gromov method with adaptive procedure and the Lee method.
It is shown that the Lambert's function using in numerical methods reduces the errors of parameters determination.
The influence of the each  parameter absolute values on the extraction accuracy of $R_s$, $\Phi_b$ and $n_\mathrm{id}$ is determined.
In particular, it was shown that evolutionary algorithms allow to obtain the most correct results for small (several Ohm) $R_s$ values and for temperatures, while the numerical methods accuracy is the most stable ones.

The paper presents the research results, which concern both influence of $\gamma$--rays $^{60}$Co on Al---$n$--$n^+$--Si structure with Schottky contact and
the firstly detected dynamic acoustically induced effects in such structures at room temperatures.
It is found that irradiation with doses of $10^6$~rad and $10^7$~rad leads to
non--monotonic changes in the Schottky barrier height;
for the first time it was shown the relationship between the non--monotony type and the contact inhomogeneity degree.
It is established that the change in the structure electrophysical parameters  at a radiation dose of $10^6$~rad
 is associated with the accumulation of acceptor--type radiation defects at the interface and
 radiation--enhanced dislocation glide, which causes the patching regroup.
For the first time, a reversible decrease in the barrier height and an increase in the reverse current of the structures Al---$n$--$n^+$--Si under USL action were revealed at $T=305$~K .
It is shown that, the acoustically induced barrier height reduction in the non--irradiated structures is associated with the change in the interface states neutrality level, which is connected to ionization of inner boundary defects, caused by fluctuations of the dislocation segments in the acoustic field.
Irradiation leads to
i)~pinning of linear defect segments as a result of a point defects gettering;
ii)~appearance of acoustically--active point radiation defects (A-centers, divacances)
which causes a change in the acousto--defect interaction mechanism.
It is shown that acoustically induced changes in reverse current are related to the elastic waves influence on the thermionic emission component only,
whereas the invariance of tunneling current under USL action indicates the  non--acoustically activity of respective
defects (in particular, the interstitial carbon atoms).


The reversible acoustically ($f_\mathtt{US}=4,1$, 8,4 and 27,8~MHz)
induced variations in the Mo---$n$--$n^+$--Si Schottky diodes parameters are investigated in the temperature range $130\div330$~К.
It is shown by using a model of inhomogeneous contact with a double Gaussian distribution for the obtained experimental data explanation that ultrasound causes a reversible both increase in the ideality factor,  effective density of patches and the $\Phi_{b}$ change,
the magnitude and sign of which depends on the temperature.
It is established that the temperature and frequency dependences of acoustically induced changes can be explained in the framework of the Brailsford’s model,
which involves diffusion of dislocation kinks in the ultrasound field.
Acoustically induced reverse current increase is found;
it is shown that the effect is deals with the enhancement of emission of electrons from traps at the interface by ultrasound in case of thermionic current component
and with the change of defect clusters size in the case of phonon--assisted tunneling current component.


The influence of microwave irradiation (frequency 2,45~GHz, specific power $1,5$~W/cm$^2$, treatment time --- up to 80~c) on the parameters of deep centers located in the near--surface region of single crystals $n$--6$H$--SiC and $n$--GaAs, as well as gallium arsenide epitaxial structures is investigated by using the acoustoelectric relaxation spectroscopy method.
It was found that microwave irradiation lead to both increase in the interstitial atoms concentration and defective subsystem transformations, which is a a result of interaction interstitial atoms with the initial defects of the vacancy type.
The obtained results correlate with measurements of the structures curvature radius and deformations in the near--surface region.
For the first time it has been experimentally shown that ultrasound treatment of Au--TiB$_x$--$n$--$n^+$--GaAs structures, which  are
madden by technology with integrated heat dissipation, causes a decrease in the dispersion of the barrier height, the ideality factor and the reverse current magnitude for the individual Schottky diodes.
The influence of near room temperature ultrasound loading on radiation defects in $\gamma$--irradiated Au--SiO$_2$--Si structures has been investigated.
It is shown that acoustic treatment result in a low--temperature annealing of both $P_b$--centers (unsaturated bonds at the boundary Si--SiO$_2$) and
$E'$--centers (oxygen vacancies in the dielectric layer), which is deals with ultrasound stimulated diffusion of hydrogen and oxygen atoms, respectively.
It was found that the effect of passivation of unsaturated bonds by hydrogen atoms depends on mechanical stress level in the vicinity of the defect.


\vspace{0.7cm}
\noindent
\keywordsEn.

\vspace{2cm}
%Список публікацій здобувача



%\clearpage
