\noindent
ABSTRACT						
					
\vspace{0.7cm}
\noindent
\thesisAuthorFIOen~\thesisTitleEn. --  Qualification scientific work.

\vspace{0.7cm}
\noindent
%\abstractBegin
Doctor of Science Thesis. Speciality 01.04.07 <<Solid State Physics>> (10 --- Nature science).  --- Taras
Shevchenko National University of Kyiv, Kyiv, \thesisYear.


\vspace{0.7cm}
Semiconductor surface barrier structures, which are originated from silicon and gallium arsenide, are the basis of microelectronics and solar energy, whose development mainly determines overall progress to date.
The more complete understanding of physical processes in similar structures under various conditions, including irradiation or elastic deformation propagation, is both prerequisite of their effective use and important task of material science.
The thesis is devoted to study of physical patterns and to establish of mechanisms of acoustically  and radiation induced phenomena in surface barrier semiconductor structures ant it determines work relevance both from a scientific and practical point of view.

The paper presents the results of experimental studies of the first discovered reversible acoustically induced effects in irradiated and non--irradiated silicon structures with a $p$--$n$ transition (solar cells).
In particular, it was found that ultrasound loading (USL, frequency $f_\mathtt{US}=4\div8$~MHz, intensity $W_\mathtt{US}\leq0,4$~W/cm$^2$, temperature $290\div340$~К ) leads to decrease in the short--circuit current density (down to 10 \%), open--circuit voltage (down to 15\%) and fill factor of voltage--current characteristic (VCC) (down to 5\%) in non-irradiated silicon solar cell.
The changes are reversible, the parameters return to their original values after the USL stopping and sample store at room temperature for several tens of minutes.
The values of acoustically induced changes are slightly dependent on temperature;
at the same time, the transverse waves using leads to more significant parameters reduction as compared to the case of longitudinal waves with the same intensity.
The USL affect on a generation time $\tau_{g}$,
an ideality factor $n_\mathrm{id}$,
an recombination time $\tau_n$
and shunt resistance $R_{sh}$ was carried out for the purpose of determination of physical mechanism of the detected effects.
It is established that the USL causes the reversible both growth of $n_\mathrm{id}$  (up to 0,04) and
decrease of $\tau_g$ (down to 70\% of initial value) and $\tau_n$ (down to 10\%).
It is shown by using intensive ($2$~kW/m$^2$) long--term (up to $15$~hour) illumination,
annealing (200~$^\circ$C) and
CVC differential coefficients  method,
that defects, which take part both in the recombination processes and in the acousto-defect interaction, are oxygen precipitates  (mainly) and
pairs Fe$_i$B$_s$ (partly) in non--irradiated silicon solar cells.
A reversible acoustically induced reduction in the shunt resistance (down to 70\% of initial value) was found;
it was shown by  using the model of the dislocation-induced impedance that effect is caused by the increase in the electron capture efficiency of linear defects located in the $p$--$n$ transition region.


To explain the revealed effects, it was proposed the model of acoustoactive complex recombination center.
The model stipulates the change of distance between complex component under USL action.
The model is used to calculate the expected acoustically induced variation in capture cross section $\sigma_{n}$ and coupled parameter, which determine recombination rate in Shockley–-Read-–Hall model and coupled defect level recombination model.
In particular,
i)~it is considered the efficiency of USL influence in the case both transverse and longitudinal waves and presence of spatially oriented dislocations is taking into account;
it is shown that the largest effect is expected for the complex of interstitial--type and and vacancy--type component under transverse oscillations conditions;
ii)~it is shown that an $\sigma_{n}$ increase and a coupled parameter decrease must cause a decrease in $\tau_g$ and an increase in $n_\mathrm{id}$; such effects are observed experimentally;
iii)~the relation $\tau_{n}^{-1}\sim u_{\mathtt{US}}^2$
(where $u_\mathtt{US}$ is the atom displacement in ultrasound wave)
must be true under USL condition and observed in experiment as well.




The results of investigation of ultrasound influence on the silicon solar cells,
irradiated by $\gamma$--rays (doses $D$ $10^6$ or $10^7$~rad) or reactor neutrons (fluence $4\cdot10^{11}$~cm$^{-2}$),
are presented too.
By using the temperature dependencies of acoustically induced change of $n_\mathrm{id}$ and $\tau_{g}$ in $\gamma$--irradiated structures, it is revealed rebuilding of metastable defect (VO$_i$) under USL action.
The coefficients, which characterizes the ultrasound interaction with the radiation defect and oxygen precipitates,  are determined by using dependencies $\tau_n(W_\mathtt{US})$:
for C$_i$O$_i$ $K_\mathtt{US}^\mathtt{CO}=0$ (defect is non--acoustically active),
for divacancy $K_\mathtt{US}=(42\pm15)$~cm$^2$~W$^{-1}$,
for oxygen precipitates $K_\mathtt{US}>5$~cm$^2$~W$^{-1}$.


The comparative analysis and optimization of the methods for metal--semiconductor structure parameters (saturation current $I_s$, Schottky barrier height $\Phi_b$, ideality factor and series resistance) calculating from VCC are done.
Ten analytical method was under consideration.
They based on VCC integration (Kaminski І method),
single (or set of) auxiliary function and linear (Chung, Lee and Kaminski ІІ methods) or non--linear (Gromov) approximation or extreme search  (Cibils);
 auxiliary function with (Norde and Bohlin methods) or differential coefficient of the first (Werner) or higher order (Mikhelashvili).
Besides, two numerical methods were used:
the standard method of least squares with statistical weights was applied to both VCC equation and solution, expressed by Lambert $W$--function.
At last, four evolutionary algorithms were under consideration: the penalty based differential evolution (DE), the particle swarm optimization (PSO), the modified artificial bee colony (MABC), and the teaching learning
based optimization (TLBO).
The optimal value (Schottky silicon diodes, temperature range $130\div330$~K) of adding parameters were determined
for Norde and Bohlin methods (1,8 and (1,6 and 3,5), respectively).
The Mikhelashvili method modification, which  allows to be used in automatic mode for a set of VCCs, is proposed;
this modification is consecutive using of the median filter and the smoothing procedure for 
function $\alpha(V)=d(\ln I)/d(\ln V)$ before maximum position determination;
it is shown the expediency of proposed procedure using for real VCCs to improve the method accuracy.

Запропоновано адаптивну процедуру вибору діапазону ВАХ, який використовується для побудови допоміжних функцій при застосуванні аналітичних методів визначення параметрів та показано, що вона дозволяє підвищити точність визначення параметрів (приблизно на порядок при кімнатних температурах у випадку низького рівня похибок вимірювання) і не викликає критичного збільшення часу розрахунку.
Проведено порівняльний аналіз точності  та швидкодії  визначення параметрів різними методами.
Показано, що найбільша точність досягається при використанні еволюцiйних алгоритмів, чисельних методів, методу Gromov з адаптивною процедурою та методу Lee.
Показано, що використання функції Ламберта при застосуванні чисельних методів дозволяє зменшити помилки визначення параметрів.
Визначено вплив абсолютних величин кожного з параметрів на точність визначення $R_s$, $\Phi_b$ та $n_\mathrm{id}$.
Зокрема показано, що еволюційні алгоритми дозволяють отримати найбільш коректні результати при малих (декілька Ом) значеннях $R_s$ або високих температурах, а найбільш стійкими до величин параметрів є точності чисельних методів.

У  роботі представлені результати досліджень
впливу $\gamma$--квантами $^{60}$Co на структури Al$-n-n^+$--Si---Al з контактом Шотткі та
вперше виявлених динамічних акустоіндукованих ефектів в цих структурах при кімнатних температурах.
Виявлено, що при
опроміненні  з дозами $10^6$~рад та $10^7$~рад відбувається
немонотонні зміни висоти бар'єра Шотткі та вперше показано взаємозв'язок характеру немонотонності та ступеня неоднорідності контакту.
Встановлено, що зміна електрофізичних параметрів структур при дозі опромінення $10^6$~рад
пов'язана з накопичення на інтерфейсній границі радіаційних дефектів акцепторного типу та радіаційно--підсиленим дислокаційним ковзанням, що викликає перегрупування патчів.
Вперше виявлено оборотні зменшення висоти бар'єру та збільшення зворотного струму  структур Al$-n-n^+$--Si---Al під дією УЗН при $T=305$~К.
Показано, що в неопромінених структурах акустоіндуковане зменшення висоти бар'єру пов'язане зі зміною рівня нейтральності інтерфейсних станів
внаслідок іонізації дефектів на границі розділу, викликане коливаннями дислокаційних відрізків у акустичному полі.
Опромінення викликає
а)~закріплення сегментів лінійних дефектів внаслідок гетерування точкових дефектів;
б)~появу акустоактивних точкових радіаційних дефектів (А--центри, дивакансії)
що спричинює зміну механізму акусто--дефектної взаємодії.
Показано, що акустоіндуковані зміни зворотного струму пов'язані з впливом пружних хвиль лише на ТЕ складову,
тоді як незмінність при УЗН тунельних струму свідчить, що відповідні дефекти (зокрема, міжвузольні атоми вуглецю) не є акустоактивними.


Досліджено оборотні акустоіндуковані ($f_\mathtt{US}=4,1$, 8,4 та 27,8~МГц) зміни параметрів діодів Шотткі Mo$/n-n^+$--Si в інтервалі температур $130\div330$~К.
Використовуючи для пояснення отриманих експериментальних даних модель неоднорідного контакту з подвійним розподілом Гауса
показано, що ультразвук викликає оборотні збільшення фактора неідеальності та зміни $\Phi_{b}$,
величина і знак яких залежить від температури,
та зростання ефективної густини патчів.
Встановлено, що температурні та частотні залежності акустоіндукованих змін в структурах
 Mo$/n-n^+$--Si  можуть бути пояснені в рамках моделі Брейсфолда,
яка передбачає дифузію дислокаційних перегинів в ультразвуковому полі.
Виявлено акустоіндуковане зростання зворотного струму та показано, що причиною цього ефекту є викликане ультразвуком підсилення емісії електронів з пасток на границі розділу (для термоемісійної складової струму) та зміна розміру дефектних кластерів (для компоненти, пов'язаної з тунелюванням, стимульованим фононами).

 Досліджено вплив надвисокочастотного випромінювання (частота 2,45 ГГц, питома потужність  $1,5$~Вт/см$^2$, час обробки --- до 80~c) на параметри глибоких центрів, розташованих у приповерхневій області монокристалів $n$--6$H$--SiC та $n$--GaAs, а також арсенід галієвих епітаксійних структур за допомогою методу акустоелектричної релаксаційної спектроскопії.
Виявлено, що внаслідок мікрохвильового опромінення біля поверхні збільшується концентрація міжвузольних атомів та відбуваються перетворення в дефектній підсистемі внаслідок їх взаємодії з вихідними дефектами вакансійного типу.
Отримані результати корелюють з вимірами радіуса кривизни структур та деформації в приповерхневому шарі.
Вперше експериментально показано, що ультразвукова обробка структур
Au--TiB$_x$--$n$--$n^+$--GaAs, виготовлених
за технологією з інтегральним тепловідведенням, викликає зменшення розкиду висоти бар'єру, фактора неідеальності та величини зворотного струму для окремих діодів Шотткі.
Досліджено можливість впливу ультразвуковго навантаження при близьких до кімнатних температурах на радіаційні дефекти у $\gamma$--опромінених структурах Si--SiO$_2$--Au.
Показано можливість низькотемпературного акустовідпалу $P_b$--центрів (ненасичених зв'язків на границі Si--SiO$_2$)
та $E'$--центрів (вакансій кисню в діелектричному шарі) внаслідок стимульованої ультразвуком дифузії атомів водню та кисню, відповідно.
Виявлено, що ефект  пасивації атомами водню ненасичених зв'язків залежить
від рівня механічних напруг в околі дефекту.

\vspace{0.7cm}
\noindent
\keywordsEn.

\vspace{2cm}
%Список публікацій здобувача



%\clearpage
