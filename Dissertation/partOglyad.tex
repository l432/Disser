%\chapter{Акустоіндуковані ефекти у мікроелектронних структурах та матеріалах\label{Oglyad}}
\chapter{Передумови та перспективи використання активного ультразвуку в мікроелектроніці\label{Oglyad}}




%Явления, связанные с механическими колебаниями упругой среды, широко применяются в микроэлектронике. Прежде всего, речь идет об акустоэлектронике, базирующейся на взаимодействии акустических и электрических сигналов. Существует большое число разнообразных акустоэлектрических фильтров, линий задержки, трансформаторов и других устройств, принцип действия большинства которых основан на использовании пьезоэлектрического эффекта. Но в данной работе мы хотим остановиться на другом аспекте применения акустических волн, а именно рассмотреть возможность их использования в качестве активного инструмента влияния.
%Известно, что рабочие характеристики разнообразных полупроводниковых приборов в значительной мере определяются их дефектным составом. Соответственно, методы позволяющие управлять состоянием дефектов могут служить инструментом как динамического, так и статического управления свойствами таких устройств. Безусловно, наиболее распространенные и изученные возможности для достижения подобных целей – применение высокотемпературных обработок (отжигов), проводимых во многообразных условиях, и облучение частицами различной природы (гамма-квантами, электронами, нейтронами, ядрами атомов) и энергии. С другой стороны, нераспространенным, но перспективным способом активного влияния на свойства кристаллов является использование акустических колебаний ультразвукового (УЗ) диапазона. Среди преимуществ данного подхода можно выделить следующие: а) поглощение УЗ происходит, прежде всего, в областях отклонения от периодичности кристалла, то есть воздействие носит локальный характер и УЗ волны могут быть инструментом, предназначенным именно для работы с дефектами решетки; б) применение волн различной поляризации и типа (как то поверхностных, объемных, Лэмба…) позволяет еще более повысить избирательность влияния; б) путем подбора частоты воздействия внешнего фактора (акустических колебаний) можно достичь резонансных превращений в дефектной подсистеме, то есть получить необходимый результат при низких энергозатратах. 
%Подтверждением возможности эффективного использования УЗ для влияния на дефекты, и, соответственно, свойства полупроводников и приборов на их основе есть накопленный экспериментальный материал. В первую очередь, речь идет об изменениях свойств в результате продолжительного (~103-104 с) возбуждения в кристаллах акустических волн   подобный способ воздействия носит название ультразвуковая обработка (УЗО). Например, было показано, что в результате УЗО могут происходить процессы изменения зарядового состояния центров, что приводит к усилению люминесценции пористого [1] и поликристаллического кремния [2], а также изменению фотопроводимости кристаллов селенида цинка [3]; акустоиндуцированные процессы перестройки и генерации дефектов могут быть причинами изменений примесного поглощения арсенида галлия [4], фоточувствительности и излучательной рекомбинации кристаллов ZnSe [5], спектра фотолюминесценции и коэффициента отражения фулереновых пленок [6]. УЗО может быть причиной и трансформации подсистемы собственных дефектов, как это наблюдалось в GaAs [7]. УЗО также может стимулировать диффузию разнообразных примесей при температурах, близких к комнатным. Например, в работе [8] показано, что благодаря акусто-усилению диффузии водорода происходит улучшение пассивации дефектов границ зерен в поликремнии. Однако, чаще акустодиффузия является причиной изменения свойств поверхности: примером может быть изменение коэффициента поверхностного отражения Si и GaAs [9] или уменьшение плотности поверхностных состояний Si [10] вследствие диффузии легирующей примеси от поверхности полупроводника. Отметим, что направление стимулированного движения примесей может быть и обратным: в работе [11] наблюдалось перемещение атомов К и Na в направлении поверхности кремния. Из других ефектов изменения свойств границ раздел, вызванных УЗО, можно отметить упрочнение поверхностного слоя кремния из-за образования точечных дефектов типа вакансионных и вакансионно-примесных кластеров [12]; уменьшение скорости поверхностной рекомбинации на границе раздела полупроводник-стекло за счет перестройки напряженных валентных связей [13]; или вызванное генерацией дефектов изменение энергетического спектра поверхностных состояний в Si [10].
%УЗО может быть причиной и изменения свойств барьерных полупроводниковых устройств. Например, из литературы известно, что такая обработка может быть причиной улучшения фотоэлектрических параметров (тока короткого замыкания, напряжения холостого хода, коэффициента полезного действия, диапазона спектральной чувствительности) AlGaAs/GaAs солнечных элементов в результате перераспределения атомов примеси, отжига рекомбинационных центров и распада скоплений примесных атомов [14], а также уменьшения концентрации носителей заряда в кремниевых p-n структурах по причине изменения спектра дефектов [15]. Кроме того, УЗО оказывает влияние на величину обратного тока арсенид-галлиевых [16] и кремниевых [17] структур с барьером Шоттки, а также на туннельную составляющую тока в InAs p-n переходах [18].
%Отметим, что УЗО оказывает влияние не только на дефекты, возникающие при выращивании кристаллов или изготовлении структур, но и на дефекты радиационного происхождения. В этом случае может происходить не только низкотемпературный отжиг дефектов такого типа (как это наблюдалось в кристаллах кремния [19] и ионных соединений [20], а также кремниевых солнечных элементах [21]), но и их перестройка, которая приводит к увеличению концентрации генерационных центров в области границы раздела Si-SiO2 [22]; к изменению вольт-амперных характеристик структур метал-полупроводник [23]; к увеличению концентрации носителей заряда в кремнии [24]. Кроме того, после УЗО наблюдалось уменьшение радиационно-индуцированного заряда в диэлектрическом слое МОП структур на основе кремния благодаря стимулированию диффузии радиационных дефектов [25]
%Еще раз подчеркнем, что практически во всех вышеперечисленных примерах речь идет о необратимых изменениях свойств, связанных с необратимыми изменениями подсистемы дефектов в результате действия УЗ колебаний. Исключением являются только результаты работ [18] и [24], авторы которых наблюдали восстановление измененных под действием УЗО свойств после хранения образцов при комнатной температуре на протяжении нескольких суток [24] или месяцев [18]. 
%С одной стороны, приведенные результаты свидетельствуют о том, что возможности УЗ влияния охватывают широкий спектр как полупроводниковых материалов, так и их свойств. С другой стороны, необходимо отметить, что такой способ воздействия не лишен недостатков, а именно, он подразумевает достаточно продолжительное время (103-104 с) обработки (103-104 с) и требует применения акустических волн значительной интенсивности (как правило, интенсивность УЗ должна составлять не меньше 1 Вт/см2). К тому же, полученные результаты часто можно продублировать с использованием более технологически привычных методов, наподобие отжига или облучения частицами. Правда, необходимо отметить, что использование УЗ нередко имеет и свои преимущества, связанные с локализацией воздействия; например, акусто-отжиг радиационных дефектов [19-21] происходит при температурах, недостаточных для существенной диффузии легирующих примеси и размытия профиля легирования, которые могут сопровождать процесс обычного термо-отжига.
%Исходя из вышесказанного, более перспективными для практического применения кажутся возможности использования УЗ а) для динамического, обратимого изменения свойств полупроводниковых устройств во время их работы; б) как фактора дополнительного влияния на структуру (и свойства) материала во время ставших уже классическими радиационной или термической обработки. То есть, по нашему мнению, эффективным будет использование УЗ не как основного инструмента, а параллельного фактора, который накладывается на процессы, происходящие во время функционирования прибора или его обработки. Основанием для такого предположения есть то, что кристаллы в условиях рабочих и технологических температур часто оказываются в неустойчивом состоянии и их дефектно-примесная подсистема более легко может быть перестроена (модифицирована) под действием УЗ. В частности, в этих случаях можно применять УЗ волны значительно меньшей интенсивности, что позволит еще более повысить локализированость воздействие именно на дефектах.
%Первый подход может быть рассмотрен в рамках функциональной электроники – достаточно новой области, суть которой состоит в том, что параметры информационного сигнала управляются динамическими неоднородностями среды, возникающими под действием управляющего сигнала. Именно роль управляющего сигнала и может взять на себя акустическая волна, динамически изменяя состояния дефектов и, таким образом, реализуя возможность создания акусто-управляемых устройств. К примеру, для реализации генератора или частотно-избирательного усилителя применяют транзисторы, резисторы, конденсаторы. При использовании подхода интегральной схемотехники все эти элементы реализуются на основе полупроводников (p-n-переходов). Изменение проводимости полупроводника и, соответственно, номинала елемента (в простейшей ситуации – сопротивления резистора) под воздействием УЗ должно привести к управляемой (напр., интенсивностью УЗ) перестройке рабочей частоты (частоты обратной связи) устройства. Возможность реализации такого подхода подтверждают результаты работы [26], в которой наблюдались динамические изменения концентрации носителей в кремнии при распространении импульсов УЗ.
%Также следует учесть возможности обратимого УЗ влияния на процессы распространения неравновесных носителей. Речь идет об акустоиндуцированной перестройке (и/или перезарядке) дефектов, которые являются центрами рекомбинации (прилипания), что приводит к изменениям сечений захвата и времен жизни носителей. К примеру, в работе [27], приведены результаты исследования акусто-динамических эффектов в кремниевых солнечных элементах. А именно, во время УЗ нагружения наблюдалось увеличение фотоэлектрических параметров, которое авторы связывают, в частности, с увеличением длины диффузии электронов вследствие перезарядки рекомбинационных центров в поле УЗ. Экспериментальному исследованию влияния УЗ на неравновесные носители (в GaAs-фотоприемниках) посвящена и работа [28].
%Если рассматривать акустические волны как часть многокомпонентной обработки, то возбуждение акустических волн во время радиационного облучения кристаллов может быть фактором повышения радиационной стойкости полупроводников благодаря акусто-увеличению скорости собственной рекомбинации (акустоотжигу) первичных радиационных дефектов (РД). Другой путь увеличения радиационной стойкости – образование электрически и рекомбинационно неактивных вторичных РД за счет реакций между первичными РД и примесями; напомним, что процессы диффузии и перестройки РД также акустоактивируемы [23-25]. Кроме того, еще одним положительным фактором влияния УЗ может быть стимуляция перевода внедренных радиационным путем примесей в положение, отвечающее электрически активному состоянию – например ионов-легантов в узельные положения.
%Экспериментальные предпосылки подобного применения УЗ представлены в работах [29-31]. В частности показано, что в случае, когда ионная имплантация сопровождается УЗ нагружением образцов, наблюдается уменьшение дефектов межузельного типа в области обеднения p-n-переходов [29]; усиливается процесс аморфизации поверхностного шара кремния [30]; улучшаются свойства и уменьшается толщина оксида вольфрама, наносимого на поверхность p-Si [31].
%Метод ионной имплантации используется не только для легирования полупроводников или создания поверхностного слоя с определенными свойствами, но и для формирования разнообразных нанокластеров в матрицах различной природы. В этом случае УЗ также может выступать в роли дополнительного фактора воздействия, о чем свидетельствуют результаты работ [32-35]. Например, было показано, что наличие сопровождающего УЗ нагружения приводит к увеличению окончательного размера внедренных металлических кластеров в кремнии [32] и оксиде кремнии [33]; изменению парамагнитных [34] и фотолюминисцентных [35] свойств нанокластеров Si в SiO2.
%УЗ волны могут использоваться не только во время облучения тяжелыми частицами, но и быть частью любой обработки, связанной с изменением дефектной подсистемы. Например, применение УЗ параллельно с лазерной обработкой систем Fe-Si-C приводит к существенному (в 2-3 раза) уменьшению остаточного аустенита по сравнению со случаем использования только лазерного облучения [36].
%Еще одно направление, где эффективно, по нашему мнению, может использоваться УЗ, следующее. Известно, что бистабильный дефект является перспективным элементом памяти нового поколения, две конфигурации которого отвечают логическим «нулю» и «единице». При таком подходе УЗ может выступать инструментом переключения подобного элемента. В частности, гипотетический механизм перевода дефекта между состояниями может быть следующим: при распространении акустических волн изменяет симметрия окружения дефекта (амплитуда колебаний максимальна именно в области отклонения от периодичности, эффект в чем-то напоминающий тепловые пики при радиационном облучении), что приводит к перезарядке центра согласно эффекту Яна-Теллера; часто минимальной энергии в разных зарядовых состояниях отвечают различные пространственные конфигурации, что и приводит к фиксации нового логического состояния. 
%Отметим, что в нашей работе затрагивались только возможности динамического применения УЗ, связанные с акусто-дефектным взаимодействием. В то же время, известно, что акустические волны могут быть причиной и многих других интересных явлений, как то пространственное разделение заряда [37], квантовая телепортация [38], манипуляция наночастицами [39-40].
%Таким образом, применение активного, динамического УЗ в различных областях, связанных с изготовлением и применением различных микроэлектронных устройств является достаточно перспективным и требует тщательного экспериментального и теоретического изучения.





\section{Залишкові акустоіндуковані ефекти}



\cite{Bahar2003,Ostap:PhotoLum,US:ZnSe,Zaver2007,ZobovFTP2008,RITTER2008,Wosinski,Ostapenko1999,Zaver,
Zaver:2008r,Ostrov2002FTPr,Ostrov2000FTPr,Vlasov2009r,Zaver2005,Davletova2008,Olikh:PZTF2006,Tagaev,
Teterkin2009r,PodolHivr,UST:OstrovCsI,YOlikh2007TPLr,Parchinskii2006r,Gorb2010,YOlikh2006TPLr,Parchinskii2000r,
YOlikhTPL2011r,Olikh:FTP2011,Olikh:SEMT2007,Zaveryukhin2002:2,YOlikh2005,RomanyukSST,ROMANYUK2005,Roman:2006JAP,
Roman:2007APL,Roman:2010JAP,YOlikh2010JL,US:FeSiC,Kuryliuk2009,Buscemi,Cuberes,Olikh:SPQEO2010}


%\cite{Ostap:SiO2} стан границі Si--SiO2, підсилення дифузії водню
%\cite{Ostap:PhotoLum,ostapenko1997} підсилення дифузії водню
%
%дефектний стан границі Si--SiO2,\cite{UST:Medvid}
%
%УЗО опромінених електронами GaAs-GaP LEDs,
%підсилення інтенсивності люмінісценції внаслідок поглинання невипромінювальних центрів викликаних рухом дислокацій в УЗ полі (часткове відновлення після радіаційної деградації)
%\cite{US:LED,UST:LED_SM}
%
%відпал радіаційних дефектів в лужногалоїдних кристалах внаслідок їх дифузії\cite{UST:OstrovCsI}
%
%структура Si--SiO2, (100), n-тип, опір 0,2 Ом см, опромінення гамма--квантами 60Со, 1е6 рад,C--V виміри
%УЗО веде до зменшення ефективного поверхневого заряду (АІ дифузія нестабільних рад дефектів в полі пружних напруг системи Si--SiO2)\cite{Parchinskii2000r}
%зменшення генераційного часу життя в області Si, що прилягає до контакту(збільшення концентрації центрів, внаслідок їх звільнення з домішкових асоціатів),
%незначні зміни швидкості поверхневої рекомбінації (перебудова
%приповерхневої області внаслідок УЗО слабша, ніж для неопромінених, бо радіація стимулювала релаксацію напруг)\cite{Parchinskii2006r}
%
%для неопромінених збільшення часу і зменшення швидкості внаслідок перебудови дефектної структури, яка виявляється більш яскраво через наявність пружних полів \cite{Parchinskii2003r}
%
%Відновлення електропровідності опроміненого кремнію \cite{OstrovRadSi}
%
%відновлення (до 70 відсотків) часу життя в опроміненому (гамма-кванти 60Со, 5е6--2е7 рад) Fz-Si,
%механізм: звільнення вакансій з Е--центрів і їх подальше захоплення на стоки \cite{Podolian2012r}
%
%низькотемпературний відпал рад дефектів в Cz--Si (розпад комплексів C--O--V2) \cite{PodolHivr}
%
%перебудова рад дефектів (гама кванти) в кремнії внаслідок УЗО.. правда зміни оборотні \cite{YOlikh2006TPLr}
%
%Опромінення гамма--квантами 60Со, 1е6 рад кремнієві сонячні елементи,
%УЗО приводить до відновлення характеристик, причина --- підвищення однорідності структури та перерозподіл радіаційних дефектів
%\cite{YOlikh2007TPLr}
%
%
%
%АІ зміна густини поверхневих станів на границі Si--SiO2
%(малі потужності -- зменшення, великі --- збільшення) \cite{Zaver:2008r}
%
%\cite{OstrovRadSi,Podolian2012r,PodolHivr,YOlikh2007TPLr}, германію \cite{Olikh:FTP1996},
%напівровідникових \cite{OlikhProc,OstrovFTTRad} та лужногалоїдних \cite{UST:OstrovCsI} сполук.


\section{Динамічні акустоіндуковані ефекти}


Основні результати даного розділу представлені в роботах \cite{Olikh:SEMT2004,Olikh:SEMT2011,1UNCPS,2013Buk}.

