%\chapter{Акустоіндуковані ефекти у мікроелектронних структурах та матеріалах\label{Oglyad}}
\chapter{Передумови та перспективи використання активного ультразвуку в мікроелектроніці\label{Oglyad}}


Загалом, явища, пов'язані з механічними коливаннями пружного середовища, широко застосовуються у мікроелектроніці.
Це стосується, насамперед акустоелектроніки --- галузі, що базується на використанні взаємодії акустичних та електричних сигналів у п'єзоелектричних середовищах.
%Існує величезна кількість різноманітних акустоелектричних фільтрів, ліній затримки, трансформаторів та інших пристроїв,
%принцип роботи яких базується на використанні п'єзоелектричного ефекту.
Проте у цьому розділі буде розглянуто, переважно, дещо інший аспект використання акустичних хвиль (АХ),
зумовлений, насамперед, можливістю акустоіндукованих (АІ) процесів перебудови дефектної підсистеми напівпровідників, яка, як відомо, є визначальною для властивостей як самих кристалів, так і приладів на їх основі.
Безумовно, найбільш поширені та вивчені методи <<інженерії дефектів>> пов'язані з
а)~використанням висотемпературних обробок (відпалів), різноманітних за тривалістю, атмосферою та іншими умовами проведення;
б)~опромінення частинками різної природи (високоенергетичними фотонами, електронами, нейтронами, іонами) та енергії;
в)~вибором режиму вирощування кристалу (температура, швидкість, атомарний склад сировини тощо).
Проте перспективним засобом активного впливу на властивості кристалів є також використання акустичних коливань ультразвукового діапазону, про що свідчить накопичений достатньо широкий експериментальний матеріал.
Надалі в розділі будуть розглянуті літературні дані, які стосуються незворотних змін властивостей внаслідок ультразвукових обробок (УЗО), ефекти, які виникають у напівпровідниках під час поширення пружних коливань,
а також можливість застосування ультразвуку (УЗ) під час виготовлення пристроїв чи як інструменту оцінки дефектної підсистеми.

%Відомо, що робочі характеристики різноманітних напівпровідникових приладів значним чином визначаються їх дефектним складом.
%Відповідно, методи, які дозволяють модифікувати властивості дефектної підсистеми кристалу (методи <<інженерії дефектів>>) можуть бути інструментом як динамічного, так і статичного керування властивостями подібних пристроїв.
%Безумовно, найбільш поширені та вивчені можливості досягнення подібної мети пов'язані з
%а)~використанням висотемпературних обробок (відпалів), різноманітних за тривалістю, атмосферою та іншими умовами проведення;
%б)~опромінення частинками різної природи (високоенергетичними фотонами, електронами, нейтронами, іонами) та енергії;
%в)~вибором режиму вирощування кристалу  та домішкового складу.
%З іншого боку, не дуже поширеним, проте, на нашу думку, перспективним засобом активного впливу на властивості кристалів є використання акустичних коливань ультразвукового діапазону.
%Цей підхід має певні переваги (зокрема, можливість реалізації вибіркового та резонансного впливу),
%підтвердженням його ефективності є накопичений експериментальний матеріал, короткому огляду якого і присвячений даний розділ.

\section{Залишкові акустоіндуковані ефекти}

В першу чергу зупинимось на зміні властивостей напівпровідникових кристалів та пристроїв на їх основі в результаті тривалого, порядку $(10^3\div10^4)$~с збудження в кристалах АХ значної інтенсивності (як правило, не менше 1~Вт/см$^2$).
Характерною особливістю подібних експериментів є те, що визначення параметрів відбувається після припинення дії пружних коливань.


Подібні ефекти спостерігалися як у кристалах напівпровідникових сполук, так і в ковалентних кристалах, матриця яких складається з атомів одного сорту.
В першому випадку нерідко механізм АІ змін пов'язаний з дислокаціями.


Нерідко виявлені АІ зміни параметрів пов'язані із рухом легуючих домішок до стоків, у ролі яких виступають дислокації.
Так, подібне переміщення акцепторів в кристалах CdTe вважається причиною зменшення як домішкової, так і екситонної люмінесценції \cite{US:CdTe},
безактиваційний рух мілких донорів у CdS --- зменшення фотоструму \cite{BorkovFTT}.
У кубічних кристалах Zn$_x$Cd$_{1-x}$Te  УЗО викликає зміни величини провідності (як темнової, так і фото) та інтенсивності фотолюмінісценції \cite{US:ZnCdTe}.
Автори вважають, що для малодислокаційних кристалів ці явища пов'язані зі збільшенням дислокацій і стіканням на них рухливих акцепторів, тоді як для у зразках з високою концентрацією дислокацій має місце відхід акцепторів у об'єм.


Акустостимульована дифузія може відбуватися внаслідок АІ зменшення  енергії активації чи надбар'єрного руху дефектів у полі пружних напруг \cite{USdif:FTT90}.



Певну окрему групу складають результати отримані при УЗО кристалів Cd$_x$Hg$_{1-x}$Te (надалі --- КРТ).
В цьому випадку суттєву роль у АІ змінах електрофізичних параметрів, як показано в роботах \cite{KRT:FTT89,KRT:FTT90}, відіграють коливання сітки малокутових границь сублоків.
Зокрема, при невисоких інтенсивностях АХ вони стають причиною гетерування електрично активних дефектів на стоки і відповідного збільшення рухливості та часу життя носіїв \cite{KRT:FTP90}, зміни ступеня компенсації та зниження шуму $1/f$ \cite{Ol_Shav}.
Ще однією особливістю цих твердих розчинів є те, що АІ ефекти мають яскравий виражений резонансний характер коли частота АХ наближається до частоти коливань малокутових границь \cite{KRT:FTP90,KRT:FTT89,KRT:FTT90,Ol_Shav}.
При надпорогових інтенсивностях в кристалах $n$--КРТ переважають процеси генерації електрично активних дефектів, що викликає зменшення концентрації та рухливості  вільних електронів \cite{KRT:FTP90,KRT:FTT89}.

АІ ефекти спостерігалися і в інших монокристалах твердих розчинів.
Наприклад, в кубічних кристалах Zn$_x$Cd$_{1-x}$Te виявлено, що УЗО викликає зміни величини провідності (як темнової, так і фото) та інтенсивності фотолюмінісценції \cite{US:ZnCdTe}.
Автори вважають, що для малодислокаційних кристалів ці явища пов'язані зі збільшенням дислокацій і стіканням на них рухливих акцепторів, тоді як для у зразках з високою концентрацією дислокацій має місце відхід акцепторів у об'єм.
Зокрема останній процес не є необоротним і з часом після УЗО провідність зменшється.







Наприклад, було показано, що в результаті УЗО відбуваються процеси зміни зарядового стані центрів, що спричинює підсилення люмінесценції пористого \cite{Bahar2003} та полікристалічного \cite{Ostap:PhotoLum} кремнію та
зміну фотопровідності кристалів селеніду цинку \cite{US:ZnSe};
акустоіндуковані (АІ) процеси перебудови та генерації дефектів є причинами змін
домішкового поглинання в арсеніді галію \cite{Zaver2007},
фоточутливості та випромінювальної рекомбінації в кристалах ZnSe \cite{ZobovFTP2008},
спектра фотолюмінісценції та коефіцієнта відбиття фулеренових плівок \cite{RITTER2008}.
УЗО може бути причиною трансформації підсистеми власних дефектів, як це спостерігалося в GaAs \cite{Wosinski}.

УЗО також може стимулювати дифузію різноманітних домішок при температурах, близьких до кімнатних.
Наприклад, в роботі \cite{Ostapenko1999} показано, що завдяки підсиленню дифузії водню в акустичному полі відбувається покращення пасивації дефектів на границях зерен в полікристалічному кремнії.
Проте частіше акустодифузія є причиною зміни властивостей поверхні:
прикладом може бути зміни коефіцієнту відбиття Si та GaAs \cite{Zaver} чи зменшення густини поверхневих станів Si \cite{Zaver:2008r} внаслідок дифузії легуючої домішки від поверхні напівпровідника.
Зауважимо, що напрям акустостимульованого руху може бути і протилежним:
в роботі \cite{Ostrov2002FTPr} спостерігалося переміщення атомів К та Na до поверхні кремнію.
З інших виявлених ефектів зміни властивостей границь розділу, викликаних УЗО,
можна виокремити зміцнення поверхневого шару кремнію через утворення точкових дефектів типу вакансійних та вакансійно--домішкових кластерів \cite{Ostrov2000FTPr},
зменшення швидкості поверхневої рекомбінації на межі напівпровідник--скло внаслідок перебудови напружених валентних зв'язків \cite{Vlasov2009r} або викликану генерацією дефектів зміну енергетичного спектру поверхневих станів Si \cite{Zaver:2008r}.

В гетероструктурах УЗО нерідко викликає релаксацію внутрішніх механічних напруг.
Подібні ефекти спостерігалися в системах Ge--GaAs \cite{BritunFTT} 


УЗО може бути причиною зміни властивостей бар'єрних напівпровідникових пристроїв.
Наприклад, з літератури відомо, що подібна обробка може викликати покращення фотоелектричних параметрів (струму короткого замикання, напруги холостого ходу, коефіцієнта корисної дії та діапазоні спектральної чутливості) AlGaAs/GaAs сонячних елементів в результаті перерозподілу домішкових атомів, відпалу рекомбінаційних центрів та розпаду домішкових скупчень \cite{Zaver2005}, а також зменшення концентрації носіїв заряду в кремнієвих $p$--$n$ структурах внаслідок зміни енергетичного спектру дефектів \cite{Davletova2008}.
Крім того, УЗО впливає на величину зворотного струму кремнієвих структур з бар'єром Шотки \cite{Tagaev}, а також на тунельну складову струму в InAs $p$--$n$ переходах \cite{Teterkin2009r}.

Деградація фотоелектричних властивостей сонячних елементів з контактом Шотки ($a$--PbSb)--$n$--Si \cite{Pashaev2012r,PashOJA}
В результаті УЗО спостерігається підсилення та зміна спектру фотолюмінісценції підкладок та приконтактних областей структур метал--GaAs, а також зменшення зворотного струму.
Причиною вважається упорядкування дислокаційної структури, викликане АІ потоком вакансій \cite{UST:SDErmol}.

Зауважимо, що УЗО впливає не лише на дефекти, які виникають під час вирощування кристалів чи викготовленні структур, але й на дефекти радіаційного походження.
В цьому випадку може відбуватися не тільки низькотемпературний відпал дефектів подібного типу (як це спостерігалося в кристалах кремнію \cite{PodolHivr} та іонних сполук \cite{UST:OstrovCsI}, в також кремнієвих сонячних елементах \cite{YOlikh2007TPLr}), але й їх перебудова, яка призводить до збільшення концентрації генераційних центрів в області границі розділу Si--SiO$_2$  \cite{Parchinskii2006r} чи концентрації вільних носіїв заряду в кремнії \cite{YOlikh2006TPLr}.
Крім того, після УЗО спостерігалося зменшення радіаційно--індукованого заряду в діелектричному прошарку структур метал--окис--напівпровідник на основі кремнію завдяки АІ дифузії радіаційних дефектів \cite{Parchinskii2000r}.

Ще раз підкреслимо, що практично у всіх вищенаведених прикладах мова йде про необоротні зміни властивостей,
пов'язані з необоротними змінами підсистеми дефектів в результаті дії УЗ коливань.


Залишкові ефекти не завжди залишаються стабільними.
Наприклад, кристали з АІ зміною провідності відновлюють свої властивості після зберігання при кімнатній температурі на протязі декількох діб \cite{YOlikh2006TPLr,US:ZnCdTe,BorkovFTT},
характерний час відновлення параметрів InAs $p$--$n$ переходів --- декілька місяців \cite{Teterkin2009r}.
  

Наведені результати свідчать, що можливості впливу УЗ охоплюють широкий спектр як напівпровідникових матеріалів, та і їх властивостей.
Проте, необхідно зауважити, що подібний спосіб впливу має і певні недоліки, пов'язані з тим, що він передбачає достатньо довгий час обробки ($(10^3\div10^4)$~с) і вимагає введення у кристал АХ зі значною інтенсивністю (як правило, не менше 1 Вт/см$^2$).
Крім того, отримані за допомогою УЗО результати нерідко можна продублювати з використанням більш технологічно звичних методів на кшталт відпалу чи радіаційного опромінення.
Правда, необхідно заувжити, що використання УЗ нерідко має і свої переваги, пов'язані з локалізацією впливу:
наприклад, акустовідпал радіаційних дефектів \cite{PodolHivr,UST:OstrovCsI,YOlikh2007TPLr} відбувається при температурах, недостатніх для суттєвої дифузії легуючих домішок, а отже і розмиття профіля легування, який може супроводжувати процес звичайного термовідпалу.



Таким чином, більш перспективними для практичного використання здається використання УЗ
а)~для динамічної (оборотної) зміни властивостей напівпровідникових пристроїв під час їх роботи;
б)~як фактору додаткового впливу на структури та властивості матеріалів під час практично класичних радіаційної чи термічної обробок.
Тобто, на думку автора, ефективним буде використання УЗ не як основного інструменту, а паралельного фактору впливу, який накладається на процеси, що відбуваються під час функціонування приладу чи його обробки (створення).
Підгрунтям для подібного припущення є те, що кристали за  робочих чи технологічних умов нерідко опиняються у нестійкому стані і їх дефектно--домішкова підсистема більш легко може бути перебудована (модифікована) під дією УЗ.
Зокрема, в цьому випадку можна застосовувати АХ значно меншої інтенсивності,
що дозволить ще більше підвищити локалізованість впливу саме на дефектах.

Цей підхід має певні переваги (зокрема, можливість реалізації вибіркового та резонансного впливу),
підтвердженням його ефективності є накопичений експериментальний матеріал, короткому огляду якого і присвячений даний розділ.


\section{Динамічні акустоіндуковані ефекти}

Перший підхід може бути розглянутий в межах функціональної електроніки --- достатньо нової галузі,
яка використовує для керування інформаційним сигналом неоднорідності середовища,  що виникають під дією керуючого сигналу.
Саме роль керуючого сигналу і може виконувати АХ, динамічно змінюючи стан дефектів і, таким чином, реалізуючи можливість створення акустокерованих пристроїв.
Наприклад, для створення генератора або частотно--вибіркового підсилювача використовують транзистори, резистори, конденсатори.
При використанні підходу інтегральної схемотехніки всі ці елементи реалізуються на базі напівпровідників.
Зміна провідності кристалу, і, відповідно, номіналу елементу (у найпростішому випадку --- опору резистора) під дією УЗ
має викликати керовану (наприклад, інтенсивністю чи частотою УЗ) перебудову робочої частоти (частоти зворотнього зв'язку) пристрою.
Можливість реалізації подібного підходу підтверджують результати роботи \cite{YOlikhTPL2011r}, де спостерігалися динамічні зміни концентрації носіїв у кремнії при поширенні імпульсів УЗ.

Також необхідно врахувати можливість оборотного УЗ впливу на процеси поширення нерівноважних носіїв.
Мова йде про АІ перебудову (та/або перезарядку)  центрів рекомбінації (прилипання), що має викликати зміну перерізу захоплення та часу життя носіїв.
Наприклад, експериментальному дослідженню впливу УЗ на нерівноважні носії у арсенід галієвих фотоприймачах присвячена робота \cite{Zaveryukhin2002:2}.

Якщо розглядати АХ як частину багатокомпонентної обробки, то збудження УЗ під час радіаційного опромінення кристалів може бути фактором підвищення радіаційної стійкості напівпровідників завдяки акустопідвищенню швидкості рекомбінації (акустовідпалу) первинних радіаційних дефектів (РД).
Інший шлях підвищення радіаційної стійкості --- створення електрично та рекомбінаційно неактивних вторинних РД внаслідок реакцій між первинними РД та домішками: згадаймо, що процеси дифузії та перебудови РД також є акустоактивованими \cite{YOlikh2006TPLr,Parchinskii2000r}.
Крім того, ще одним позитивним фактором впливу УЗ може бути стимуляція переведення занурених радіаційним чином домішок у положення, що відповідає електричноактивному стану --- наприклад, іонів--легантів у вузлове положення.

Експериментальні передумови подібного застосування УЗ представлені в роботах \cite{YOlikh2005,RomanyukSST,ROMANYUK2005}.
Зокрема показано, що для випадку, коли іонна імплантація супроводжується УЗ навантаженням, спостерігається зменшення дефектів міжвузлового типу в області збіднення $p$--$n$ переходів \cite{YOlikh2005},
підсилюється процес аморфізації поверхневого шару кремнію \cite{RomanyukSST},
покращуються властивості та зменшується товщина оксиду вольфраму, що наноситься на поверхню $p$--Si \cite{ROMANYUK2005}.

Метод іонної імплантації використовується не лише для легування напівпровідників чи створення поверхневого шару із заданими властивостями, але й для формування різноманітних кластерів в матриці різної природи.
В цьому випадку УЗ також може бути додатковим позитивним фактором впливу, про що свідчать результати робіт \cite{Roman:2006JAP,Roman:2007APL,Roman:2010JAP,YOlikh2010JL}.
Наприклад, було показано, що наявність супроводжуючого УЗ навантаження викликає збільшення кінцевих розмірів занурених металевих кластерів у кремнії \cite{Roman:2006JAP} та оксиді кремнію \cite{Roman:2007APL}, зміни парамагнітних \cite{Roman:2010JAP} та фотолюмінесцентних \cite{YOlikh2010JL}властивостей нанокластерів Si в SiO$_2$.

АХ можуть використовуватися не лише під час радіаційного опромінення, але й бути частиною будь--якої обробки, пов'язаної зі зміною дефектної підсистеми.
Наприклад, застосування УЗ паралельно з лазерною обробкою систем Fe--Si--C викликає суттєве (в 2--3 рази) зменшення залишкового аустеніту порівняно з використанням лише лазерного опромінення \cite{US:FeSiC}.


Ще один напрям, де ефективно може бути використаний УЗ, наступний.
Відомо, що бістабільний дефект є перспективним елементом пам'яті нового покоління,
дві конфігурації якого відповідають логічним <<нулю>> та <<одиниці>>.
В такому випадку УЗ може виступати інструментом перемикання подібного елементу.
Зокрема, гітотетичний механізм переведення дефекту між станами може бути наступним:
при поширенні АХ дефект зміщується відносно свого оточення, це призводить до зміни його симетрії, що, згідно з ефектом Яна--Теллера, може викликати перезарядку центру.
Нерідко у різних зарядових станах мінімальній енергії відповідають різні просторові конфігурації, що и призводить до фіксації нового логічного стану.
Про конфігураційну перебудову бістабільного центру Fe--B як можливу причину АІ зміни довжини дифузії носіїв у кремнії повідомлялося в роботі \cite{OlikhFTT}.

Зауважимо, що в цьому розділі були порушені можливості динамічного використання УЗ, пов'язані з акусто--дефектною взаємодією.
Водночас відомо, що АХ можуть бути причиною і багатьох інших цікавих явищ, на кшталт просторове розділення заряду у квантових ямах \cite{Kuryliuk2009}, квантова телепортація \cite{Buscemi}, маніпуляція наночастинками \cite{Cuberes,Olikh:SPQEO2010}.

Таким чином, застосування активного, динамічного УЗ у різноманітних областях, пов'язаних з виготовленням та застосуванням різноманітних мікроелектронних пристроїв є достатньо перспективним та потребує ретельного експериментального та теоретичного дослідження.



%\cite{Ostap:SiO2} стан границі Si--SiO2, підсилення дифузії водню
%\cite{Ostap:PhotoLum,ostapenko1997} підсилення дифузії водню
%
%дефектний стан границі Si--SiO2,\cite{UST:Medvid}
%
%УЗО опромінених електронами GaAs-GaP LEDs,
%підсилення інтенсивності люмінісценції внаслідок поглинання невипромінювальних центрів викликаних рухом дислокацій в УЗ полі (часткове відновлення після радіаційної деградації)
%\cite{US:LED,UST:LED_SM}
%
%відпал радіаційних дефектів в лужногалоїдних кристалах внаслідок їх дифузії\cite{UST:OstrovCsI}
%
%структура Si--SiO2, (100), n-тип, опір 0,2 Ом см, опромінення гамма--квантами 60Со, 1е6 рад,C--V виміри
%УЗО веде до зменшення ефективного поверхневого заряду (АІ дифузія нестабільних рад дефектів в полі пружних напруг системи Si--SiO2)\cite{Parchinskii2000r}
%зменшення генераційного часу життя в області Si, що прилягає до контакту(збільшення концентрації центрів, внаслідок їх звільнення з домішкових асоціатів),
%незначні зміни швидкості поверхневої рекомбінації (перебудова
%приповерхневої області внаслідок УЗО слабша, ніж для неопромінених, бо радіація стимулювала релаксацію напруг)\cite{Parchinskii2006r}
%
%для неопромінених збільшення часу і зменшення швидкості внаслідок перебудови дефектної структури, яка виявляється більш яскраво через наявність пружних полів \cite{Parchinskii2003r}
%
%Відновлення електропровідності опроміненого кремнію \cite{OstrovRadSi}
%
%відновлення (до 70 відсотків) часу життя в опроміненому (гамма-кванти 60Со, 5е6--2е7 рад) Fz-Si,
%механізм: звільнення вакансій з Е--центрів і їх подальше захоплення на стоки \cite{Podolian2012r}
%
%низькотемпературний відпал рад дефектів в Cz--Si (розпад комплексів C--O--V2) \cite{PodolHivr}
%
%перебудова рад дефектів (гама кванти) в кремнії внаслідок УЗО.. правда зміни оборотні \cite{YOlikh2006TPLr}
%
%Опромінення гамма--квантами 60Со, 1е6 рад кремнієві сонячні елементи,
%УЗО приводить до відновлення характеристик, причина --- підвищення однорідності структури та перерозподіл радіаційних дефектів
%\cite{YOlikh2007TPLr}
%
%
%
%АІ зміна густини поверхневих станів на границі Si--SiO2
%(малі потужності -- зменшення, великі --- збільшення) \cite{Zaver:2008r}
%
%\cite{OstrovRadSi,Podolian2012r,PodolHivr,YOlikh2007TPLr}, германію \cite{Olikh:FTP1996},
%напівровідникових \cite{OlikhProc,OstrovFTTRad} та лужногалоїдних \cite{UST:OstrovCsI} сполук.





Основні результати даного розділу представлені в роботах \cite{Olikh:SEMT2004,Olikh:SEMT2011,1UNCPS,2013Buk}.

