%\chapter{Акустоіндуковані ефекти у мікроелектронних структурах та матеріалах\label{Oglyad}}
\chapter{Передумови та перспективи використання активного ультразвуку в мікроелектроніці\label{Oglyad}}


Загалом, явища, пов'язані з механічними коливаннями пружного середовища, широко застосовуються у мікроелектроніці.
Це стосується, насамперед акустоелектроніки --- галузі, що базується на використанні взаємодії акустичних та електричних сигналів у п'єзоелектричних середовищах.
%Існує величезна кількість різноманітних акустоелектричних фільтрів, ліній затримки, трансформаторів та інших пристроїв,
%принцип роботи яких базується на використанні п'єзоелектричного ефекту.
Проте цей класичний ефект акустоелектронної взаємодії у об'ємних кристалах ми залишимо поза увагою, і в цьому розділі
%Проте у цьому розділі буде розглянуто,
розглянемо,переважно, дещо інший аспект використання акустичних хвиль (АХ),
зумовлений, насамперед, можливістю акустоіндукованих (АІ) процесів перебудови дефектної підсистеми напівпровідників, яка, як відомо, є визначальною для властивостей як самих кристалів, так і приладів на їх основі.
Безумовно, найбільш поширені та вивчені методи <<інженерії дефектів>> пов'язані з
а)~використанням висотемпературних обробок (відпалів), різноманітних за тривалістю, атмосферою та іншими умовами проведення;
б)~опромінення частинками різної природи (високоенергетичними фотонами, електронами, нейтронами, іонами) та енергії;
в)~вибором режиму вирощування кристалу (температура, швидкість, атомарний склад сировини тощо).
Проте перспективним засобом активного впливу на властивості кристалів є також використання акустичних коливань ультразвукового діапазону, про що свідчить накопичений достатньо широкий експериментальний матеріал.
Надалі в розділі будуть розглянуті літературні дані, які стосуються змін властивостей внаслідок ультразвукових обробок (УЗО), ефекти, які виникають у напівпровідниках під час поширення пружних коливань,
а також можливість застосування ультразвуку (УЗ) під час виготовлення пристроїв чи як інструменту оцінки дефектної підсистеми.

%Відомо, що робочі характеристики різноманітних напівпровідникових приладів значним чином визначаються їх дефектним складом.
%Відповідно, методи, які дозволяють модифікувати властивості дефектної підсистеми кристалу (методи <<інженерії дефектів>>) можуть бути інструментом як динамічного, так і статичного керування властивостями подібних пристроїв.
%Безумовно, найбільш поширені та вивчені можливості досягнення подібної мети пов'язані з
%а)~використанням висотемпературних обробок (відпалів), різноманітних за тривалістю, атмосферою та іншими умовами проведення;
%б)~опромінення частинками різної природи (високоенергетичними фотонами, електронами, нейтронами, іонами) та енергії;
%в)~вибором режиму вирощування кристалу  та домішкового складу.
%З іншого боку, не дуже поширеним, проте, на нашу думку, перспективним засобом активного впливу на властивості кристалів є використання акустичних коливань ультразвукового діапазону.
%Цей підхід має певні переваги (зокрема, можливість реалізації вибіркового та резонансного впливу),
%підтвердженням його ефективності є накопичений експериментальний матеріал, короткому огляду якого і присвячений даний розділ.

\section{Залишкові акустоіндуковані ефекти}

В першу чергу зупинимось на зміні властивостей напівпровідникових кристалів та пристроїв на їх основі в результаті тривалого, порядку $(10^3\div10^4)$~с збудження в кристалах АХ значної інтенсивності (як правило, не менше 1~Вт/см$^2$).
Характерною особливістю подібних експериментів є те, що визначення параметрів відбувається після припинення дії пружних коливань.

Подібні ефекти спостерігалися як у кристалах напівпровідникових сполук, так і в ковалентних кристалах, матриця яких складається з атомів одного сорту.
В першому випадку нерідко механізм змін пов'язаний з рухом і розмноженням дислокаціями в УЗ полі.
Зокрема АІ перерозподіл дислокацій викликає зміну пружних модулів GaP та GaAS \cite{UST:GaP}.
УЗО також може стимулювати дифузію різноманітних дефектів при температурах, близьких до кімнатних, що пов'язано
з АІ зменшенням  енергії активації чи надбар'єрним рухом дефектів у полі пружних напруг \cite{USdif:FTT90}.
Наприклад, в роботі \cite{Ostapenko1999} показано, що завдяки підсиленню дифузії водню в акустичному полі відбувається покращення пасивації дефектів на границях зерен \cite{Ostapenko1999} та підсилення фотолюмінесценції \cite{Ostap:PhotoLum,ostapenko1997} в полікристалічному кремнії.
Проте частіше акустодифузія є причиною зміни властивостей поверхні:
прикладом може бути зміни коефіцієнту відбиття Si та GaAs \cite{Zaver}, зменшення густини  поверхневих станів \cite{Zaver:2008r} та зростання адгезійної здатності \cite{Zaver96} Si внаслідок дифузії домішок від поверхні напівпровідника.
Зауважимо, що напрям акустостимульованого руху може бути і протилежним:
в роботі \cite{Ostrov2002FTPr} спостерігалося переміщення атомів К та Na до поверхні кремнію.
З інших виявлених ефектів зміни властивостей поверхні можна виокремити зміцнення поверхневого шару кремнію через утворення точкових дефектів типу вакансійних та вакансійно--домішкових кластерів \cite{Ostrov2000FTPr},
 або викликану генерацією дефектів зміну енергетичного спектру поверхневих станів Si \cite{Zaver:2008r}.

Нерідко виявлені АІ зміни параметрів пов'язані із рухом легуючих домішок до стоків, у ролі яких виступають дислокації.
Так, подібне переміщення акцепторів у кристалах CdTe вважається причиною зменшення як домішкової, так і екситонної люмінесценції \cite{US:CdTe},
безактиваційний рух мілких донорів у CdS --- зменшення фото-- та термостимульованого струмів \cite{BorkovFTT,sheinkman1995,BORKOVSKA2003}.
У кубічних кристалах Zn$_x$Cd$_{1-x}$Te  УЗО викликає зміни величини провідності (як темнової, так і фото--) та інтенсивності фотолюмінісценції \cite{US:ZnCdTe}.
Автори вважають, що для малодислокаційних кристалів ці явища пов'язані зі збільшенням дислокацій і стіканням на них рухливих акцепторів, тоді як для у зразках з високою концентрацією лінійних дефектів має місце відхід акцепторів у об'єм.

Ряд виявлених ефектів дослідники пов'язують з перебудовою точкових дефектів внаслідок УЗО.
Наприклад, саме таке пояснення запропоновано для акустоіндукованих
підсилення фотолюмінесценції поруватого кремнію \cite{Bahar2003},
змін фотопровідності \cite{US:ZnSe},  фоточутливості та випромінювальної рекомбінації  \cite{ZobovFTP2008} кристалів ZnSe,
домішкового поглинання \cite{Zaver2007} та спектру фотопровідності \cite{UST:GaAs2015} в арсеніді галію,
спектра фотолюмінісценції та коефіцієнта відбиття фулеренових плівок \cite{RITTER2008}.
Крім того, немало досліджень присвячено впливу УЗО безпосередньо на точкові дефекти в кристалах.
Наприклад, виявлено перебудови  власних дефектів в GaAs \cite{Wosinski,Ostapenko1994,buyanova1994} та Si \cite{UST:Onanko},
розпад домішкових пар в кремнії \cite{Ostapenko1995SST,Ostapenko1995,Ostapenko1994APL}.


Певну окрему групу складають результати отримані при УЗО кристалів Cd$_x$Hg$_{1-x}$Te (надалі --- КРТ).
В цьому випадку суттєву роль у АІ змінах електрофізичних параметрів, як показано в роботах \cite{KRT:FTT89,KRT:FTT90}, відіграють коливання сітки малокутових границь сублоків.
Зокрема, при невисоких інтенсивностях АХ вони стають причиною гетерування електрично активних дефектів на стоки і відповідного збільшення рухливості та часу життя носіїв \cite{KRT:FTP90,Savkina:SPQEO2006}, зміни ступеня компенсації та зниження шуму $1/f$ \cite{Ol_Shav}.
Ще однією особливістю цих твердих розчинів є те, що АІ ефекти мають яскравий виражений резонансний характер коли частота АХ наближається до частоти коливань малокутових границь \cite{KRT:FTP90,KRT:FTT89,KRT:FTT90,Ol_Shav}.
При надпорогових інтенсивностях в кристалах $n$--КРТ переважають процеси генерації електричноактивних дефектів, що викликає зменшення концентрації та рухливості  вільних електронів \cite{KRT:FTP90,KRT:FTT89}.
Крім того, в епітаксійних структурах на основі КРТ виявлено АІ ефекти зміни типу провідності, появи негативного диференційного опору, підвищення фоточутливості, модифікації спектру фотопровідності \cite{Savkina:SST07,SavkinaPSSB2002}.


В гетероструктурах УЗО нерідко викликає релаксацію внутрішніх механічних напруг.
Подібні ефекти спостерігалися в системах Ge--GaAs \cite{BritunFTT,UST:GeGaAs1990} та Si--SiO$_2$ \cite{Zdeb1989}.
Крім того, відбуваються зміни і електрофізичних параметрів.
Так в системах  на основі кремнію УЗО викликає зміну часу життя неосновних носіїв у приконтактній області напівпровідника
внаслідок зміни числа генераційно--рекомбінаційних центрів \cite{Parchinskii2003r,Zdeb1989};
зменшення швидкості поверхневої рекомбінації внаслідок перебудови напружених валентних зв'язків \cite{Vlasov2009r,Parchinskii2003r};
підвищення яскравості електролюмінесценції систем Y$_2$O$_3$--ZnS--SiO$_2$, викликане дифузією домішкових центрів \cite{UST:ZnS}.
Про АІ зміну дефектного стану границі Si--SiO$_2$ повідомляється в роботах \cite{Ostap:SiO2,UST:Medvid}.


УЗО може бути причиною зміни властивостей бар'єрних напівпровідникових пристроїв.
Наприклад, з літератури відомо, що подібна обробка може викликати покращення фотоелектричних параметрів AlGaAs/GaAs \cite{Zaver2005} та CuInSe$_2$ \cite{OstapSC} сонячних елементів в результаті перерозподілу домішкових атомів, відпалу рекомбінаційних центрів та розпаду домішкових скупчень;
зменшення концентрації носіїв заряду \cite{Davletova2008}
та зростання фактору неідеальності \cite{Davletova2009}
внаслідок зміни енергетичного спектру дефектів в кремнієвих $p$--$n$ структурах;
зміну тунельної складової струму в InAs $p$--$n$ переходах \cite{Teterkin2009r},
немонотонне зменшення світності GAAsP світловипромінюючих діодів внаслідок захоплення дислокаціями невипромінюючих центрів (при малих часах обробки) чи акустогенерації дефектів (при тривалих навантаженнях) \cite{US:LED}.
У фотодіодах $p$--Si/$n$--CdS/$n^+$--CdS УЗО викликає зменшення густини поверхневих станів на інтерфейсній границі,
що призводить до змін вольт--амперних характеристик та підвищення фоточутливості \cite{Mirsagatov,Mirsagatov2}
Не залишилися поза увагою дослідників і структури з контактом Шотки.
Так виявлено, що УЗО викликає деградацію фотоелектричних властивостей структур ($a$--PbSb)--$n$--Si \cite{Pashaev2012r,PashOJA},
зменшує величини зворотного струму кремнієвих \cite{Tagaev} та арсенід галієвих \cite{UST:SDErmol} систем,
викликає підсилення та зміни спектру фотолюмінісценції підкладок та приконтактних областей структур метал--GaAs \cite{UST:SDErmol}.
В останньому випадку причиною вважається впорядкування дислокаційної структури, викликане АІ потоком вакансій \cite{UST:SDErmol}.

Зауважимо, що УЗО впливає не лише на ростові чи технологічні дефекти, але й викликає відпал порушень періодичності радіаційного походження внаслідок їх розпаду, перебудови та дифузії до стоків.
Повне чи, частіше, часткове відновлення радіаційно--деградованих властивостей спостерігалося
в кристалах Si  \cite{OstrovRadSi,Podolian2012r,PodolHivr,YOlikh2006TPLr}, Ge \cite{Olikh:FTP1996},
InP \cite{OlikhProc}, CsI \cite{UST:OstrovCsI}, структурах Si--SiO$_2$ \cite{Parchinskii2000r,Parchinskii2006r},
кремнієвих сонячних елементах \cite{YOlikh2007TPLr},
$\alpha$--NiTi--$n$--Si дідах Шотки \cite{Pashaev2014r},
GaAsP світловипромінюючих діодах \cite{US:LED,UST:LED_SM}.



%УЗО опромінених електронами GaAs-GaP LEDs,
%підсилення інтенсивності люмінісценції внаслідок поглинання невипромінювальних центрів викликаних рухом дислокацій в УЗ полі (часткове відновлення після радіаційної деградації)
%\cite{US:LED,UST:LED_SM}
%
%відпал радіаційних дефектів в лужногалоїдних кристалах внаслідок їх дифузії\cite{UST:OstrovCsI}
%
%структура Si--SiO2, (100), n-тип, опір 0,2 Ом см, опромінення гамма--квантами 60Со, 1е6 рад,C--V виміри
%УЗО веде до зменшення ефективного поверхневого заряду (АІ дифузія нестабільних рад дефектів в полі пружних напруг системи Si--SiO2)\cite{Parchinskii2000r}
%зменшення генераційного часу життя в області Si, що прилягає до контакту(збільшення концентрації центрів, внаслідок їх звільнення з домішкових асоціатів),
%незначні зміни швидкості поверхневої рекомбінації (перебудова
%приповерхневої області внаслідок УЗО слабша, ніж для неопромінених, бо радіація стимулювала релаксацію напруг)\cite{Parchinskii2006r}
%
%Відновлення електропровідності опроміненого кремнію \cite{OstrovRadSi}
%
%відновлення (до 70 відсотків) часу життя в опроміненому (гамма-кванти 60Со, 5е6--2е7 рад) Fz-Si,
%механізм: звільнення вакансій з Е--центрів і їх подальше захоплення на стоки \cite{Podolian2012r}
%
%низькотемпературний відпал рад дефектів в Cz--Si (розпад комплексів C--O--V2) \cite{PodolHivr}
%
%перебудова рад дефектів (гама кванти) в кремнії внаслідок УЗО.. правда зміни оборотні \cite{YOlikh2006TPLr}
%
%Опромінення гамма--квантами 60Со, 1е6 рад кремнієві сонячні елементи,
%УЗО приводить до відновлення характеристик, причина --- підвищення однорідності структури та перерозподіл радіаційних дефектів
%\cite{YOlikh2007TPLr}
%
%
%\cite{OstrovRadSi,Podolian2012r,PodolHivr,YOlikh2007TPLr}, германію \cite{Olikh:FTP1996},
%напівровідникових \cite{OlikhProc,OstrovFTTRad} та лужногалоїдних \cite{UST:OstrovCsI} сполук.


Залишкові ефекти не завжди є стабільними.
Наприклад, кристали з АІ зміною провідності та  CuInSe$_2$ сонячні елементи відновлюють свої попередні властивості після зберігання при кімнатній температурі на протязі декількох діб \cite{YOlikh2006TPLr,US:ZnCdTe,BorkovFTT,OstapSC},
комплексоутворення зруйнованих під дією УЗО домішкових пар чи перебудованих радіаційних дефектів відбувається протягом десятків хвилин \cite{Ostapenko1995SST,Ostapenko1995,YOlikh2006TPLr},
характерний час відновлення параметрів InAs $p$--$n$ переходів --- декілька місяців \cite{Teterkin2009r}.



Наведені результати свідчать, що можливості УЗО охоплюють широкий спектр як напівпровідникових матеріалів, та і їх властивостей.
Отримані за допомогою УЗО результати нерідко можна продублювати з використанням більш технологічно звичних методів на кшталт відпалу чи радіаційного опромінення.
Проте необхідно зауважити, що використання УЗ нерідко має і свої переваги, пов'язані з локалізацією впливу:
наприклад, ступінь релаксації внутрішніх напруг при УЗО більш глибокий, ніж при радіаційному опроміненні \cite{UST:GeGaAs1990},
а акустовідпал радіаційних дефектів \cite{PodolHivr,UST:OstrovCsI,YOlikh2007TPLr} відбувається при температурах, недостатніх для суттєвої дифузії легуючих домішок, а отже і розмиття профіля легування, який може супроводжувати процес звичайного термовідпалу.

Значна тривалість та висока інтенсивність УЗО не завжди є доречними з технологічної точки зору.
На думку автора, більш перспективним для практичного застосування є використання УЗ не як основного інструменту керування властивостями мікроелектронних структур, а як додаткового фактору впливу під час класичних технологічних операцій.
Підгрунтям для подібного припущення є те, що за таких умов напівпровідникові структури нерідко опиняються у нерівноважному стані і їх дефектно--домішкова підсистема більш легко може бути перебудована (модифікована) під дією пружних коливань.
Тобто йде мова про те, що УЗ може виконувати лише керуючу роль, в той час як переважні енергетичні затрати
лягають на плечі радіаційної чи термічної обробок.
Застосовувати АХ значно меншої інтенсивності дозволить ще більше підвищити локалізованість впливу саме на дефектах.

Чи не найяскравішим експериментальним доказом даного припущення є результати, отримані при
використанні УЗ одночасно з іонною імплантацією, яка мала на меті легування або формування аморфного чи діелектричного шарів \cite{US:ImplantUFJ2015,US:ImplantUFJ2001,US:ImplantUFJ2008,ROMANYUK2005,Roman2006,RomanyukSST,
YOlikh2005,ROMANJUK2005MatSci,USImplant:JVacSci}.
Зокрема показано, що за наявності АХ
підсилюється процес аморфізації поверхневого шару кремнію \cite{RomanyukSST,US:ImplantUFJ2001},
відбувається зменшення механічних напруг біля внутрішніх границь \cite{US:ImplantUFJ2008,ROMANJUK2005MatSci}
та концентрації дефектів міжвузлового типу в області збіднення $p$--$n$ переходів \cite{YOlikh2005},
створюються умови для формування ультра--мілких переходів \cite{USImplant:JVacSci},
покращуються властивості та зменшується товщина шару WO на поверхні $p$--Si \cite{ROMANYUK2005,Roman2006}.

АХ можуть використовуватися не лише під час радіаційного опромінення, але й бути частиною будь--якої обробки, пов'язаної зі зміною дефектної підсистеми.
Наприклад, застосування УЗ паралельно з лазерною обробкою систем Fe--Si--C викликає суттєве (в 2--3 рази) зменшення залишкового аустеніту порівняно з використанням лише лазерного опромінення \cite{US:FeSiC};
використання УЗО під час виготовлення поруватого кремнію, люмінофорів на основі ZnS чи осаджені ZnO --- певне структурне впорядкування \cite{Kalem2000}, послаблення фото-- та посилення електролюмінесценцій \cite{Wang:JLum} чи підвищення однорідно плівок \cite{US:ZnOfilm}, відповідно.



Збудження УЗ під час радіаційного опромінення кристалів може бути фактором підвищення радіаційної стійкості напівпровідників завдяки
а)~підвищенню швидкості рекомбінації (акустовідпалу) первинних радіаційних дефектів (РД);
б)~створенню електрично та рекомбінаційно неактивних вторинних РД внаслідок реакцій між первинними РД та домішками --- процеси дифузії та перебудови РД також є акустоактивованими \cite{YOlikh2006TPLr,Parchinskii2000r}.
Крім того, ще одним позитивним фактором впливу АХ може бути стимуляція переведення занурених радіаційним чином домішок у положення, що відповідає електричноактивному стану --- наприклад, іонів--легантів у вузлове положення.


Метод іонної імплантації використовується не лише для легування напівпровідників чи створення поверхневого шару із заданими властивостями, але й для формування різноманітних кластерів в матриці різної природи.
У цьому випадку УЗ також може бути додатковим позитивним фактором впливу, про що свідчать результати робіт \cite{Roman:2006JAP,Roman:2007APL,Roman:2010JAP,YOlikh2010JL}.
Наприклад, було показано, що наявність супроводжуючого УЗ навантаження викликає збільшення розмірів занурених металевих кластерів у кремнії \cite{Roman:2006JAP} та оксиді кремнію \cite{Roman:2007APL}, зміни парамагнітних \cite{Roman:2010JAP} та фотолюмінесцентних \cite{YOlikh2010JL} властивостей нанокластерів Si в SiO$_2$.
Зауважимо, що акустичні коливання активно використовуються не лише під час створення наночастинок у кристалічній матриці.
Так достатньо широко застосовується УЗ під час хімічного синтезу різноманітних за природою (CdS, ZnO, CdSe, Au, Cu, Al$_2$O$_3$, NiS) наночастинок у розчинах (див., наприклад, роботи \cite{US:nanoParticle,US:nanoCdSe,US:nanoCu,US:nanoCdS,US:nanoCdSNiS,US:nanoZnOAu} та посилання в них),
що дозволяє покрашити якість кінцевого виробу (зменшити розміри частинок, підвищити їх паруватість тощо).
Теоретично та експериментально досліджується можливість самоорганізації наночастинок, зокрема вуглецевих, безпосередньо в акустичному полі  \cite{US:nanoAPL2016}.



\section{Динамічні акустоіндуковані ефекти}

Іншим перспективним напрямом досліджень є пошук можливості використання УЗ для динамічної зміни властивостей напівпровідникових пристроїв під час їх роботи,
або й, взагалі, практичне використання ефектів, які відбуваються в системах лише за умов поширення в них пружних коливань.
Зокрема, перший підхід може бути розглянутий в рамках функціональної електроніки --- галузі,
яка використовує для керування інформаційним сигналом неоднорідності середовища,  що виникають під дією керуючого сигналу.
Саме роль керуючого сигналу і може виконувати АХ, динамічно змінюючи стан дефектів і, таким чином, реалізуючи можливість створення акустокерованих пристроїв.
Наприклад, для створення генератора або частотно--вибіркового підсилювача використовують транзистори, резистори, конденсатори.
При використанні підходу інтегральної схемотехніки всі ці елементи реалізуються на базі напівпровідників.
Зміна провідності кристалу, і, відповідно, номіналу елементу (у найпростішому випадку --- опору резистора) під дією УЗ
має викликати керовану (наприклад, інтенсивністю чи частотою УЗ) перебудову робочої частоти (частоти зворотнього зв'язку) пристрою.

Можливість реалізації подібного підходу підтверджують, зокрема, результати спостереження оборотного
збільшення електропровідності у кремнії \cite{YOlikhTPL2011r}, кристалах CdTe:Cl \cite{YOlikh:UFG2016,YOlikh:SupMicr} та КРТ \cite{OlikhYFTP99,OlikhYFTP2000} під час поширення УЗ.
Зміни концентрації носіїв та їх рухливості пов'язуються з АІ
збільшенням ефективного радіуса дислокаційних кластерів та дифузійною перебудовою системи точкових дефектів всередині них\cite{YOlikh:UFG2016,YOlikh:SupMicr},
зі звільненням зв'язаних дефектів донорного типу та сгладжування розсіюючого потенціалу \cite{OlikhYFTP99,OlikhYFTP2000}
чи перебудовою метастабільних дефектів \cite{YOlikhTPL2011r}.
Зміною стану бістабільного центру Fe--B під час УЗН пояснюються і динамічні зміни довжини вільного пробігу електронів в кристалах кремнію авторами робіт \cite{Ostrovskii2001,OlikhFTT}.
У кремнії також виявлено підсилення емісії електронів з донорних домішок, пов'язане з їх зміщенням відносно оточення \cite{Korotchenkov1995}.
Зауважимо, АІ зміни стану окремих точкових дефектів є перспективними з точки зору створення запам'ятовуючих пристроїв.
Відомо \cite{MetaUFN}, що бістабільний дефект є перспективним елементом пам'яті нового покоління,
дві конфігурації якого відповідають логічним <<0>> та <<1>>.
В такому випадку УЗ може виступати інструментом перемикання подібного елементу.
Зокрема, гітотетичний механізм переведення дефекту між станами може бути наступним:
при поширенні АХ дефект зміщується відносно свого оточення, це викликає зміни його симетрії, що, згідно з ефектом Яна--Теллера, може спричинити перезарядку центру.
Нерідко у різних зарядових станах мінімальній енергії відповідають різні просторові конфігурації, що і дозволяє зафіксувати новий логічний стан.
Окрім вже згаданих вище ефектів динамічної конфігураційної перебудови центрів у кремнії, на користь
можливості такого механізму свідчить виявлена АІ трансформація DX--центру в плівках Al$_x$Ga$_{1-x}$As \cite{belyaev1994}.
Звичайно, подібний підхід вимагає, в першу чергу, вирішити задачу по локалізації УЗ впливу на окремому дефекті.

Також необхідно врахувати можливість оборотного УЗ впливу на процеси поширення нерівноважних носіїв.
Мова йде про АІ перебудову (та/або перезарядку)  центрів рекомбінації (прилипання), що має викликати зміну перерізу захоплення та часу життя носіїв.
Наприклад, експериментальному дослідженню впливу УЗ на нерівноважні носії у арсенід галієвих фотоприймачах присвячена робота \cite{Zaveryukhin2002:2}.

зменшення інтенсивності світіння GaP світлодіодів при УЗН пов'язане з руйнуванням екситонів в результаті вимушених коливань дислокацій та виникненння нерівноважних дислокаційних скупчень \cite{USL:GaP}

підвищення фотонапруги та зміни часової залежності її релаксації в гетероструктурах SiGe/Si \cite{Ostrovskii2001}
Вплив ультразвуку на вольт--амперні характеристики гетероструктур GaAs/AlGaAs \cite{Olikh:Visn2007}

Підкреслимо, що в цьому випадку мова не йд






широкий спектр оптичних явищ --- поява випромінювання, зміни спектрів фотопровідності --- огляд \cite{OstrBook,OSTROVSKII1999}
зміна спектрів фотопровідності в ZnS, ZnSe внаслідок генерації (звільнення з дислокацій) власних дефектів ультразвуком надпорогової інтенсивності \cite{OSTROVSKII1992}
сонолюмінісценція гранульованого середовища (переважно частинки ZnS мікрометрового розміру) \cite{KorotRep},
Трохи випереджаючи події (розгляду методів характеризації напівпровідників за допомогою УЗ присвячено параграф~\ref{secUSMethod}),
зауважимо що просторовий розподіл інтенсивності сонолюмінісценції може бути викоританий для характеризації ступеню пакування гранульованих систем\cite{KorotZnSdens}.

%зміна електропровідності кристалів CdTe:Cl, при високих температурах (>200 К) невелике зменшення рухливості і невелике зростання кількості електронів,
%при низьких --- зростання рухливості, значне (до 20\%) зростання концентрації електронів,
%причини --- збільшення ефективного радіуса дислокаційних кластерів та дифузійна перебудова точково дефектної структури всередині кластера \cite{YOlikh:UFG2016,YOlikh:SupMicr}

зміна спектру (інтенсивності?) випромінювання лазерних структур InGaAsP/InP \cite{Kulakova:09,KulacFTT2009,KulakJETF2007},
перебудова частоти \cite{KulakPJETF}
обертання площини поляризації гетеролазер з квантовими ямами \cite{Kulakova:2012SSC,KulakFTP2013}
керування напрямом випромінювання \cite{KulakPJTF2010}

зміни часу життя екситонів при УЗН, що відбиваєть на інтенсивності фотолюмінісценції,
причини --- дія п'єзополя, яке викликає зміни пложення дефектів чи дисоціацію екситонів \cite{KOROTCHENKOV1998,KorotchenAPL1998}
знову п'єзополе -- зменшення інтеніивності домішкової та екситонної фотолюмінісценції в GaAs \cite{Zhuravlev}




квантові ями ZnSe/ZnS, п'єзополе, вплив на екситони (їх просторове переміщення) --- пришвидшення спаду фотолюмінісценції \cite{KorotAPL1999,Ostrovskii2001}
КЯ на основі GaAs розділення носіїв п'єзополем --- дисоціація екситонів --- зміна інтенсивності фотолюмінісценції \cite{PhysRevLett78}

перетворення зв'язаних екситонів на вільні в GaAs/AlAs квантових ямах \cite{PhysRevB80:165307}

переміщення доменних стінок в магнітних нанодротах (рухи використовуються в елементах пам'яті)теор робота\cite{US:nanowire}
контроль намагніченості стінки (магнітозапис) під дією АХ \cite{LI2014}
поява спінового струму в плівках \cite{PhysRevLett108:176601}
керування феромагнітним резонансом в плівках магнітних напівпровідників \cite{PhysRevB90:094401}
вплив на одноелектронний транспорт в GaAs/AlGaAs\cite{US:single}
акустоелектричний струм в одноелектронних пристроях AlGaAs/GaAs \cite{US:single2}
одноелектронне перенесення з однієї квантової точки в іншу \cite{US:Nature}
акустоелектричний струм в одномірному каналі\cite{US:1D,NETO2016}
акустоелектричний струм в графені (посилання про АЕ явища в наноматеріалах) \cite{US:graphen}
все-- п'єзополе, ГГц





ідентифікація дефектів на границі SiGe/Si внаслідок їх спустошення (зарядове) (вплив на DLTS спектри) \cite{KorotchFTP1996}
модуляція спектра відбиття від епітаксійних плівок GaAs \cite{KorotFTP1994,OSTROVSKII2000,Ostrovskii2001}, спектрів пропускання CdS \cite{KorotFTT93} (іонізація, перезарідка комплексів) --- різницева спектроскопія, в різницевих спектрах піки

п'єзополе --- збурення GaAs/AlGaAs multiple quantum well, різнецева спектроскопія (спектри відбивання)  \cite{SST:USmethod}

визначення розташування дефектів в кремнієвих пластинах по їх пружних коливаннях \cite{BELYAEV2001}
підвищення температури в околі дислокацій в КРТ \cite{savkina2004disl,SAVKINA2005disl}


%виготовлення наночастинок в розчині хім методом + УЗ -- зменшення розміру, збільшення паруватості.. є ще посилання \cite{US:nanoParticle}
%самоорганізація наночастинок в УЗ полі на прикладі вуглецевих\cite{US:nanoAPL2016}
%під час осадження плівок ZnO --- покращується їх якість \cite{US:ZnOfilm}





Зауважимо, що в цьому розділі були порушені можливості динамічного використання УЗ, пов'язані з акусто--дефектною взаємодією.
Водночас відомо, що АХ можуть бути причиною і багатьох інших цікавих явищ, на кшталт просторове розділення заряду у квантових ямах \cite{Kuryliuk2009}, квантова телепортація \cite{Buscemi}, маніпуляція наночастинками \cite{Cuberes,Olikh:SPQEO2010}.

Таким чином, застосування активного, динамічного УЗ у різноманітних областях, пов'язаних з виготовленням та застосуванням різноманітних мікроелектронних пристроїв є достатньо перспективним та потребує ретельного експериментального та теоретичного дослідження.


\section{Використання ультразвуку до характеризації дефектної структури напівпровідників\label{secUSMethod}}




Основні результати даного розділу представлені в роботах \cite{Olikh:SEMT2004,Olikh:SEMT2011,1UNCPS,2013Buk}.

