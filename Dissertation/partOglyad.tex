%\chapter{Акустоіндуковані ефекти у мікроелектронних структурах та матеріалах\label{Oglyad}}
\chapter{Передумови та перспективи використання активного ультразвуку в мікроелектроніці\label{Oglyad}}
\section{Залишкові акустоіндуковані ефекти}

%\cite{Ostap:SiO2} стан границі Si--SiO2, підсилення дифузії водню
%\cite{Ostap:PhotoLum,ostapenko1997} підсилення дифузії водню
%
%дефектний стан границі Si--SiO2,\cite{UST:Medvid}
%
%УЗО опромінених електронами GaAs-GaP LEDs,
%підсилення інтенсивності люмінісценції внаслідок поглинання невипромінювальних центрів викликаних рухом дислокацій в УЗ полі (часткове відновлення після радіаційної деградації)
%\cite{US:LED,UST:LED_SM}
%
%відпал радіаційних дефектів в лужногалоїдних кристалах внаслідок їх дифузії\cite{UST:OstrovCsI}
%
%структура Si--SiO2, (100), n-тип, опір 0,2 Ом см, опромінення гамма--квантами 60Со, 1е6 рад,C--V виміри
%УЗО веде до зменшення ефективного поверхневого заряду (АІ дифузія нестабільних рад дефектів в полі пружних напруг системи Si--SiO2)\cite{Parchinskii2000r}
%зменшення генераційного часу життя в області Si, що прилягає до контакту(збільшення концентрації центрів, внаслідок їх звільнення з домішкових асоціатів),
%незначні зміни швидкості поверхневої рекомбінації (перебудова
%приповерхневої області внаслідок УЗО слабша, ніж для неопромінених, бо радіація стимулювала релаксацію напруг)\cite{Parchinskii2006r}
%
%для неопромінених збільшення часу і зменшення швидкості внаслідок перебудови дефектної структури, яка виявляється більш яскраво через наявність пружних полів \cite{Parchinskii2003r}
%
%Відновлення електропровідності опроміненого кремнію \cite{OstrovRadSi}
%
%відновлення (до 70 відсотків) часу життя в опроміненому (гамма-кванти 60Со, 5е6--2е7 рад) Fz-Si,
%механізм: звільнення вакансій з Е--центрів і їх подальше захоплення на стоки \cite{Podolian2012r}
%
%низькотемпературний відпал рад дефектів в Cz--Si (розпад комплексів C--O--V2) \cite{PodolHivr}
%
%перебудова рад дефектів (гама кванти) в кремнії внаслідок УЗО.. правда зміни оборотні \cite{YOlikh2006TPLr}
%
%Опромінення гамма--квантами 60Со, 1е6 рад кремнієві сонячні елементи,
%УЗО приводить до відновлення характеристик, причина --- підвищення однорідності структури та перерозподіл радіаційних дефектів
%\cite{YOlikh2007TPLr}
%
%
%
%АІ зміна густини поверхневих станів на границі Si--SiO2
%(малі потужності -- зменшення, великі --- збільшення) \cite{Zaver:2008r}
%
%\cite{OstrovRadSi,Podolian2012r,PodolHivr,YOlikh2007TPLr}, германію \cite{Olikh:FTP1996}, 
%напівровідникових \cite{OlikhProc,OstrovFTTRad} та лужногалоїдних \cite{UST:OstrovCsI} сполук.


\section{Динамічні акустоіндуковані ефекти}


Основні результати даного розділу представлені в роботах \cite{Olikh:SEMT2004,Olikh:SEMT2011,1UNCPS,2013Buk}.

