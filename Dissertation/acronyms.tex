\chapter*{Перелік умовних скорочень та позначень}             % Заголовок
\addcontentsline{toc}{chapter}{Перелік умовних скорочень та позначень}  % Добавляем его в оглавление
\noindent
%\begin{longtabu} to \dimexpr \textwidth-5\tabcolsep {r X}
\begin{longtabu} to \textwidth {r X}
  DE & differential evolution, метод диференційної еволюції \\
  NIEL & non--ionizing energy losses, енергетичні втрати, не пов'язані з іонізацією \\
  MABC & modified artificial bee colony, метод модифікованої штучної бджолиної сім'ї\\
  PSO & particle swarm optimization, метод оптимізації зграї частинок\\
  RT & running time, час, необхідний для визначення параметрів\\
  TLBO & teaching learning based optimization, метод  оптимізованого викладання та навчання\\
  АІ & акусто--індукований\\
  АХ & акустична хвиля\\
  АЧХ & амплутудно--частотна характеристика\\
  ВАХ & вольт--амперна характеристика\\
  ВБШ & висота бар'єру Шотки\\
  ВФХ & вольт--фарадна характеристика\\
  ДШ & діод Шотки\\
  ЕА & еволюційний алгоритм\\
  КСЕ & кремнієвий сонячний елемент\\
  MH & метал--напівпровідник \\
  ОПЗ & область просторового заряду \\
  РД & радіаційний дефект \\
  УЗ & ультразвук \\
  УЗН & ультразвукове навантаження \\
  УЗО & ультразвукова обробка \\
$\alpha$ & коефіцієнт поглинання світла  \\
$\beta$ & коефіцієнт квантового виходу  \\
$\beta_1$, $\beta_2$  & коефіцієнти Варшні  \\
$\Delta P$ & абсолютна АІ зміна параметра $P$\\
$\varepsilon$ & діелектрична проникність матеріалу  \\
$\varepsilon_0$ & діелектрична стала \\
$\varepsilon_P$ & відносна AI зміна параметра $P$\\
$\xi_\mathtt{US}$& амплітуда деформації ґратки при поширенні УЗ\\
$\lambda$ &довжина хвилі падаючого світла\\
$\rho_\mathtt{LNO}$ & густина ніобату літію\\
$\rho_\mathtt{Si}$ & густина кремнію\\
$\tau_{g}$ &ефективний час життя носіїв заряду в ОПЗ\\
$\tau_{n}$ &ефективний час життя електронів\\
$\upsilon_\mathtt{LNO}$ & швидкість звуку в ніобаті літію\\
$\upsilon_\mathtt{Si}$ & швидкість звуку в кремнії\\
$\Phi_b$ & ВБШ при нульовому зміщенні\\
$\Psi$ & флюєнс опромінення\\
$A$ & площа зразка \\
$A_\mathtt{LNO}$ & площа п'єзоперетворювача\\
$A^*$ & ефективна стала Річардсона \\
$c$ & швидкість світла\\
$D$ & доза опромінення\\
$D_d$ & displacement damage dose, ефективна доза, пов'язана з дефектоутворенням\\
$E_g$ & ширина забороненої зони\\
$F\!F$ & фактор форми освітленої ВАХ СЕ\\
$f_r$& резонансна частота п'єзоперетворювача\\
$f_\mathtt{US}$& частота УЗ\\
$h$ & стала Планка\\
$I$ & струм\\
$I_s$ & струм насичення\\
$J$ & густина струму\\
$J_{ph}$ & густина фотогенерованого струму\\
$J_{sс}$ & густина струму короткого замикання\\
$k$ & стала Больцмана\\
$L_n$ & довжина дифузії електронів\\
$N_c$ & ефективна густина станів біля дна зони провідності\\
$N_v$ & ефективна густина станів біля вершини валентної зони\\
$n_i$ & концентрація власних носіїв заряду\\
$n_\mathrm{id}$ & фактор неідеальності\\
$n_n$ & концентрація основних носіїв у електронному напівпровіднику \\
$n_p$ & концентрація неосновних носіїв у дірковому напівпровіднику \\
$q$ & елементарний заряд\\
$p_n$ & концентрація неосновних носіїв у електронному напівпровіднику \\
$p_p$ & концентрація основних носіїв у дірковому напівпровіднику \\
$R_s$ & послідовний опір\\
$R_{sh}$ & шунтуючий опір\\
$T$ & абсолютна температура\\
$u_\mathtt{US}$&амплітуда зміщень атомів при поширенні УЗ\\
$V$ & напруга\\
$V_d$ & падіння напруги в околі бар'ру\\
$V_{oc}$ & напруга холостого ходу\\
$V_R$ & зворотна напруга\\
$V_\mathtt{RF}$ & амплітуда високочастотної напруги, прикладеної до п'єзоперетворювача\\
$W_{ph}$ & інтенсивність освітлення \\
$W_\mathtt{US}$ & інтенсивність акустичної хвилі\\

\end{longtabu}
\addtocounter{table}{-1}% Нужно откатить на единицу счетчик номеров таблиц, так как предыдующая таблица сделана для удобства представления информации по ГОСТ





