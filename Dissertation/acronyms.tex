\chapter*{Перелік умовних скорочень та позначень}             % Заголовок
\addcontentsline{toc}{chapter}{Перелік умовних скорочень та позначень}  % Добавляем его в оглавление
\noindent
%\begin{longtabu} to \dimexpr \textwidth-5\tabcolsep {r X}
\begin{longtabu} to \textwidth {r X}
  CDLR& coupled defect level recombination,  рекомбінація у системі спарених рівнів дефектів\\
  DAT & defect--assisted tunneling, тунелювання за участю рівнів дефектів \\
  DE & differential evolution, метод диференційної еволюції \\
  FRC & fast--formed recombination center, швидко сформовані ВО дефекти \\
  NIEL & non--ionizing energy losses, втрати, не пов'язані з іонізацією \\
  MABC & modified artificial bee colony, метод  штучної бджолиної сім'ї\\
  OSFR & oxidization induced stacking--faults ring, кільцеві дефекти пакування, що виникли при окисненні \\
  PAT & phonon-assisted tunneling, стимулюване фононами тунелювання \\
  PSO & particle swarm optimization, метод оптимізації зграї частинок\\
  RT & running time, час, необхідний для визначення параметрів\\
  SCLC & space-charge limited current, струм, обмеженим просторовим зарядом \\
  SRC & slow--formed recombination center, повільно сформовані ВО дефекти\\
  TLBO & teaching learning based optimization, метод  оптимізованого викладання та навчання\\
  VRHC &thermally-assisted variable-range-hopping conduction, термічно--активована стрибкова провідність зі змінною довжиною стрибка \\
  AAД & акустоактивний дефект\\
  АДВ & акусто--дефектна взаємодія \\
  АІ & акустоіндукований\\
  АХ & акустична хвиля\\
  АЧХ & амплітудно--частотна характеристика\\
  ВАХ & вольт--амперна характеристика\\
  ВБШ & висота бар'єру Шотки\\
  ВТКС & високотемпературна компонента струму\\
  ВФХ & вольт--фарадна характеристика\\
  ГР &глибокий рівень \\
  ДШ & діод Шотки\\
  ЕА & еволюційний алгоритм\\
  КНО &  квазі--нейтральна область \\
  КП & кисневмісні преципітати\\
  КСЕ & кремнієвий сонячний елемент\\
  MH & метал--напівпровідник \\
  MОH & метал--окис--напівпровідник \\
  МХО & мікро--хвильова обробка\\
  НВЧ & надвисокочастотне \\
  НТКС & низькотемпературна компонента струму\\
  ОПЗ & область просторового заряду \\
  ПАН & поперечна акустоелектрична напруга\\
  ПЕ & польова емісія\\
  ППЗ & поперечний переріз захоплення \\
  РД & радіаційний дефект \\
  ТД &точковий дефект \\
  ТЕ & термоелектронна емісія \\
  ТПЕ & термопольова емісія \\
  УЗ & ультразвук \\
  УЗН & ультразвукове навантаження \\
  УЗО & ультразвукова обробка \\
  ШРХ & теорія Шоклі--Ріда--Хола  \\
$\alpha$ & коефіцієнт поглинання світла  \\
$\alpha_R$ & температурний коефіцієнт опору\\
$\alpha_\mathrm{\,FB}$ & температурний коефіцієнт ВБШ в наближені плоских зон\\
$\beta$ & коефіцієнт квантового виходу  \\
$\beta_1$, $\beta_2$  & коефіцієнти Варшні  \\
$\Delta P$ & абсолютна АІ зміна параметра $P$\\
$\varepsilon$ & діелектрична проникність матеріалу  \\
$\varepsilon_0$ & діелектрична стала \\
$\varepsilon_P$ & відносна AI зміна параметра $P$\\
$\xi_\mathtt{cur}$ & відносна деформація приповерхневих кристалічних площин\\
$\xi_\mathtt{US}$& амплітуда деформації ґратки при поширенні УЗ\\
$\vartheta$ & темп генерації РД\\
$\lambda$ &довжина хвилі падаючого світла\\
$\rho_\mathtt{LNO}$ & густина ніобату літію\\
$\rho_\mathtt{Si}$ & густина кремнію\\
$\sigma_{\Phi0}$ & стандартне відхилення висоти бар'єру при нульовому зміщенні\\
$\sigma_n$& поперечний переріз захоплення електронів дефектом\\
$\sigma_p$& поперечний переріз захоплення дірок дефектом\\
$\tau$ & час релаксації заряду на пастках\\
$\tau_{g}$ &ефективний час життя носіїв заряду в ОПЗ\\
$\tau_{n}$ &ефективний час життя електронів\\
$\tau_{n,\mathtt{RD}}$ & час життя електронів при рекомбінації на РД\\
$\upsilon_\mathtt{LNO}$ & швидкість звуку в ніобаті літію\\
$\upsilon_{\mathrm{th},n}$ & теплова швидкість електронів\\
$\upsilon_{\mathrm{th},p}$ & теплова швидкість дірок\\
$\upsilon_\mathtt{Si}$ & швидкість звуку в кремнії\\
$\Phi_b$ & ВБШ при нульовому зміщенні\\
$\Phi_{b}^0$ & середнє значення ВБШ при нульовому зміщенні (ВБШ в однорідній області) \\
$\Phi_{b}^\mathrm{FB}$ & ВБШ в наближені плоских зон \\
$\Psi$ & флюєнс опромінення\\
$\phi_0$ & рівень нейтральності інтерфейсних станів у структурі МН\\
$\zeta$ & диференційний показник нахилу ВАХ \\
$\omega_{ph}$ & частота фонону\\
$\omega_\mathtt{US}$ & циклічна частота АХ\\
$A$ & площа зразка \\
$A_\mathtt{LNO}$ & площа п'єзоперетворювача\\
$A^*$ & ефективна стала Річардсона \\
$a$ & стала ґратки \\
$a_B$ & радіус Бора\\
$B$ & коефіцієнт динамічної в'язкості \\
$b$ & модуль вектора Бюргерса \\
$C$ & ємність діоду Шотки\\
$c$ & швидкість світла\\
$D$ & доза опромінення\\
$D_d$ & displacement damage dose, ефективна доза, пов'язана з дефектоутворенням\\
$D_{ss}$ & густина інтерфейсних станів у структурі МН\\
$E_g$ & ширина забороненої зони\\
$E_i$ & положення рівня Фермі у власному напівпровіднику\\
$E_t$ & положення енергетичного рівня, зв'язаного з дефектом\\
$F\!F$ & фактор форми КСЕ\\
$F_m$ & напруженість електричного поля на границі розділу метал-напівпровідник \\
$f_r$& резонансна частота п'єзоперетворювача\\
$f_\mathtt{US}$& частота УЗ\\
$G$ & модуль зсуву \\
$h$, $\hbar$ & стала Планка\\
$I$ & струм\\
$I_s$ & струм насичення\\
$I_R$ & зворотний струм\\
$J$ & густина струму\\
$J_{ph}$ & густина фотогенерованого струму\\
$J_{sс}$ & густина струму короткого замикання\\
$k$ & стала Больцмана\\
$L_n$ & довжина дифузії електронів\\
$m*$ &  ефективна маса електрону \\
$N_c$ & ефективна густина станів біля дна зони провідності\\
$N_d$ & концентрація електронів поблизу контакту МН\\
$N_{t,\mathtt{RD}}$ & концентрація радіаційних дефектів\\
$N_v$ & ефективна густина станів біля вершини валентної зони\\
$n_i$ & концентрація власних носіїв заряду\\
$n$ & концентрація електронів\\
$n_\mathrm{id}$ & фактор неідеальності\\
$n_n$ & концентрація основних носіїв у електронному напівпровіднику \\
$n_p$ & концентрація неосновних носіїв у дірковому напівпровіднику \\
$q$ & елементарний заряд\\
$p$ & концентрація дірок \\
$p_n$ & концентрація неосновних носіїв у електронному напівпровіднику \\
$p_p$ & концентрація основних носіїв у дірковому напівпровіднику \\
$R$ & темп рекомбінації \\
$R_\mathtt{cur}$ & радіус кривизни зразка \\
$R_{\mathtt{DA}}$ & параметр зв'язку у моделі CDLR\\
$R_{ph}$ & коефіцієнт відбивання світла\\
$R_s$ & послідовний опір\\
$R_{sh}$ & шунтуючий опір\\
$T$ & абсолютна температура\\
$T_0$ & константа температурної залежності фактора неідеальності\\
$T_\mathtt{US}$ & період АХ\\
$t$ & час\\
$t_\mathtt{MWT}$ & час експозиції при МХО\\
$t_\mathtt{UST}$ & час експозиції при УЗО\\
$u_\mathtt{US}$&амплітуда зміщень атомів при поширенні УЗ\\
$V$ & напруга\\
$V_{bb}$ & вигин зон напівпровідника поблизу контакту\\
$V_d$ & падіння напруги в околі бар'ру\\
$V_n$ & різниця потенціалів між дном зони провідності та положенням рівня Фермі в об'ємі напівпровідника\\
$V_{oc}$ & напруга холостого ходу\\
$V_R$ & зворотна напруга\\
$V_\mathtt{TAV}$ & величина ПАН\\
$V_\mathtt{RF}$ & амплітуда високочастотної напруги, прикладеної до п'єзоперетворювача\\
$V_v$ & об'єм кристалу\\
$W_{ph}$ & інтенсивність освітлення \\
$W_\mathtt{US}$ & інтенсивність акустичної хвилі\\

\end{longtabu}
\addtocounter{table}{-1}% Нужно откатить на единицу счетчик номеров таблиц, так как предыдующая таблица сделана для удобства представления информации по ГОСТ





