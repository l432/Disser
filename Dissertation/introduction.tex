%\chapter*{Вступ}							% Заголовок
\chapter*{
\textcolor{white}{[1--25]}
Вступ
\textcolor{white}
{\cite{Olikh2018JAP,Olikh2018SM,Olikh:Ultras2016,Olikh2016JSem,
Olikh:Rev,OlikhJAP,Olikh:Ultras,Olikh:UPJ2014,
Olikh:2013IEEE,Olikh:SEMT2013,Olikh:FTP2013,Olikh:UPJ2013,
Olikh:FTP2011,Olikh:SEMT2011,Olikh:UPJ2010,Gorb2010,Olikh:FTP2009,
Olikh:SEMT2007,Olikh:MRS2007,Olikh:PZTF2006,
Olikh:PhChOM2005,Olikh:PJE2004,Olikh:SEMT2004,Olikh:SPQEO2003,
Olikh:Visn2003,
1UNCPS,3Tomsk,1SEMST,50IUFFC,9APTTE,2005IUS,ICU2007SC,ICU2007GA,2007MRS,3UNCPS,6DrogGorb,6Drog,
4UNCPS,4Kremen,7Drog,5UNCPS,2012Ternop,14Plivk,8Drog,2013Buk,6UNCPS,2014IUSOl,2014IUS,6SEMST,
2015ICU,6CPFCS,7UNCPS,2017MEICS}}
}							% Заголовок
\addcontentsline{toc}{chapter}{Вступ}	% Добавляем его в оглавление

%обґрунтування вибору теми дослідження (висвітлюється зв’язок теми дисертації із сучасними дослідженнями у відповідній галузі знань шляхом критичного аналізу з визначенням сутності наукової проблеми або завдання);
%
%мета і завдання дослідження відповідно до предмета та об’єкта дослідження;
%
%методи дослідження (перераховуються використані наукові методи дослідження та змістовно відзначається, що саме досліджувалось кожним методом; обґрунтовується вибір методів, що забезпечують достовірність отриманих результатів та висновків);
%
%наукова новизна отриманих результатів (аргументовано, коротко та чітко представляються основні наукові положення, які виносяться на захист, із зазначенням відмінності одержаних результатів від відомих раніше);
%
%особистий внесок здобувача (якщо у дисертації використано ідеї або розробки, що належать співавторам, разом з якими здобувачем опубліковано наукові праці, обов’язково зазначається конкретний особистий внесок здобувача в такі праці або розробки; здобувач має також додати посилання на дисертації співавторів, у яких було використано результати спільних робіт);




%{\actuality} Обзор, введение в тему, обозначение места данной работы в
%мировых исследованиях и~т.\:п., можно использовать ссылки на~другие
%работы\ifnumequal{\value{bibliosel}}{1}{~\autocite{Gosele1999161}}{}
%(если их~нет, то~в~автореферате
%автоматически пропадёт раздел <<Список литературы>>). Внимание! Ссылки
%на~другие работы в разделе общей характеристики работы можно
%использовать только при использовании \verb!biblatex! (из-за технических
%ограничений \verb!bibtex8!. Это связано с тем, что одна
%и~та~же~характеристика используются и~в~тексте диссертации, и в
%автореферате. В~последнем, согласно ГОСТ, должен присутствовать список
%работ автора по~теме диссертации, а~\verb!bibtex8! не~умеет выводить в одном
%файле два списка литературы).
%При использовании \verb!biblatex! возможно использование исключительно
%в~автореферате подстрочных ссылок
%для других работ командой \verb!\autocite!, а~также цитирование
%собственных работ командой \verb!\cite!. Для этого в~файле
%\verb!Synopsis/setup.tex! необходимо присвоить положительное значение
%счётчику \verb!\setcounter{usefootcite}{1}!.
%
%Для генерации содержимого титульного листа автореферата, диссертации
%и~презентации используются данные из файла \verb!common/data.tex!. Если,
%например, вы меняете название диссертации, то оно автоматически
%появится в~итоговых файлах после очередного запуска \LaTeX. Согласно
%ГОСТ 7.0.11-2011 <<5.1.1 Титульный лист является первой страницей
%диссертации, служит источником информации, необходимой для обработки и
%поиска документа>>. Наличие логотипа организации на титульном листе
%упрощает обработку и поиск, для этого разметите логотип вашей
%организации в папке images в формате PDF (лучше найти его в векторном
%варианте, чтобы он хорошо смотрелся при печати) под именем
%\verb!logo.pdf!. Настроить размер изображения с логотипом можно
%в~соответствующих местах файлов \verb!title.tex!  отдельно для
%диссертации и автореферата. Если вам логотип не~нужен, то просто
%удалите файл с логотипом.
%
%\ifsynopsis
%Этот абзац появляется только в~автореферате.
%Для формирования блоков, которые будут обрабатываться только в~автореферате,
%заведена проверка условия \verb!\!\verb!ifsynopsis!.
%Значение условия задаётся в~основном файле документа (\verb!synopsis.tex! для
%автореферата).
%\else
%Этот абзац появляется только в~диссертации.
%Через проверку условия \verb!\!\verb!ifsynopsis!, задаваемого в~основном файле
%документа (\verb!dissertation.tex! для диссертации), можно сделать новую
%команду, обеспечивающую появление цитаты в~диссертации, но~не~в~автореферате.
%\fi
%
%% {\progress}
%% Этот раздел должен быть отдельным структурным элементом по
%% ГОСТ, но он, как правило, включается в описание актуальности
%% темы. Нужен он отдельным структурынм элемементом или нет ---
%% смотрите другие диссертации вашего совета, скорее всего не нужен.
%
%{\aim} данной работы является \ldots
%
%Для~достижения поставленной цели необходимо было решить следующие {\tasks}:
%\begin{enumerate}
%  \item Исследовать, разработать, вычислить и~т.\:д. и~т.\:п.
%  \item Исследовать, разработать, вычислить и~т.\:д. и~т.\:п.
%  \item Исследовать, разработать, вычислить и~т.\:д. и~т.\:п.
%  \item Исследовать, разработать, вычислить и~т.\:д. и~т.\:п.
%\end{enumerate}
%
%
%{\novelty}
%\begin{enumerate}
%  \item Впервые \ldots
%  \item Впервые \ldots
%  \item Было выполнено оригинальное исследование \ldots
%\end{enumerate}
%
%{\influence} \ldots
%
%{\methods} \ldots
%
%{\defpositions}
%\begin{enumerate}
%  \item Первое положение
%  \item Второе положение
%  \item Третье положение
%  \item Четвертое положение
%\end{enumerate}
%В папке Documents можно ознакомиться в решением совета из Томского ГУ
%в~файле \verb+Def_positions.pdf+, где обоснованно даются рекомендации
%по~формулировкам защищаемых положений.
%
%{\reliability} полученных результатов обеспечивается \ldots \ Результаты находятся в соответствии с результатами, полученными другими авторами.
%
%

{\actualityTXT}
Напівпровідникові поверхнево--бар'єрні структури --- основа мікроелектроніки та сонячної енергетики, галузей, розвиток яких на сучасному етапі визначає загальний прогрес людства.
Незважаючи на все різноманіття наявних типів фотовольтаїчних перетворювачів, на ринку промислового використання переважають моно-- та полікристалічні кремнієві сонячні елементи.
Загалом, серед всіх напівпровідникових систем кремнієві структури використовують найширше.
Це зумовлено величезними запасами даного елементу, його нетоксичністю та високою технологічністю як вирощування самих кристалів, так і створення різноманітних структур.
Зокрема кремнієві структури з контактом Шотткі  застосовують при виготовленні високошвидкісних логічних та інтегральних елементів.
У цьому ж сегменті високочастотних мікроелектронних пристроїв широко представлені системи на основі арсеніду ґалію --- матеріалу, який характеризується високою рухливістю носіїв заряду.
У дисертаційній роботі наводяться результати дослідження кремнієвих сонячних елементів та структур метал---напівпровідник на основі Si та GaAs, що визначає її актуальність з прикладної точки зору.

Загальною задачею матеріалознавства є створення матеріалів та структур із заданими властивостями.
Для її реалізації необхідне чітке розуміння фізичних процесів, які відбуваються в матеріалах за різних умов.
Зокрема, умови функціонування напівпровідникових приладів нерідко передбачають наявність різноманітного радіаційного впливу.
Вивченню радіаційно--індукованих процесів у напівпровідниках присвячена значна кількість робіт, що також свідчить про актуальність подібних досліджень,
проте окремі аспекти, наприклад немонотонність зміни характеристик діодів Шотткі при дії $\gamma$--квантів чи механізми модифікації приповерхневого шару при мікрохвильовому опроміненні, досі залишалися поза увагою.
У виконаній роботі показано взаємозв'язок між ступенем неоднорідності контакту Шотткі та характером дозової немонотонності зміни висоти бар'єру, а також з'ясовано, що перетворення у дефектній структурі приповерхневого шару зумовлені збільшенням концентрації міжвузлових атомів.
Іншим зовнішнім чинником, який може впливати на параметри напівпровідникових структур, є знакозмінні пружні деформації, зумовлені, наприклад, поширенням акустичних хвиль.
У роботі вперше проведено дослідження перенесення заряду в кремнієвих бар'єрних структурах за умов ультразвукового навантаження.
Перераховані напрямки проведених досліджень свідчать про актуальність виконаної роботи в області матеріалознавства.

%Загальною задачею напівпровідникового матерiалознавства є створення матерiалiв та структур iз заданими властивостями. Фундаментальні дослідження дефектів у напівпровідниках, поглиблення розуміння їхньої поведінки мають важливе значення для розширення функціональних можливостей пристроїв та ідентифікації і усунення небажаних дефектів. Сфера інтересів фізики дефектів, як це відзначається в матеріалах останнього світового форуму по дефектах у напівпровідниках (Defects in Semiconductors Journal of Applied Physics 123, 161301 (2018); https://doi.org/10.1063/1.5036660)  зараз також охоплює і методи  зовнішньої активації технологічно функціональних дефектів та домішок для управління електричними, оптичними, тепловими та магнітними властивостями напівпровідників, що дозволяє інженерам додавати нові функції до напівпровідникових пристроїв. Одним з таких чинників є знакозмiннi високочастотнi пружнi деформацiї, зумовленi, наприклад, поширенням акустичних хвиль


Поглиблення розуміння поведінки дефектів у напівпровідниках мають фундаментальне значення для розширення можливостей відповідних пристроїв.
Як видно з матеріалів останніх світових форумів, сфера інтересів фізики дефектів охоплює методи  зовнішньої активації технологічно функціональних дефектів для управління властивостями напівпровідників.
Загальновизнаними подібними способами є опромінення та термообробка, які, проте, суттєво впливають і на стан кристала загалом.
%
%Вирішення матеріалознавчих задач потребує розробки методів керування параметрами матеріалів та структур.
%Відомо, що дефекти структури є визначальними для фізичних властивостей кристалів і мають фундаментальне значення у фізиці твердого тіла.
%Найпоширенішими способами впливу на дефектну підсистему напівпровідників є опромінення швидкими частинками та термообробка, які суттєво впливають на стан кристала в цілому.
Представлені результати свідчать про здатність ультразвукового навантаження навіть допорогової інтенсивності модифікувати дефекти у кремнієвих кристалах, причому до переваг такого підходу варто віднести вибірковість впливу саме на області з порушеннями періодичності та оборотність змін при кімнатних температурах.
Тому виконана робота актуальна з погляду розробки нових методів керування параметрами бар'єрних структур.

Вважається, що причинами змін стану точкових дефектів у напівпровідникових кристалах під дією акустичних хвиль є вимушені коливання дислокацій, акусто--стимульована дифузія домішок та генерація дефектів при надпороговій інтенсивності пружних коливань.
Проте для оборотних акусто--індукованих ефектів у малодислокаційних матеріалах
%при допороговій інтенсивності ультразвука
подібні механізми  не є визначальними.
Проведене дослідження особливостей перенесення заряду при ультразвуковому навантаженні кремнієвих структур та ідентифікація <<акустоактивних>>, тобто здатних до ефективної взаємодії з пружними коливаннями, дефектів технологічного та радіаційного походження
%, у тому числі і радіаційних,
є актуальним з погляду встановлення фізичних причин акусто--дефектної взаємодії у подібних матеріалах.

Відтак,  дослідження фізичних  закономірностей  та встановлення механізмів акусто-- та радіаційно--індукованих
ефектів у поверхнево--бар'єрних напівпровідникових структурах  є  важливим  для  вирішення  перелічених  проблем  і визначає наукову та практичну актуальність дисертаційної роботи.


{\InterconnectionTXT}
Дисертаційна робота  пов’язана із планами науково--дослідних робіт, які проводились у рамках
держбюджетних тем та міжнародних проектів на кафедрі загальної фізики фізичного факультету \thesisOfOrganization.
А саме:
\textnumero01БФ051--09 <<Теоретичне та експериментальне дослідження фізичних властивостей неоднорідних систем на основі матеріалів акусто--опто--електроніки та мікроелектроніки>>
(\textnumero~держ. реєстрації 01БФ051--09, 2001--2005рр.);
\textnumero06БФ051--04 <<Експериментальне та теоретичне дослідження структури та фізичних властивостей низькорозмірних систем на основі напівпровідникових структур, різних модифікацій вуглецю та композитів>>
(\textnumero~держ. реєстрації 0106U006390, 2006--2010рр.);
\textnumero11БФ051--01 <<Фундаментальні дослідження в галузі фізики конденсованого стану і елементарних частинок, астрономії і матеріалознавства для створення основ новітніх технологій>>
(\textnumero~держ. реєстрації 0111U004954, 2011--2015рр.);
\textnumero16БФ051--01 <<Формування та фізичні властивості наноструктурованих композитних матеріалів та функціональних поверхневих шарів на основі карбону, напівпровідникових та діелектричних складових>>
(\textnumero~держ. реєстрації  0116U004781, 2016--2018рр.) та
проект УНТЦ №3555 <<Дослідження та створення методів опто-- акустичного контролю матеріалів>> (2006--2008рр.).

{\AimAndTasksTXT}
Метою дисертаційної роботи є
встановлення основних закономірностей акусто--індукованих динамічних ефектів у кремнієвих структурах із $p$---$n$--переходом та контактом Шотткі,
вияснення фізичних механізмів впливу опромінення та ультразвукового навантаження на проходження струму в напівпровідникових поверхнево--бар'єрних структурах.
% розробка нових способів модифікації дефектної підсистеми кристалів з використанням ультразвуку.

Для досягнення поставленої мети вирішувалися \textbf{наступні задачі}:
\begin{itemize}[leftmargin=0em,itemindent=1.5em]
\renewcommand{\labelitemi}{$\bullet$}
  \item Підбір бар'єрних структур для досліджень та обрання потрібних режимів опромінення (тип частинок, доза) та ультразвукового навантаження (тип акустичних хвиль, їхня інтенсивність і частота);
%  їх обробки нейтронами, $\gamma$--квантами та мікрохвильовим опроміненнням.

  \item З'ясування механізмів перенесення заряду в широкому температурному діапазоні як у вихідних структурах, так і в радіаційно--модифікованих, визначення характерних параметрів (висота бар'єру, фактор неідеальності, час життя неосновних носіїв заряду тощо);

\item Встановлення закономірностей впливу ультразвукового навантаження на процеси проходження струму та фотоелектричного перетворення у поверхнево-бар'єрних структурах до та після опромінення;

%  , у тому числі і за умов ультразвукового навантаження з використанням акустичних хвиль різного типу, інтенсивності та частоти.

 \item Проведення порівняльного аналізу та оптимізації методів визначення параметрів напівпровідникових бар'єрних структур;
%
%  \item Встановлення механізмів перенесення заряду як у вихідних структурах, так і в радіаційно--модифікованих; визначення характерних параметрів (висота бар'єру, фактор неідеальності, час життя неосновних носіїв заряду тощо).

%  \item З'ясування закономірностей впливу акустичного навантаження на процеси фотоелектричного перетворення в кристалічних кремнієвих сонячних елементах до та після нейтронного опромінення.

  \item Вияснення фізичних механізмів і розробка та обґрунтування фізичних моделей акусто-- та радіаційно--індукованих ефектів;

  \item З'ясування механізмів впливу мікрохвильового опромінення та акустичного навантаження на параметри глибоких рівнів, пов'язаних із порушеннями кристалічної структури,
  визначення природи основних акусто--активних дефектів.

\end{itemize}


{\ObjectTXT} --
процес проходження струму в напівпровідникових структурах.

{\PredmetTXT} --
ефекти впливу ультразвукового навантаження та опромінення на
процеси проходження струму та фотоелектричного перетворення у поверхнево--бар'єрних кремнієвих та арсенід--ґалієвих структурах.


{\MethodTXT}
З метою вирішення поставлених задач використано комплекс експериментальних та розрахункових методів, який включає:
аналіз вольт--амперних і
вольт--фарадних характеристик;
акустоелектричну релаксаційну спектроскопію та метод диференційних коефіцієнтів ВАХ для визначення параметрів глибоких рівнів;
метод стаціонарного струму короткого замикання для визначення довжини дифузії неосновних носіїв;
аналітичні та числові методи визначення параметрів діодів Шотткі;
еволюційні алгоритми мінімізації функцій;
імпульсний метод вимірювання поглинання акустичної хвилі;
резонансний метод вимірювання імпедансу навантаженого акустичного перетворювача;
профілометрію;
рентгенівські дифрактометрію поверхні та топографію;
контрольоване радіаційне та мікрохвильове опромінення для зміни дефектного стану зразків;
метод ультразвукового навантаження.



{\noveltyTXT}
У результаті виконання дисертаційної роботи отримано ряд нових науково--обґрунтованих
результатів, які
сприяють розв'язку актуальної проблеми фізики твердого тіла ---
встановлення механізмів впливу опромінення та акустичного навантаження на процеси перенесення заряду в поверхнево--бар'єрних напівпровідникових структурах.
%мають важливе значення для розуміння процесів процесів перенесення заряду в поверхнево--бар'єрних структурах, у тому числі радіаційно опромінених, за умов акустичного навантаження.
Наукова новизна зумовлена застосуванням нових ультразвукових методів динамічного керування станом дефектів у напівпровідникових структурах, а також вперше проведеними комплексними дослідженнями низки фундаментальних процесів електроперенесення та рекомбінації нерівноважних носіїв заряду в кремнієвих та арсенід--ґалієвих  поверхнево--бар'єрних структурах за умов керованих змін у системі дефектів кристалу як за допомогою опромінення, так і акустичного навантаження.
Зіставлення отриманих експериментальних результатів із даними теоретичного аналізу та окремими результатами інших авторів, дозволили повністю якісно та, у більшості випадків, кількісно описати всі виявлені ефекти.
Досягнутий рівень розуміння деталей проходження струму в поверхнево--бар'єрних структурах дозволяє надійно оцінювати ефективність роботи відповідних  напівпровідникових пристроїв при дії зовнішніх чинників.
\begin{itemize}[leftmargin=0em,itemindent=1.5em]
\renewcommand{\labelitemi}{$\bullet$}
  \item Вперше виявлено оборотні ефекти впливу ультразвукового навантаження на електрофізичні
   властивості кремнієвих структур із $p$---$n$--переходом і контактом метал---напівпровідник
   та встановлено їхні закономірності.
%   ;
%   показано, що застосування ультразвукового навантаження розширює можливості вивчення фундаментальних характеристик і параметрів подібних структур.

  \item Вперше встановлено відмінності впливу акустичного навантаження на параметри неопромінених та радіаційно--опромінених кремнієвих поверхнево--бар'єрних структур,
  які зумовлені різницею складу дефектів;
      вперше визначено природу основних акустоактивних радіаційних дефектів.

  \item Запропоновано фізичну модель акустоактивного комплексного дефекту для пояснення особливостей виявлених акусто--індукованих ефектів.

  \item Проведено порівняльний аналіз аналітичних, числових та еволюційних методів розрахунку параметрів діодів Шотткі з вольт--амперних характеристик та визначено найоптимальніші з них з погляду точності та швидкодії.

  \item Використовуючи модель поглинання ультразвуку Брейсфорда встановлено, що акусто--індукована зміна висоти бар'єру Шотткі у кремнієвих структурах метал---напівпровідник  зумовлена рухом дислокаційних перегинів.

%  \item Вперше показано доцільність застосування моделі поглинання ультразвуку Брейсфорда до пояснення динамічних  акустоіндукованих ефектів в кремнієвих структурах метал--напівпровідник.

  \item Вперше виявлено  взаємозв'язок характеру немонотонності дозової залежності зміни висоти бар'єру Шотткі при $\gamma$--опроміненні зі ступенем неоднорідності контакту та встановлено його фізичні причини.

  \item Встановлено, що  вплив мікрохвильового опромінення на параметри дефектів, розташованих у приповерхневих шарах кристалів GaAs, 6$H$--SiC та на внутрішніх межах арсенід--ґалієвих епітаксійних структур, зумовлений збільшення кількості міжвузлових атомів.

\end{itemize}



{\influenceTXT}
Отримані в роботі результати сприяють глибшому розумінню фізичних процесів у поверхнево--бар'єрних структурах при
знакозмінних механічних навантаженнях та опроміненні нейтронами і високоенергетичними фотонами, що
дозволяє підвищити точність прогнозування реальних робочих характеристик подібних систем залежно від умов функціонування.
Використовуючи результати дослідження частотних, амплітудних та температурних залежностей акусто--індукованих ефектів у бар'єрних структурах,
запропоновано новий метод динамічного акустичного керування струмом напівпровідникових діодів з $p$---$n$--переходом та контактом Шотткі.
%Дослідження частотних, амплітудних та температурних залежностей акусто--індукованих ефектів у бар'єрних структурах дозволяє
%ефективно контролювати процеси перенесення заряду.
Проведене тестування та порівняльне дослідження методів визначення параметрів діодів Шотткі дозволяє вибирати найефективніший з них залежно
від експериментальних умов вимірювання характеристик, типу структур, вимог до швидкодії.
Запропоновано новий метод оптимізації вибору діапазону експериментальних даних для побудови аналітичних функцій, який підвищує точність визначення параметрів структур метал---напівпровідник.
%Запропоновано новий принцип дії сенсорів
Виявлені зміни амплітудної залежності акусто--індукованого зростання зворотного струму  діодів Шотткі
 після дії $\gamma$--квантів
можуть бути використані для створення нових сенсорів опромінення.
%А саме, амплітудна залежність АІ змін зворотного струму діодів Шотткі дозволяє оцінити поглинуту дозу $\gamma$--квантів, тоді
%як величина та знак впливу ультразвуку на фактор неідельності та рекомбінаційний струм кремнієвих $p-n$ структур дозволяють розрізнити
%нейтронно-- та $\gamma$--опромінені структури.




{\contributionTXT}
Внесок автора в отримання наукових результатів полягає у постановці задач
та визначенні методів їхнього вирішення, виборі об'єктів та формулюванні
основних напрямків досліджень,
розробці методології експериментальних досліджень та програмного забезпечення для обробки експериментальних даних.
Переважна більшість експериментальних та теоретичних досліджень виконані автором особисто.
12 із 25 наукових статей, опублікованих за темою дисертації, є одноосібними роботами здобувача.
У наукових працях, опублікованих зі співавторами, автору належить проведення значної частини досліджень, аналіз і узагальнення отриманих
даних, інтерпретація результатів, участь у написанні наукових статей.
Співавторами робіт \cite{Olikh2018JAP,Olikh:Ultras2016,Olikh2016JSem,OlikhJAP,Olikh:PZTF2006} є студенти \thesisOfOrganization,
які виконували кваліфікаційні роботи під керівництвом здобувача.
У роботах \cite{Olikh2018JAP,Olikh:Ultras2016,Olikh2016JSem,OlikhJAP,Olikh:SEMT2007,Olikh:MRS2007a,Olikh:PZTF2006} автором здійснено підбір структур для досліджень, вибір режимів вимірювань та радіаційного опромінення,
проведено основну частину експериментальних вимірювань та аналіз механізмів перенесення заряду і впливу ультразвукових хвиль на ці процеси,
підготовлено тексти статей.
У роботі \cite{Olikh2018JAP} автором запропоновано модель акустоактивного дефектного комплексу,
в роботі \cite{Olikh:Ultras2016} --- встановлено можливість застосування моделі поглинання ультразвука внаслідок руху дислокаційних перегинів до пояснення акусто--індукованих змін параметрів діодів Шотткі.
Внесок здобувача у роботу \cite{Olikh:UPJ2014} визначався проведенням розрахунків у межах моделей дислокаційного поглинання ультразвука.
В роботі \cite{Olikh:UPJ2013} вимірювання вольт--фарадних характеристик  проведені співробітником фізичного факультету, канд. фіз.--мат. наук Надточієм~А.\:Б.
%Пошук та аналіз літературних даних щодо впливу ультразвуку на параметри напівпровідникових кристалів та структур на їх основі, а також їх узагальненняУ
У роботах \cite{Olikh:SEMT2004,Olikh:SEMT2011} автором проводився пошук, аналіз та узагальнення літературних даних щодо впливу ультразвука на параметри напівпровідникових кристалів та структур.
% на їх основі, а також їхнє узагальнення.
Внесок здобувача у роботу \cite{Gorb2010} визначався постановкою дослідів по вимірюванню вольт--амперних характеристик,
інтерпретацією відповідних результатів,
%(саме ця частина представлена у дисертаційній роботі),
участю в написанні статті.
У роботах \cite{Olikh:PhChOM2005,Olikh:PJE2004} автор провів дослідження параметрів глибоких рівнів,
проаналізував отримані дані, взяв участь у написанні статей.
Постановка наукової задачі в цих роботах, а також загальна інтерпретація результатів виконана сумісно з докт. техн. наук Конаковою~Р.\:В.;
рентгенографічні та профілометричні дослідження проводились канд. фіз.--мат. наук Литвином~П.\:М.  (обидва --- Інститут фізики напівпровідників ім. В.\:Є.~Лашкарьова НАНУ).
Основна частина результатів
представлена автором особисто на вітчизняних і міжнародних конференціях
та наукових семінарах.
% кафедри загальної фізики \thesisOfOrganization.






{\probationTXT}
Основні результати, викладені в роботі, доповідалися на наукових семінарах
кафедри загальної фізики \thesisOfOrganization
%Київського національного університету імені Тараса Шевченка
~та були представлені на наступних наукових конференціях:
І, ІІІ, IV, V, VI та VII Українські наукові конференції з фізики напівпровідників
(Одеса, Україна, 2002; Одеса, Україна, 2007; Запоріжжя, Україна, 2009;
Ужгород, Україна, 2011; Чернівці, Україна, 2013; Дніпро, Україна, 2016);
III международная конференция <<Радиационно-термические эффекты и процессы в неорганических материалах>> (Томск, Россия, 2002);
1--ша та 6-та Міжнародні науково-технічна конференції <<Сенсорна електроніка і мікросистемні технології СЕМСТ>> (Одеса, Україна, 2004; 2014);
2004 IEEE International Ultrasonics, Ferroelectrics and Frequency Control Joint 50th Anniversary Conference (Montreal, Canada, 2004);
Девятая международная научно--техническая конференция <<Актуальные проблемы твердотельной электроники и микроэлектроники>> (Дивноморское, Россия, 2004);
2005 та 2014 IEEE International Ultrasonics Symposium (Rotterdam, Netherlands, 2005; Chicago, USA, 2014);
2007 та 2015 International Congress on Ultrasonics (Vienna, Austria, 2007; Metz, France, 2015);
MRS 2007 Spring Meeting, Symposium F: Semiconductor Defect Engineering -- Materials, Synthetic Structures, and Devices II (San Francisco, USA, 2007);
VІ та VІІ Міжнародні школи--конференції <<Актуальні проблеми фізики напівпровідників>> (Дрогобич, Україна, 2008; 2010);
13th International Conference on Defects – Recognition, Imaging and Physics in Semiconductors (Wheeling, USA, 2009);
ХІІ та ХІV Міжнародні конференції <<Фізика і технологія тонких плівок та наносистем>> (Івано--Франківськ, Україна, 2009; Буковель, Україна, 2013);
Четверта міжнародна науково--практична конференція <<Матеріали електронної техніки та сучасні інформаційні технології>> (Кременчук, Україна, 2010);
Всеукраїнська наукова конференція <<Актуальні проблеми теоретичної, експериментальної та прикладної фізики>> (Тернопіль, Україна, 2012);
International research and practice conference <<Nanotechnology and nanomaterials>> (Bukovel, Ukraine, 2013);
IV міжнародна конференція <<Сучасні проблеми фізики конденсованого стану>> (Київ, Україна, 2015);
ІІ Всеукраїнська науково--практична конференція МЕІСS--2017 (Дніпро, Україна, 2017).

{\publicationsTXT}
%За отриманими результатами опубліковано 25 наукових праць,
%з них 25 статті у фахових журналах.
За результатами дослідження опубліковано 54 наукові праці:
%загальним обсягом 13,9 д. а. (з них 10,4 д. а. належать особисто автору)
 25 наукових статей
 %(11,2 д. а., з них 8,2 д. а. – авторські), з яких 25 –
 у фахових журналах (17 статей у виданнях, які входять до наукометричної бази даних Scopus); 29 тез доповідей на наукових конференціях.


%{\structureTXT}
%Дисертація складається із вступу, шести розділів, загальних висновків та списку використаних джерел.
%Загальних обсяг дисертації складає
%%% на случай ошибок оставляю исходный кусок на месте, закомментированным
%%\ref*{TotPages}~сторінки з~\totalfigures{}~рисунками та~\totaltables{}~таблицями.
%%Список використаних джерел містить \total{citenum}~найменувань.
%%
%\formbytotal{TotPages}{сторінк}{у}{и}{ок}, включаючи
%\formbytotal{totalcount@figure}{рисун}{ок}{ки}{ків} та
%\formbytotal{totalcount@table}{таблиц}{ю}{і}{ь}.
%%Список використаних джерел містить
%%\formbytotal{citenum}{найменуван}{ня}{ь}{ь}.



%
%%\publications\ Основные результаты по теме диссертации изложены в ХХ печатных изданиях~\cite{Sokolov,Gaidaenko,Lermontov,Management},
%%Х из которых изданы в журналах, рекомендованных ВАК~\cite{Sokolov,Gaidaenko},
%%ХХ --- в тезисах докладов~\cite{Lermontov,Management}.
%
%\ifnumequal{\value{bibliosel}}{0}{% Встроенная реализация с загрузкой файла через движок bibtex8
%    \publications\ Основные результаты по теме диссертации изложены в XX печатных изданиях,
%    X из которых изданы в журналах, рекомендованных ВАК,
%    X "--- в тезисах докладов.%
%}{% Реализация пакетом biblatex через движок biber
%%Сделана отдельная секция, чтобы не отображались в списке цитированных материалов
%    \begin{refsection}[vak,papers,conf]% Подсчет и нумерация авторских работ. Засчитываются только те, которые были прописаны внутри \nocite{}.
%        %Чтобы сменить порядок разделов в сгрупированном списке литературы необходимо перетасовать следующие три строчки, а также команды в разделе \newcommand*{\insertbiblioauthorgrouped} в файле biblio/biblatex.tex
%        \printbibliography[heading=countauthorvak, env=countauthorvak, keyword=biblioauthorvak, section=1]%
%        \printbibliography[heading=countauthorconf, env=countauthorconf, keyword=biblioauthorconf, section=1]%
%        \printbibliography[heading=countauthornotvak, env=countauthornotvak, keyword=biblioauthornotvak, section=1]%
%        \printbibliography[heading=countauthor, env=countauthor, keyword=biblioauthor, section=1]%
%        \nocite{%Порядок перечисления в этом блоке определяет порядок вывода в списке публикаций автора
%                vakbib1,vakbib2,%
%                confbib1,confbib2,%
%                bib1,bib2,%
%        }%
%        \publications\ Основные результаты по теме диссертации изложены в~\arabic{citeauthor}~печатных изданиях,
%        \arabic{citeauthorvak} из которых изданы в журналах, рекомендованных ВАК,
%        \arabic{citeauthorconf} "--- в~тезисах докладов.
%    \end{refsection}
%    \begin{refsection}[vak,papers,conf]%Блок, позволяющий отобрать из всех работ автора наиболее значимые, и только их вывести в автореферате, но считать в блоке выше общее число работ
%        \printbibliography[heading=countauthorvak, env=countauthorvak, keyword=biblioauthorvak, section=2]%
%        \printbibliography[heading=countauthornotvak, env=countauthornotvak, keyword=biblioauthornotvak, section=2]%
%        \printbibliography[heading=countauthorconf, env=countauthorconf, keyword=biblioauthorconf, section=2]%
%        \printbibliography[heading=countauthor, env=countauthor, keyword=biblioauthor, section=2]%
%        \nocite{vakbib2}%vak
%        \nocite{bib1}%notvak
%        \nocite{confbib1}%conf
%    \end{refsection}
%}
%При использовании пакета \verb!biblatex! для автоматического подсчёта
%количества публикаций автора по теме диссертации, необходимо
%их~здесь перечислить с использованием команды \verb!\nocite!.


{\structureTXT}
Дисертація складається із вступу, шести розділів, загальних висновків та списку використаних джерел.
Загальних обсяг дисертації складає
%% на случай ошибок оставляю исходный кусок на месте, закомментированным
%\ref*{TotPages}~сторінки з~\totalfigures{}~рисунками та~\totaltables{}~таблицями.
%Список використаних джерел містить \total{citenum}~найменувань.
%
\formbytotal{TotPages}{сторінк}{у}{и}{ок}, включаючи
\formbytotal{totalcount@figure}{рисун}{ок}{ки}{ків} та
\formbytotal{totalcount@table}{таблиц}{ю}{і}{ь}.
%Список використаних джерел містить
%\formbytotal{citenum}{найменуван}{ня}{ь}{ь}.


%
%За наявності у вступі можуть також вказуватися:
%
%зв’язок роботи з науковими програмами, планами, темами, грантами - вказується, в рамках яких програм, тематичних планів, наукових тематик і грантів, зокрема галузевих, державних та/або міжнародних, виконувалося дисертаційне дослідження, із зазначенням номерів державної реєстрації науково-дослідних робіт і найменуванням організації, де виконувалася робота;
%практичне значення отриманих результатів - надаються відомості про використання результатів досліджень або рекомендації щодо їх практичного використання.

%
%{\actuality} Обзор, введение в тему, обозначение места данной работы в
%мировых исследованиях и~т.\:п., можно использовать ссылки на~другие
%работы\ifnumequal{\value{bibliosel}}{1}{~\autocite{Gosele1999161}}{}
%(если их~нет, то~в~автореферате
%автоматически пропадёт раздел <<Список литературы>>). Внимание! Ссылки
%на~другие работы в разделе общей характеристики работы можно
%использовать только при использовании \verb!biblatex! (из-за технических
%ограничений \verb!bibtex8!. Это связано с тем, что одна
%и~та~же~характеристика используются и~в~тексте диссертации, и в
%автореферате. В~последнем, согласно ГОСТ, должен присутствовать список
%работ автора по~теме диссертации, а~\verb!bibtex8! не~умеет выводить в одном
%файле два списка литературы).
%При использовании \verb!biblatex! возможно использование исключительно
%в~автореферате подстрочных ссылок
%для других работ командой \verb!\autocite!, а~также цитирование
%собственных работ командой \verb!\cite!. Для этого в~файле
%\verb!Synopsis/setup.tex! необходимо присвоить положительное значение
%счётчику \verb!\setcounter{usefootcite}{1}!.
%
%Для генерации содержимого титульного листа автореферата, диссертации
%и~презентации используются данные из файла \verb!common/data.tex!. Если,
%например, вы меняете название диссертации, то оно автоматически
%появится в~итоговых файлах после очередного запуска \LaTeX. Согласно
%ГОСТ 7.0.11-2011 <<5.1.1 Титульный лист является первой страницей
%диссертации, служит источником информации, необходимой для обработки и
%поиска документа>>. Наличие логотипа организации на титульном листе
%упрощает обработку и поиск, для этого разметите логотип вашей
%организации в папке images в формате PDF (лучше найти его в векторном
%варианте, чтобы он хорошо смотрелся при печати) под именем
%\verb!logo.pdf!. Настроить размер изображения с логотипом можно
%в~соответствующих местах файлов \verb!title.tex!  отдельно для
%диссертации и автореферата. Если вам логотип не~нужен, то просто
%удалите файл с логотипом.
%
%\ifsynopsis
%Этот абзац появляется только в~автореферате.
%Для формирования блоков, которые будут обрабатываться только в~автореферате,
%заведена проверка условия \verb!\!\verb!ifsynopsis!.
%Значение условия задаётся в~основном файле документа (\verb!synopsis.tex! для
%автореферата).
%\else
%Этот абзац появляется только в~диссертации.
%Через проверку условия \verb!\!\verb!ifsynopsis!, задаваемого в~основном файле
%документа (\verb!dissertation.tex! для диссертации), можно сделать новую
%команду, обеспечивающую появление цитаты в~диссертации, но~не~в~автореферате.
%\fi
%
%% {\progress}
%% Этот раздел должен быть отдельным структурным элементом по
%% ГОСТ, но он, как правило, включается в описание актуальности
%% темы. Нужен он отдельным структурынм элемементом или нет ---
%% смотрите другие диссертации вашего совета, скорее всего не нужен.
%
%{\aim} данной работы является \ldots
%
%Для~достижения поставленной цели необходимо было решить следующие {\tasks}:
%\begin{enumerate}
%  \item Исследовать, разработать, вычислить и~т.\:д. и~т.\:п.
%  \item Исследовать, разработать, вычислить и~т.\:д. и~т.\:п.
%  \item Исследовать, разработать, вычислить и~т.\:д. и~т.\:п.
%  \item Исследовать, разработать, вычислить и~т.\:д. и~т.\:п.
%\end{enumerate}
%
%
%{\novelty}
%\begin{enumerate}
%  \item Впервые \ldots
%  \item Впервые \ldots
%  \item Было выполнено оригинальное исследование \ldots
%\end{enumerate}
%
%{\influence} \ldots
%
%{\methods} \ldots
%
%{\defpositions}
%\begin{enumerate}
%  \item Первое положение
%  \item Второе положение
%  \item Третье положение
%  \item Четвертое положение
%\end{enumerate}
%В папке Documents можно ознакомиться в решением совета из Томского ГУ
%в~файле \verb+Def_positions.pdf+, где обоснованно даются рекомендации
%по~формулировкам защищаемых положений.
%
%{\reliability} полученных результатов обеспечивается \ldots \ Результаты находятся в соответствии с результатами, полученными другими авторами.
%
%

{\actualityTXT}
Напівпровідникові поверхнево--бар'єрні структури --- основа мікроелектроніки та сонячної енергетики, галузей, розвиток яких на сучасному етапі визначає загальний прогрес людства.
Незважаючи на все різноманіття наявних типів фотовольтаїчних перетворювачів, на ринку промислового використання переважають моно-- та полікристалічні кремнієві сонячні елементи.
Загалом, серед всіх напівпровідникових систем кремнієві структури використовують найширше.
Це зумовлено величезними запасами даного елементу, його нетоксичністю та високою технологічністю як вирощування самих кристалів, так і створення різноманітних структур.
Зокрема кремнієві структури з контактом Шотткі  застосовують при виготовленні високошвидкісних логічних та інтегральних елементів.
У цьому ж сегменті високочастотних мікроелектронних пристроїв широко представлені системи на основі арсеніду ґалію --- матеріалу, який характеризується високою рухливістю носіїв заряду.
У дисертаційній роботі наводяться результати дослідження кремнієвих сонячних елементів та структур метал---напівпровідник на основі Si та GaAs, що визначає її актуальність з прикладної точки зору.

Загальною задачею матеріалознавства є створення матеріалів та структур із заданими властивостями.
Для її реалізації необхідне чітке розуміння фізичних процесів, які відбуваються в матеріалах за різних умов.
Зокрема, умови функціонування напівпровідникових приладів нерідко передбачають наявність різноманітного радіаційного впливу.
Вивченню радіаційно--індукованих процесів у напівпровідниках присвячена значна кількість робіт, що також свідчить про актуальність подібних досліджень,
проте окремі аспекти, наприклад немонотонність зміни характеристик діодів Шотткі при дії $\gamma$--квантів чи механізми модифікації приповерхневого шару при мікрохвильовому опроміненні, досі залишалися поза увагою.
У виконаній роботі показано взаємозв'язок між ступенем неоднорідності контакту Шотткі та характером дозової немонотонності зміни висоти бар'єру, а також з'ясовано, що перетворення у дефектній структурі приповерхневого шару зумовлені збільшенням концентрації міжвузлових атомів.
Іншим зовнішнім чинником, який може впливати на параметри напівпровідникових структур, є знакозмінні пружні деформації, зумовлені, наприклад, поширенням акустичних хвиль.
У роботі вперше проведено дослідження перенесення заряду в кремнієвих бар'єрних структурах за умов ультразвукового навантаження.
Перераховані напрямки проведених досліджень свідчать про актуальність виконаної роботи в області матеріалознавства.

%Загальною задачею напівпровідникового матерiалознавства є створення матерiалiв та структур iз заданими властивостями. Фундаментальні дослідження дефектів у напівпровідниках, поглиблення розуміння їхньої поведінки мають важливе значення для розширення функціональних можливостей пристроїв та ідентифікації і усунення небажаних дефектів. Сфера інтересів фізики дефектів, як це відзначається в матеріалах останнього світового форуму по дефектах у напівпровідниках (Defects in Semiconductors Journal of Applied Physics 123, 161301 (2018); https://doi.org/10.1063/1.5036660)  зараз також охоплює і методи  зовнішньої активації технологічно функціональних дефектів та домішок для управління електричними, оптичними, тепловими та магнітними властивостями напівпровідників, що дозволяє інженерам додавати нові функції до напівпровідникових пристроїв. Одним з таких чинників є знакозмiннi високочастотнi пружнi деформацiї, зумовленi, наприклад, поширенням акустичних хвиль


Поглиблення розуміння поведінки дефектів у напівпровідниках мають фундаментальне значення для розширення можливостей відповідних пристроїв.
Як видно з матеріалів останніх світових форумів, сфера інтересів фізики дефектів охоплює методи  зовнішньої активації технологічно функціональних дефектів для управління властивостями напівпровідників.
Загальновизнаними подібними способами є опромінення та термообробка, які, проте, суттєво впливають і на стан кристала загалом.
%
%Вирішення матеріалознавчих задач потребує розробки методів керування параметрами матеріалів та структур.
%Відомо, що дефекти структури є визначальними для фізичних властивостей кристалів і мають фундаментальне значення у фізиці твердого тіла.
%Найпоширенішими способами впливу на дефектну підсистему напівпровідників є опромінення швидкими частинками та термообробка, які суттєво впливають на стан кристала в цілому.
Представлені результати свідчать про здатність ультразвукового навантаження навіть допорогової інтенсивності модифікувати дефекти у кремнієвих кристалах, причому до переваг такого підходу варто віднести вибірковість впливу саме на області з порушеннями періодичності та оборотність змін при кімнатних температурах.
Тому виконана робота актуальна з погляду розробки нових методів керування параметрами бар'єрних структур.

Вважається, що причинами змін стану точкових дефектів у напівпровідникових кристалах під дією акустичних хвиль є вимушені коливання дислокацій, акусто--стимульована дифузія домішок та генерація дефектів при надпороговій інтенсивності пружних коливань.
Проте для оборотних акусто--індукованих ефектів у малодислокаційних матеріалах
%при допороговій інтенсивності ультразвука
подібні механізми  не є визначальними.
Проведене дослідження особливостей перенесення заряду при ультразвуковому навантаженні кремнієвих структур та ідентифікація <<акустоактивних>>, тобто здатних до ефективної взаємодії з пружними коливаннями, дефектів технологічного та радіаційного походження
%, у тому числі і радіаційних,
є актуальним з погляду встановлення фізичних причин акусто--дефектної взаємодії у подібних матеріалах.

Відтак,  дослідження фізичних  закономірностей  та встановлення механізмів акусто-- та радіаційно--індукованих
ефектів у поверхнево--бар'єрних напівпровідникових структурах  є  важливим  для  вирішення  перелічених  проблем  і визначає наукову та практичну актуальність дисертаційної роботи.


{\InterconnectionTXT}
Дисертаційна робота  пов’язана із планами науково--дослідних робіт, які проводились у рамках
держбюджетних тем та міжнародних проектів на кафедрі загальної фізики фізичного факультету \thesisOfOrganization.
А саме:
\textnumero01БФ051--09 <<Теоретичне та експериментальне дослідження фізичних властивостей неоднорідних систем на основі матеріалів акусто--опто--електроніки та мікроелектроніки>>
(\textnumero~держ. реєстрації 01БФ051--09, 2001--2005рр.);
\textnumero06БФ051--04 <<Експериментальне та теоретичне дослідження структури та фізичних властивостей низькорозмірних систем на основі напівпровідникових структур, різних модифікацій вуглецю та композитів>>
(\textnumero~держ. реєстрації 0106U006390, 2006--2010рр.);
\textnumero11БФ051--01 <<Фундаментальні дослідження в галузі фізики конденсованого стану і елементарних частинок, астрономії і матеріалознавства для створення основ новітніх технологій>>
(\textnumero~держ. реєстрації 0111U004954, 2011--2015рр.);
\textnumero16БФ051--01 <<Формування та фізичні властивості наноструктурованих композитних матеріалів та функціональних поверхневих шарів на основі карбону, напівпровідникових та діелектричних складових>>
(\textnumero~держ. реєстрації  0116U004781, 2016--2018рр.) та
проект УНТЦ №3555 <<Дослідження та створення методів опто-- акустичного контролю матеріалів>> (2006--2008рр.).

{\AimAndTasksTXT}
Метою дисертаційної роботи є
встановлення основних закономірностей акусто--індукованих динамічних ефектів у кремнієвих структурах із $p$---$n$--переходом та контактом Шотткі,
вияснення фізичних механізмів впливу опромінення та ультразвукового навантаження на проходження струму в напівпровідникових поверхнево--бар'єрних структурах.
% розробка нових способів модифікації дефектної підсистеми кристалів з використанням ультразвуку.

Для досягнення поставленої мети вирішувалися \textbf{наступні задачі}:
\begin{itemize}[leftmargin=0em,itemindent=1.5em]
\renewcommand{\labelitemi}{$\bullet$}
  \item Підбір бар'єрних структур для досліджень та обрання потрібних режимів опромінення (тип частинок, доза) та ультразвукового навантаження (тип акустичних хвиль, їхня інтенсивність і частота);
%  їх обробки нейтронами, $\gamma$--квантами та мікрохвильовим опроміненнням.

  \item З'ясування механізмів перенесення заряду в широкому температурному діапазоні як у вихідних структурах, так і в радіаційно--модифікованих, визначення характерних параметрів (висота бар'єру, фактор неідеальності, час життя неосновних носіїв заряду тощо);

\item Встановлення закономірностей впливу ультразвукового навантаження на процеси проходження струму та фотоелектричного перетворення у поверхнево-бар'єрних структурах до та після опромінення;

%  , у тому числі і за умов ультразвукового навантаження з використанням акустичних хвиль різного типу, інтенсивності та частоти.

 \item Проведення порівняльного аналізу та оптимізації методів визначення параметрів напівпровідникових бар'єрних структур;
%
%  \item Встановлення механізмів перенесення заряду як у вихідних структурах, так і в радіаційно--модифікованих; визначення характерних параметрів (висота бар'єру, фактор неідеальності, час життя неосновних носіїв заряду тощо).

%  \item З'ясування закономірностей впливу акустичного навантаження на процеси фотоелектричного перетворення в кристалічних кремнієвих сонячних елементах до та після нейтронного опромінення.

  \item Вияснення фізичних механізмів і розробка та обґрунтування фізичних моделей акусто-- та радіаційно--індукованих ефектів;

  \item З'ясування механізмів впливу мікрохвильового опромінення та акустичного навантаження на параметри глибоких рівнів, пов'язаних із порушеннями кристалічної структури,
  визначення природи основних акусто--активних дефектів.

\end{itemize}


{\ObjectTXT} --
процес проходження струму в напівпровідникових структурах.

{\PredmetTXT} --
ефекти впливу ультразвукового навантаження та опромінення на
процеси проходження струму та фотоелектричного перетворення у поверхнево--бар'єрних кремнієвих та арсенід--ґалієвих структурах.


{\MethodTXT}
З метою вирішення поставлених задач використано комплекс експериментальних та розрахункових методів, який включає:
аналіз вольт--амперних і
вольт--фарадних характеристик;
акустоелектричну релаксаційну спектроскопію та метод диференційних коефіцієнтів ВАХ для визначення параметрів глибоких рівнів;
метод стаціонарного струму короткого замикання для визначення довжини дифузії неосновних носіїв;
аналітичні та числові методи визначення параметрів діодів Шотткі;
еволюційні алгоритми мінімізації функцій;
імпульсний метод вимірювання поглинання акустичної хвилі;
резонансний метод вимірювання імпедансу навантаженого акустичного перетворювача;
профілометрію;
рентгенівські дифрактометрію поверхні та топографію;
контрольоване радіаційне та мікрохвильове опромінення для зміни дефектного стану зразків;
метод ультразвукового навантаження.



{\noveltyTXT}
У результаті виконання дисертаційної роботи отримано ряд нових науково--обґрунтованих
результатів, які
сприяють розв'язку актуальної проблеми фізики твердого тіла ---
встановлення механізмів впливу опромінення та акустичного навантаження на процеси перенесення заряду в поверхнево--бар'єрних напівпровідникових структурах.
%мають важливе значення для розуміння процесів процесів перенесення заряду в поверхнево--бар'єрних структурах, у тому числі радіаційно опромінених, за умов акустичного навантаження.
Наукова новизна зумовлена застосуванням нових ультразвукових методів динамічного керування станом дефектів у напівпровідникових структурах, а також вперше проведеними комплексними дослідженнями низки фундаментальних процесів електроперенесення та рекомбінації нерівноважних носіїв заряду в кремнієвих та арсенід--ґалієвих  поверхнево--бар'єрних структурах за умов керованих змін у системі дефектів кристалу як за допомогою опромінення, так і акустичного навантаження.
Зіставлення отриманих експериментальних результатів із даними теоретичного аналізу та окремими результатами інших авторів, дозволили повністю якісно та, у більшості випадків, кількісно описати всі виявлені ефекти.
Досягнутий рівень розуміння деталей проходження струму в поверхнево--бар'єрних структурах дозволяє надійно оцінювати ефективність роботи відповідних  напівпровідникових пристроїв при дії зовнішніх чинників.
\begin{itemize}[leftmargin=0em,itemindent=1.5em]
\renewcommand{\labelitemi}{$\bullet$}
  \item Вперше виявлено оборотні ефекти впливу ультразвукового навантаження на електрофізичні
   властивості кремнієвих структур із $p$---$n$--переходом і контактом метал---напівпровідник
   та встановлено їхні закономірності.
%   ;
%   показано, що застосування ультразвукового навантаження розширює можливості вивчення фундаментальних характеристик і параметрів подібних структур.

  \item Вперше встановлено відмінності впливу акустичного навантаження на параметри неопромінених та радіаційно--опромінених кремнієвих поверхнево--бар'єрних структур,
  які зумовлені різницею складу дефектів;
      вперше визначено природу основних акустоактивних радіаційних дефектів.

  \item Запропоновано фізичну модель акустоактивного комплексного дефекту для пояснення особливостей виявлених акусто--індукованих ефектів.

  \item Проведено порівняльний аналіз аналітичних, числових та еволюційних методів розрахунку параметрів діодів Шотткі з вольт--амперних характеристик та визначено найоптимальніші з них з погляду точності та швидкодії.

  \item Використовуючи модель поглинання ультразвуку Брейсфорда встановлено, що акусто--індукована зміна висоти бар'єру Шотткі у кремнієвих структурах метал---напівпровідник  зумовлена рухом дислокаційних перегинів.

%  \item Вперше показано доцільність застосування моделі поглинання ультразвуку Брейсфорда до пояснення динамічних  акустоіндукованих ефектів в кремнієвих структурах метал--напівпровідник.

  \item Вперше виявлено  взаємозв'язок характеру немонотонності дозової залежності зміни висоти бар'єру Шотткі при $\gamma$--опроміненні зі ступенем неоднорідності контакту та встановлено його фізичні причини.

  \item Встановлено, що  вплив мікрохвильового опромінення на параметри дефектів, розташованих у приповерхневих шарах кристалів GaAs, 6$H$--SiC та на внутрішніх межах арсенід--ґалієвих епітаксійних структур, зумовлений збільшення кількості міжвузлових атомів.

\end{itemize}



{\influenceTXT}
Отримані в роботі результати сприяють глибшому розумінню фізичних процесів у поверхнево--бар'єрних структурах при
знакозмінних механічних навантаженнях та опроміненні нейтронами і високоенергетичними фотонами, що
дозволяє підвищити точність прогнозування реальних робочих характеристик подібних систем залежно від умов функціонування.
Використовуючи результати дослідження частотних, амплітудних та температурних залежностей акусто--індукованих ефектів у бар'єрних структурах,
запропоновано новий метод динамічного акустичного керування струмом напівпровідникових діодів з $p$---$n$--переходом та контактом Шотткі.
%Дослідження частотних, амплітудних та температурних залежностей акусто--індукованих ефектів у бар'єрних структурах дозволяє
%ефективно контролювати процеси перенесення заряду.
Проведене тестування та порівняльне дослідження методів визначення параметрів діодів Шотткі дозволяє вибирати найефективніший з них залежно
від експериментальних умов вимірювання характеристик, типу структур, вимог до швидкодії.
Запропоновано новий метод оптимізації вибору діапазону експериментальних даних для побудови аналітичних функцій, який підвищує точність визначення параметрів структур метал---напівпровідник.
%Запропоновано новий принцип дії сенсорів
Виявлені зміни амплітудної залежності акусто--індукованого зростання зворотного струму  діодів Шотткі
 після дії $\gamma$--квантів
можуть бути використані для створення нових сенсорів опромінення.
%А саме, амплітудна залежність АІ змін зворотного струму діодів Шотткі дозволяє оцінити поглинуту дозу $\gamma$--квантів, тоді
%як величина та знак впливу ультразвуку на фактор неідельності та рекомбінаційний струм кремнієвих $p-n$ структур дозволяють розрізнити
%нейтронно-- та $\gamma$--опромінені структури.




{\contributionTXT}
Внесок автора в отримання наукових результатів полягає у постановці задач
та визначенні методів їхнього вирішення, виборі об'єктів та формулюванні
основних напрямків досліджень,
розробці методології експериментальних досліджень та програмного забезпечення для обробки експериментальних даних.
Переважна більшість експериментальних та теоретичних досліджень виконані автором особисто.
12 із 25 наукових статей, опублікованих за темою дисертації, є одноосібними роботами здобувача.
У наукових працях, опублікованих зі співавторами, автору належить проведення значної частини досліджень, аналіз і узагальнення отриманих
даних, інтерпретація результатів, участь у написанні наукових статей.
Співавторами робіт \cite{Olikh2018JAP,Olikh:Ultras2016,Olikh2016JSem,OlikhJAP,Olikh:PZTF2006} є студенти \thesisOfOrganization,
які виконували кваліфікаційні роботи під керівництвом здобувача.
У роботах \cite{Olikh2018JAP,Olikh:Ultras2016,Olikh2016JSem,OlikhJAP,Olikh:SEMT2007,Olikh:MRS2007a,Olikh:PZTF2006} автором здійснено підбір структур для досліджень, вибір режимів вимірювань та радіаційного опромінення,
проведено основну частину експериментальних вимірювань та аналіз механізмів перенесення заряду і впливу ультразвукових хвиль на ці процеси,
підготовлено тексти статей.
У роботі \cite{Olikh2018JAP} автором запропоновано модель акустоактивного дефектного комплексу,
в роботі \cite{Olikh:Ultras2016} --- встановлено можливість застосування моделі поглинання ультразвука внаслідок руху дислокаційних перегинів до пояснення акусто--індукованих змін параметрів діодів Шотткі.
Внесок здобувача у роботу \cite{Olikh:UPJ2014} визначався проведенням розрахунків у межах моделей дислокаційного поглинання ультразвука.
В роботі \cite{Olikh:UPJ2013} вимірювання вольт--фарадних характеристик  проведені співробітником фізичного факультету, канд. фіз.--мат. наук Надточієм~А.\:Б.
%Пошук та аналіз літературних даних щодо впливу ультразвуку на параметри напівпровідникових кристалів та структур на їх основі, а також їх узагальненняУ
У роботах \cite{Olikh:SEMT2004,Olikh:SEMT2011} автором проводився пошук, аналіз та узагальнення літературних даних щодо впливу ультразвука на параметри напівпровідникових кристалів та структур.
% на їх основі, а також їхнє узагальнення.
Внесок здобувача у роботу \cite{Gorb2010} визначався постановкою дослідів по вимірюванню вольт--амперних характеристик,
інтерпретацією відповідних результатів,
%(саме ця частина представлена у дисертаційній роботі),
участю в написанні статті.
У роботах \cite{Olikh:PhChOM2005,Olikh:PJE2004} автор провів дослідження параметрів глибоких рівнів,
проаналізував отримані дані, взяв участь у написанні статей.
Постановка наукової задачі в цих роботах, а також загальна інтерпретація результатів виконана сумісно з докт. техн. наук Конаковою~Р.\:В.;
рентгенографічні та профілометричні дослідження проводились канд. фіз.--мат. наук Литвином~П.\:М.  (обидва --- Інститут фізики напівпровідників ім. В.\:Є.~Лашкарьова НАНУ).
Основна частина результатів
представлена автором особисто на вітчизняних і міжнародних конференціях
та наукових семінарах.
% кафедри загальної фізики \thesisOfOrganization.






{\probationTXT}
Основні результати, викладені в роботі, доповідалися на наукових семінарах
кафедри загальної фізики \thesisOfOrganization
%Київського національного університету імені Тараса Шевченка
~та були представлені на наступних наукових конференціях:
І, ІІІ, IV, V, VI та VII Українські наукові конференції з фізики напівпровідників
(Одеса, Україна, 2002; Одеса, Україна, 2007; Запоріжжя, Україна, 2009;
Ужгород, Україна, 2011; Чернівці, Україна, 2013; Дніпро, Україна, 2016);
III международная конференция <<Радиационно-термические эффекты и процессы в неорганических материалах>> (Томск, Россия, 2002);
1--ша та 6-та Міжнародні науково-технічна конференції <<Сенсорна електроніка і мікросистемні технології СЕМСТ>> (Одеса, Україна, 2004; 2014);
2004 IEEE International Ultrasonics, Ferroelectrics and Frequency Control Joint 50th Anniversary Conference (Montreal, Canada, 2004);
Девятая международная научно--техническая конференция <<Актуальные проблемы твердотельной электроники и микроэлектроники>> (Дивноморское, Россия, 2004);
2005 та 2014 IEEE International Ultrasonics Symposium (Rotterdam, Netherlands, 2005; Chicago, USA, 2014);
2007 та 2015 International Congress on Ultrasonics (Vienna, Austria, 2007; Metz, France, 2015);
MRS 2007 Spring Meeting, Symposium F: Semiconductor Defect Engineering -- Materials, Synthetic Structures, and Devices II (San Francisco, USA, 2007);
VІ та VІІ Міжнародні школи--конференції <<Актуальні проблеми фізики напівпровідників>> (Дрогобич, Україна, 2008; 2010);
13th International Conference on Defects – Recognition, Imaging and Physics in Semiconductors (Wheeling, USA, 2009);
ХІІ та ХІV Міжнародні конференції <<Фізика і технологія тонких плівок та наносистем>> (Івано--Франківськ, Україна, 2009; Буковель, Україна, 2013);
Четверта міжнародна науково--практична конференція <<Матеріали електронної техніки та сучасні інформаційні технології>> (Кременчук, Україна, 2010);
Всеукраїнська наукова конференція <<Актуальні проблеми теоретичної, експериментальної та прикладної фізики>> (Тернопіль, Україна, 2012);
International research and practice conference <<Nanotechnology and nanomaterials>> (Bukovel, Ukraine, 2013);
IV міжнародна конференція <<Сучасні проблеми фізики конденсованого стану>> (Київ, Україна, 2015);
ІІ Всеукраїнська науково--практична конференція МЕІСS--2017 (Дніпро, Україна, 2017).

{\publicationsTXT}
%За отриманими результатами опубліковано 25 наукових праць,
%з них 25 статті у фахових журналах.
За результатами дослідження опубліковано 54 наукові праці:
%загальним обсягом 13,9 д. а. (з них 10,4 д. а. належать особисто автору)
 25 наукових статей
 %(11,2 д. а., з них 8,2 д. а. – авторські), з яких 25 –
 у фахових журналах (17 статей у виданнях, які входять до наукометричної бази даних Scopus); 29 тез доповідей на наукових конференціях.


%{\structureTXT}
%Дисертація складається із вступу, шести розділів, загальних висновків та списку використаних джерел.
%Загальних обсяг дисертації складає
%%% на случай ошибок оставляю исходный кусок на месте, закомментированным
%%\ref*{TotPages}~сторінки з~\totalfigures{}~рисунками та~\totaltables{}~таблицями.
%%Список використаних джерел містить \total{citenum}~найменувань.
%%
%\formbytotal{TotPages}{сторінк}{у}{и}{ок}, включаючи
%\formbytotal{totalcount@figure}{рисун}{ок}{ки}{ків} та
%\formbytotal{totalcount@table}{таблиц}{ю}{і}{ь}.
%%Список використаних джерел містить
%%\formbytotal{citenum}{найменуван}{ня}{ь}{ь}.



%
%%\publications\ Основные результаты по теме диссертации изложены в ХХ печатных изданиях~\cite{Sokolov,Gaidaenko,Lermontov,Management},
%%Х из которых изданы в журналах, рекомендованных ВАК~\cite{Sokolov,Gaidaenko},
%%ХХ --- в тезисах докладов~\cite{Lermontov,Management}.
%
%\ifnumequal{\value{bibliosel}}{0}{% Встроенная реализация с загрузкой файла через движок bibtex8
%    \publications\ Основные результаты по теме диссертации изложены в XX печатных изданиях,
%    X из которых изданы в журналах, рекомендованных ВАК,
%    X "--- в тезисах докладов.%
%}{% Реализация пакетом biblatex через движок biber
%%Сделана отдельная секция, чтобы не отображались в списке цитированных материалов
%    \begin{refsection}[vak,papers,conf]% Подсчет и нумерация авторских работ. Засчитываются только те, которые были прописаны внутри \nocite{}.
%        %Чтобы сменить порядок разделов в сгрупированном списке литературы необходимо перетасовать следующие три строчки, а также команды в разделе \newcommand*{\insertbiblioauthorgrouped} в файле biblio/biblatex.tex
%        \printbibliography[heading=countauthorvak, env=countauthorvak, keyword=biblioauthorvak, section=1]%
%        \printbibliography[heading=countauthorconf, env=countauthorconf, keyword=biblioauthorconf, section=1]%
%        \printbibliography[heading=countauthornotvak, env=countauthornotvak, keyword=biblioauthornotvak, section=1]%
%        \printbibliography[heading=countauthor, env=countauthor, keyword=biblioauthor, section=1]%
%        \nocite{%Порядок перечисления в этом блоке определяет порядок вывода в списке публикаций автора
%                vakbib1,vakbib2,%
%                confbib1,confbib2,%
%                bib1,bib2,%
%        }%
%        \publications\ Основные результаты по теме диссертации изложены в~\arabic{citeauthor}~печатных изданиях,
%        \arabic{citeauthorvak} из которых изданы в журналах, рекомендованных ВАК,
%        \arabic{citeauthorconf} "--- в~тезисах докладов.
%    \end{refsection}
%    \begin{refsection}[vak,papers,conf]%Блок, позволяющий отобрать из всех работ автора наиболее значимые, и только их вывести в автореферате, но считать в блоке выше общее число работ
%        \printbibliography[heading=countauthorvak, env=countauthorvak, keyword=biblioauthorvak, section=2]%
%        \printbibliography[heading=countauthornotvak, env=countauthornotvak, keyword=biblioauthornotvak, section=2]%
%        \printbibliography[heading=countauthorconf, env=countauthorconf, keyword=biblioauthorconf, section=2]%
%        \printbibliography[heading=countauthor, env=countauthor, keyword=biblioauthor, section=2]%
%        \nocite{vakbib2}%vak
%        \nocite{bib1}%notvak
%        \nocite{confbib1}%conf
%    \end{refsection}
%}
%При использовании пакета \verb!biblatex! для автоматического подсчёта
%количества публикаций автора по теме диссертации, необходимо
%их~здесь перечислить с использованием команды \verb!\nocite!.
 % Характеристика работы по структуре во введении и в автореферате не отличается (ГОСТ Р 7.0.11, пункты 5.3.1 и 9.2.1), потому её загружаем из одного и того же внешнего файла, предварительно задав форму выделения некоторым параметрам

%\cite{Olikh2018JAP,Olikh2018SM,Olikh:Ultras2016,Olikh2016JSem,
%Olikh:Rev,OlikhJAP,Olikh:Ultras,Olikh:UPJ2014,
%Olikh:2013IEEE,Olikh:SEMT2013,Olikh:FTP2013,Olikh:UPJ2013,
%Olikh:FTP2011,Olikh:SEMT2011,Olikh:UPJ2010,Gorb2010,Olikh:FTP2009,
%Olikh:SEMT2007,Olikh:MRS2007,Olikh:PZTF2006,
%Olikh:PhChOM2005,Olikh:PJE2004,Olikh:SEMT2004,Olikh:SPQEO2003,
%Olikh:Visn2003,
%1UNCPS,3Tomsk,1SEMST,50IUFFC,9APTTE,2005IUS,ICU2007SC,ICU2007GA,2007MRS,3UNCPS,6DrogGorb,6Drog,
%4UNCPS,4Kremen,7Drog,5UNCPS,2012Ternop,14Plivk,8Drog,2013Buk,6UNCPS,2014IUSOl,2014IUS,6SEMST,
%2015ICU,6CPFCS,7UNCPS,2017MEICS}
