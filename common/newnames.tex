% Новые переменные, которые могут использоваться во всём проекте
% ГОСТ 7.0.11-2011
% 9.2 Оформление текста автореферата диссертации
% 9.2.1 Общая характеристика работы включает в себя следующие основные структурные
% элементы:
% актуальность темы исследования;
\newcommand{\actualityTXT}{Актуальність темы.}
% степень ее разработанности;
%\newcommand{\progressTXT}{Степень разработанности темы.}
% цели и задачи;
\newcommand{\aimTXT}{Метою}
\newcommand{\tasksTXT}{задачі}
% научную новизну;
\newcommand{\noveltyTXT}{Наукова новизна:}
% теоретическую и практическую значимость работы;
%\newcommand{\influenceTXT}{Теоретическая и практическая значимость}
% или чаще используют просто
%\newcommand{\influenceTXT}{Практическая значимость}
% методологию и методы исследования;
%\newcommand{\methodsTXT}{Mетодология и методы исследования.}
% положения, выносимые на защиту;
%\newcommand{\defpositionsTXT}{Основные положения, выносимые на~защиту:}
% степень достоверности и апробацию результатов.
%\newcommand{\reliabilityTXT}{Достоверность}
%\newcommand{\probationTXT}{Апробация работы.}

%\newcommand{\contributionTXT}{Личный вклад.}
%\newcommand{\publicationsTXT}{Публикации.}


\newcommand{\authorbibtitle}{Список публікацій здобувача за темою дисертації}
%\newcommand{\vakbibtitle}{В изданиях из списка ВАК РФ}
%\newcommand{\notvakbibtitle}{В прочих изданиях}
%\newcommand{\confbibtitle}{В сборниках трудов конференций}
\newcommand{\fullbibtitle}{Список використаних джерел} % (ГОСТ Р 7.0.11-2011, 4)



%%% Переопределение именований %%%
\renewcommand{\contentsname}{Зміст} % (ГОСТ Р 7.0.11-2011, 4)
\renewcommand{\figurename}{Рисунок} % (ГОСТ Р 7.0.11-2011, 5.3.9)
\renewcommand{\tablename}{Таблиця} % (ГОСТ Р 7.0.11-2011, 5.3.10)
\renewcommand{\listfigurename}{Перелік рисунків}
\renewcommand{\listtablename}{Перелік таблиць}
\renewcommand{\bibname}{\fullbibtitle}
%\addto{\captionukrainian}{\renewcommand{\bibname}{\fullbibtitle}}
\renewcommand{\bibsection}{\chapter*{\fullbibtitle}\addcontentsline{toc}{chapter}{\fullbibtitle}}
%\renewcommand{\bibname}{Привіт}
%\renewcommand{\refname}{\fullbibtitle}

%%% Основные сведения %%%
\newcommand{\thesisAuthorLastName}{\todo{Оліх}}
\newcommand{\thesisAuthorOtherNames}{\todo{Олег Ярославович}}
\newcommand{\thesisAuthorInitials}{\todo{О.\,Я.}}
%\newcommand{\thesisAuthor}             % Диссертация, ФИО автора
%{%
%    \texorpdfstring{% \texorpdfstring takes two arguments and uses the first for (La)TeX and the second for pdf
%        \thesisAuthorLastName~\thesisAuthorOtherNames% так будет отображаться на титульном листе или в тексте, где будет использоваться переменная
%    }{%
%        \thesisAuthorLastName, \thesisAuthorOtherNames% эта запись для свойств pdf-файла. В таком виде, если pdf будет обработан программами для сбора библиографических сведений, будет правильно представлена фамилия.
%    }
%}
\newcommand{\thesisAuthor}             % Диссертация, ФИО автора
{\thesisAuthorLastName~\thesisAuthorOtherNames}% так будет отображаться на титульном листе или в тексте, где будет использоваться переменная

\newcommand{\thesisAuthorShort}        % Диссертация, ИОФ автора инициалами
{\thesisAuthorInitials~\thesisAuthorLastName}

\newcommand{\thesisAuthorFIO}        % Диссертация, ФИО автора инициалами
{\thesisAuthorLastName~\thesisAuthorInitials}

\newcommand{\thesisAuthorFIOen}        % Диссертация, ФИО автора инициалами
{\todo{Olikh~O.\,Ya.}}


\newcommand{\thesisUdk}                % Диссертация, УДК
{\todo{534.29, 537.312.5/.6/.9}}
\newcommand{\thesisTitle}              % Диссертация, название
{\todo{Акусто--індуковані ефекти в опромінених та неопромінених напівпровідникових структурах}}
\newcommand{\thesisSpecialtyNumber}    % Диссертация, специальность, номер
{\todo{104}}
\newcommand{\thesisSpecialtyTitle}     % Диссертация, специальность, название
{\todo{Фізика та астрономія}}
\newcommand{\thesisKnowledgeTitle}     % Диссертация, галузь знань
{\todo{Природничі науки}}
\newcommand{\thesisKnowledgeNumber}     % Диссертация, шифр галузі знань
{\todo{10}}

\newcommand{\thesisDegree}             % Диссертация, ученая степень
{\todo{доктора фізико-математичних наук}}
\newcommand{\thesisDegreeShort}        % Диссертация, ученая степень, краткая запись
{\todo{д-р~фіз.-мат. наук}}
\newcommand{\thesisCity}               % Диссертация, город написания диссертации
{\todo{Київ}}
\newcommand{\thesisCityEn}               % Диссертация, город написания диссертации
{\todo{Kyiv}}
\newcommand{\thesisYear}               % Диссертация, год написания диссертации
{\todo{2018}}
\newcommand{\thesisOrganization}       % Диссертация, организация
{\todo{Київський національний університет імені Тараса Шевченка}}
%\newcommand{\thesisOrganizationShort}  % Диссертация, краткое название организации для доклада
%{\todo{НазУчДисРаб}}

\newcommand{\thesisOrganizationEn}       % Диссертация, организация
{\todo{Taras Shevchenko National University of Kyiv}}


\newcommand{\thesisInOrganization}     % Диссертация, организация в предложном падеже: Работа выполнена в ...
{\todo{Київському національному університеті імені Тараса Шевченка}}

\newcommand{\thesisMON}       % Диссертация, организация
{\todo{Міністерство освіти і науки України}}



\newcommand{\supervisorFio}            % Научный руководитель, ФИО
{\todo{Іванов Іван Іванович}}
\newcommand{\supervisorRegalia}        % Научный руководитель, регалии
{\todo{доктор фізико--математичних наук, професор}}
\newcommand{\supervisorFioShort}       % Научный руководитель, ФИО
{\todo{І.\,І.~Іванов}}
\newcommand{\supervisorRegaliaShort}   % Научный руководитель, регалии
{\todo{д-р~фіз.-мат. наук,~професор}}


\newcommand{\opponentOneFio}           % Оппонент 1, ФИО
{\todo{Фамилия Имя Отчество}}
\newcommand{\opponentOneRegalia}       % Оппонент 1, регалии
{\todo{доктор физико-математических наук, профессор}}
\newcommand{\opponentOneJobPlace}      % Оппонент 1, место работы
{\todo{Не очень длинное название для места работы}}
\newcommand{\opponentOneJobPost}       % Оппонент 1, должность
{\todo{старший научный сотрудник}}

\newcommand{\opponentTwoFio}           % Оппонент 2, ФИО
{\todo{Фамилия Имя Отчество}}
\newcommand{\opponentTwoRegalia}       % Оппонент 2, регалии
{\todo{кандидат физико-математических наук}}
\newcommand{\opponentTwoJobPlace}      % Оппонент 2, место работы
{\todo{Основное место работы c длинным длинным длинным длинным названием}}
\newcommand{\opponentTwoJobPost}       % Оппонент 2, должность
{\todo{старший научный сотрудник}}

\newcommand{\leadingOrganizationTitle} % Ведущая организация, дополнительные строки
{\todo{Федеральное государственное бюджетное образовательное учреждение высшего профессионального образования с~длинным длинным длинным длинным названием}}

\newcommand{\defenseDate}              % Защита, дата
{\todo{DD mmmmmmmm YYYY~г.~в~XX часов}}
\newcommand{\defenseCouncilNumber}     % Защита, номер диссертационного совета
{\todo{Д\,123.456.78}}
\newcommand{\defenseCouncilTitle}      % Защита, учреждение диссертационного совета
{\todo{Название учреждения}}
\newcommand{\defenseCouncilAddress}    % Защита, адрес учреждение диссертационного совета
{\todo{Адрес}}
\newcommand{\defenseCouncilPhone}      % Телефон для справок
{\todo{+7~(0000)~00-00-00}}

\newcommand{\defenseSecretaryFio}      % Секретарь диссертационного совета, ФИО
{\todo{Фамилия Имя Отчество}}
\newcommand{\defenseSecretaryRegalia}  % Секретарь диссертационного совета, регалии
{\todo{д-р~физ.-мат. наук}}            % Для сокращений есть ГОСТы, например: ГОСТ Р 7.0.12-2011 + http://base.garant.ru/179724/#block_30000

\newcommand{\synopsisLibrary}          % Автореферат, название библиотеки
{\todo{Название библиотеки}}
\newcommand{\synopsisDate}             % Автореферат, дата рассылки
{\todo{DD mmmmmmmm YYYY года}}


\newcommand{\FigCaptionSSC}
{\todo{
Криві 1 та 5 (незаповнені точки) отримані без УЗН,
решта --- під час УЗН: U--L (криві 2 та 6),
U--Ts1 (3),  U--Ts2 (7) та U--Tb3 (4 та 8).
}}

% To avoid conflict with beamer class use \providecommand
\providecommand{\keywords}%            % Ключевые слова для метаданных PDF диссертации и автореферата
{ультразвук, гамма-опромінення, кремній, бар'єрні структури, акусто--дефектна взаємодія, перенесення заряду, оборотні акусто--індуковані зміни}

\providecommand{\keywordsEn}%            % Ключевые слова для метаданных PDF диссертации и автореферата
{ultrasound, gamma-rays, silicon, barrier structures, acousto-defect interaction, charge transport, reversible acoustically--induced change}

\renewcommand{\chaptername}{Розділ}
\renewcommand{\appendixname}{Додаток} % (ГОСТ Р 7.0.11-2011, 5.7)

