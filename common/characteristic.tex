
%{\actuality} Обзор, введение в тему, обозначение места данной работы в
%мировых исследованиях и~т.\:п., можно использовать ссылки на~другие
%работы\ifnumequal{\value{bibliosel}}{1}{~\autocite{Gosele1999161}}{}
%(если их~нет, то~в~автореферате
%автоматически пропадёт раздел <<Список литературы>>). Внимание! Ссылки
%на~другие работы в разделе общей характеристики работы можно
%использовать только при использовании \verb!biblatex! (из-за технических
%ограничений \verb!bibtex8!. Это связано с тем, что одна
%и~та~же~характеристика используются и~в~тексте диссертации, и в
%автореферате. В~последнем, согласно ГОСТ, должен присутствовать список
%работ автора по~теме диссертации, а~\verb!bibtex8! не~умеет выводить в одном
%файле два списка литературы).
%При использовании \verb!biblatex! возможно использование исключительно
%в~автореферате подстрочных ссылок
%для других работ командой \verb!\autocite!, а~также цитирование
%собственных работ командой \verb!\cite!. Для этого в~файле
%\verb!Synopsis/setup.tex! необходимо присвоить положительное значение
%счётчику \verb!\setcounter{usefootcite}{1}!.
%
%Для генерации содержимого титульного листа автореферата, диссертации
%и~презентации используются данные из файла \verb!common/data.tex!. Если,
%например, вы меняете название диссертации, то оно автоматически
%появится в~итоговых файлах после очередного запуска \LaTeX. Согласно
%ГОСТ 7.0.11-2011 <<5.1.1 Титульный лист является первой страницей
%диссертации, служит источником информации, необходимой для обработки и
%поиска документа>>. Наличие логотипа организации на титульном листе
%упрощает обработку и поиск, для этого разметите логотип вашей
%организации в папке images в формате PDF (лучше найти его в векторном
%варианте, чтобы он хорошо смотрелся при печати) под именем
%\verb!logo.pdf!. Настроить размер изображения с логотипом можно
%в~соответствующих местах файлов \verb!title.tex!  отдельно для
%диссертации и автореферата. Если вам логотип не~нужен, то просто
%удалите файл с логотипом.
%
%\ifsynopsis
%Этот абзац появляется только в~автореферате.
%Для формирования блоков, которые будут обрабатываться только в~автореферате,
%заведена проверка условия \verb!\!\verb!ifsynopsis!.
%Значение условия задаётся в~основном файле документа (\verb!synopsis.tex! для
%автореферата).
%\else
%Этот абзац появляется только в~диссертации.
%Через проверку условия \verb!\!\verb!ifsynopsis!, задаваемого в~основном файле
%документа (\verb!dissertation.tex! для диссертации), можно сделать новую
%команду, обеспечивающую появление цитаты в~диссертации, но~не~в~автореферате.
%\fi
%
%% {\progress}
%% Этот раздел должен быть отдельным структурным элементом по
%% ГОСТ, но он, как правило, включается в описание актуальности
%% темы. Нужен он отдельным структурынм элемементом или нет ---
%% смотрите другие диссертации вашего совета, скорее всего не нужен.
%
%{\aim} данной работы является \ldots
%
%Для~достижения поставленной цели необходимо было решить следующие {\tasks}:
%\begin{enumerate}
%  \item Исследовать, разработать, вычислить и~т.\:д. и~т.\:п.
%  \item Исследовать, разработать, вычислить и~т.\:д. и~т.\:п.
%  \item Исследовать, разработать, вычислить и~т.\:д. и~т.\:п.
%  \item Исследовать, разработать, вычислить и~т.\:д. и~т.\:п.
%\end{enumerate}
%
%
%{\novelty}
%\begin{enumerate}
%  \item Впервые \ldots
%  \item Впервые \ldots
%  \item Было выполнено оригинальное исследование \ldots
%\end{enumerate}
%
%{\influence} \ldots
%
%{\methods} \ldots
%
%{\defpositions}
%\begin{enumerate}
%  \item Первое положение
%  \item Второе положение
%  \item Третье положение
%  \item Четвертое положение
%\end{enumerate}
%В папке Documents можно ознакомиться в решением совета из Томского ГУ
%в~файле \verb+Def_positions.pdf+, где обоснованно даются рекомендации
%по~формулировкам защищаемых положений.
%
%{\reliability} полученных результатов обеспечивается \ldots \ Результаты находятся в соответствии с результатами, полученными другими авторами.
%
%

{\actualityTXT}

{\InterconnectionTXT}
Дисертаційна робота  пов’язана із планами науково--дослідних робіт, які проводились в рамках 
держбюджетних тем та міжнародних проектів на кафедрі загальної фізики фізичного факультету \thesisOfOrganization.
А саме:
\textnumero01БФ051--09 <<Теоретичне та експериментальне дослідження фізичних властивостей неоднорідних систем на основі матеріалів акусто--опто--електроніки та мікроелектроніки>> 
(\textnumero~держ. реєстрації 01БФ051--09, 2001--2005рр.);
\textnumero06БФ051--04 <<Експериментальне та теоретичне дослідження структури та фізичних властивостей низькорозмірних систем на основі напівпровідникових структур, різних модифікацій вуглецю та композитів>> 
(\textnumero~держ. реєстрації 0106U006390, 2006--2010рр.);
\textnumero11БФ051--01 <<Фундаментальні дослідження в галузі фізики конденсованого стану і елементарних частинок, астрономії і матеріалознавства для створення основ новітніх технологій>>
(\textnumero~держ. реєстрації 0111U004954, 2011--2015рр.);
\textnumero16БФ051--01 <<Формування та фізичні властивості наноструктурованих композитних матеріалів та функціональних поверхневих шарів на основі карбону, напівпровідникових та діелектричних складових>>
(\textnumero~держ. реєстрації ????, 2016--2018рр.) та
проект УНТЦ №3555 <<Дослідження та створення методів опто-- акустичного контролю матеріалів>> (2006--2008рр.).

{\AimAndTasksTXT}

{\ObjectTXT} --

{\PredmetTXT}--

{\MethodTXT}
Для виконання поставлених завдань було використано комплекс технологічних, експериментальних та розрахункових методів, який включає
вольт--амперні характеристики;
вольт--фарадні характеристики;
метод диференційних коефіцієнтів ВАХ для визначення параметрів глибоких рівнів;
метод стаціонарного струму короткого замикання (SSSCC) для визначення довжини дифузії неосновних носіїв;
аналітичні та чисельні методи визначення параметрів діодів Шотки;
еволюційні алгоритми мінімізації функції;
імпульсний метод вимірювання коефіцієнта поглинання акустичної хвилі;
резонансний метод вимірювання імпедансу навантаженого акустичного перетворювача;
акустоелектрична релаксаційна спектроскопія глибоких рівнів;
профілометрія;
метод визначення деформації приповерхневих кристалічних площин по зміні кутового положення дифракційного максимуму при трансляції зразка;
контрольоване радіаційне та мікрохвильове опромінення для зміни дефектного стану зразків;
метод ультразвукового навантаження.



{\noveltyTXT}


{\influenceTXT}
Отримані в роботі результати сприяють більш глибокому розумінню фізичних процесів у поверхнево--бар'єрних структурах при дії зовнішніх чинників 
(надвисокочастотного--, нейтронного-- та гамма--опромінення, знакозмінних механічних навантажень), що 
дозволяє підвищити точність прогнозування реальних робочих характеристик подібних систем в залежності від умов їх функціонування.
Запропоновано новий метод динамічного акустичного керування струмом напівпровідникових діодів різного типу, а саме сонячних елементів та структур з контактом Шотки.
Дослідження частотних, амплітудних та температурних залежностей акустоіндукованих ефектів у бар'єрних структурах дозволяє 
ефективно контролювати процеси перенесення заряду.
Проведене тестування та порівняльне дослідження різноманітних методів визначення параметрів діодів Шотки дозволяє вибрати найфективніший залежно 
від експериментальних умов вимірювання характеристик, типу структур, вимог до швидкодії.
Запропоновано новий метод оптимізації вибору діапазону даних для побудови аналітичних функцій, що дозволяє підвищити точність визначення параметрів структур металл--напівпровідник.
%Запропоновано новий принцип дії сенсорів
Виявлені зміни особливостей акустоіндукованих ефектів у бар'єрних структурах після опромінення 
можуть бути використані для створення нових сенсорів типу та дози радіації.
А саме, амплітудна залежність АІ змін зворотного струму діодів Шотки дозволяє оцінити поглинуту дозу гамма--квантів, тоді
як величина та знак впливу ультразвуку на фактор неідельності та рекомбінаційний струм кремнієвих $p-n$ структур дозволяють розрізнити 
нейтронно-- та гамма--опромінені структури.




{\contributionTXT}
Внесок автора у отримання наукових результатів полягає у постановці задач
та визначенні методів їх вирішення, виборі об'єктів та формулюванні
основних напрямків досліджень,
розробці методології експериментальних досліджень та програмного забезпечення для обробки експериментальних даних.
Переважна більшість експериментальних та теоретичних досліджень виконані автором особисто.
12 з 25 наукових публікацій опублікованих за темою дисертації є одноосібними роботами здобувача.
У наукових працях, опублікованих зі співавторами, автору належить проведення значної частини досліджень та аналіз і узагальнення отриманих
даних, інтерпретація результатів, участь у написанні наукових статей.
Співавторами частини робіт (\cite{Olikh2018JAP,Olikh:Ultras2016,Olikh2016JSem,OlikhJAP,Olikh:PZTF2006}) були студенти фізичного факультету \thesisOfOrganization,
які виконували кваліфікаційні роботи під керівництвом здобувача.
В роботах \cite{Olikh2018JAP,Olikh:Ultras2016,Olikh2016JSem,OlikhJAP,Olikh:SEMT2007,Olikh:MRS2007,Olikh:PZTF2006} автором здійснено підбір структур для досліджень, вибір режимів вимірювань та радіаційного опромінення,
проведено переважну частину експериментальних вимірювань та аналіз механізмів перенесення заряду і впливу ультразвукових хвиль на ці процеси,
підготовлено тексти статей.
В роботі \cite{Olikh2018JAP} автором запропоновано модель акустоактивного дефектного комплексу,
в роботі \cite{Olikh:Ultras2016} --- проведено аналіз можливості застосування моделі поглинання ультразвуку внаслідок руху дислокаційних перегинів до пояснення акустоіндукованих змін параметрів діодів Шотки.
Внесок здобувача у роботу \cite{Olikh:UPJ2014} визначався проведенням розрахунків в межах моделей дислокаційного поглинання ультразвуку.
В роботі \cite{Olikh:UPJ2013} вимірювання вольт--фарадних характеристик були проведені співробітником фізичного факультету, канд. фіз.--мат. наук Надточієм~А.\:Б.
Пошук та аналіз літературних даних щодо впливу ультразвуку на параметри напівпровідникових кристалів та структур на їх основі, а також їх узагальнення
у роботах \cite{Olikh:SEMT2004,Olikh:SEMT2011} проводилось сумісно з 
докт. фіз.--мат. наук Оліхом~Я.\:М. (Інститут фізики напівпровідників ім. В.\:Є.~Лашкарьова НАНУ).
Внесок здобувача у роботу \cite{Gorb2010} визначався постановкою дослідів по вимірюванню вольт--амперних характеристик,
інтерпретацією відповідних результатів (саме ця частину представлена у дисертаційній роботі), участю у написанні статті.
В роботах \cite{Olikh:PhChOM2005,Olikh:PJE2004} автор провів дослідження параметрів глибоких рівнів з використанням методу акустоелектронної релаксаційної спектроскопії,
здійснив аналіз отриманих даних, взяв участь у написанні статей.
Постановка наукової задачі в цих роботах, а також загальна інтерпретація результатів виконана сумісно з докт. техн. наук Конаковою~Р.\:В.;
рентгенографічні та профілометричні дослідження проводились канд. фіз.--мат. наук Литвином~П.\:М.  (обидва --- Інститут фізики напівпровідників ім. В.\:Є.~Лашкарьова НАНУ).
Основна частина результатів 
представлялася автором особисто на вітчизняних і міжнародних конференціях 
та наукових семінарах кафедри загальної фізики \thesisOfOrganization.






{\probationTXT}
Основні результати, викладені в роботі, доповідались на наукових семінарах
кафедри загальної фізики \thesisOfOrganization
%Київського національного університету імені Тараса Шевченка
і були представлені на наступних наукових конференціях:
І, ІІІ, IV, V, VI та VII Українська наукова конференція з фізики напівпровідників
(Одеса, Україна, 2002; Одеса, Україна, 2007; Запоріжжя, Україна, 2009;
Ужгород, Україна, 2011; Чернівці, Україна, 2013; Дніпро, Україна, 2016);
III международная конференция <<Радиационно-термические эффекты и процессы в неорганических материалах>> (Томск, Россия, 2002);
1--ша та 6-та Міжнародна науково-технічна конференція <<Сенсорна електроніка і мікросистемні технології СЕМСТ>> (Одеса, Україна, 2004; 2014);
2004 IEEE International Ultrasonics, Ferroelectrics and Frequency Control Joint 50th Anniversary Conference (Montreal, Canada, 2004);
Девятая международная научно--техническая конференция <<Актуальные проблемы твердотельной электроники и микроэлектроники>> (Дивноморское, Россия, 2004);
2005 та 2014 IEEE International Ultrasonics Symposium (Rotterdam, Netherlands, 2005; Chicago, USA, 2014);
2007 та 2015 International Congress on Ultrasonics (Vienna, Austria, 2007; Metz, France, 2015);
MRS 2007 Spring Meeting, Symposium F: Semiconductor Defect Engineering – Materials, Synthetic Structures, and Devices II (San Francisco, USA, 2007);
VІ та VІІ Міжнародна школа-конференція <<Актуальні проблеми фізики напівпровідників>> (Дрогобич, Україна, 2008; 2010);
ХІІ та ХІV Міжнародна конференція <<Фізика і технологія тонких плівок та наносистем>> (Івано--Франківськ, Україна, 2009; Буковель, Україна, 2013);
Четверта міжнародна науково--практична конференція <<Матеріали електронної техніки та сучасні інформаційні технології>> (Кременчук, Україна, 2010);
Всеукраїнська наукова конференція <<Актуальні проблеми теоретичної, експериментальної та прикладної фізики>> (Тернопіль, Україна, 2012);
International research and practice conference <<Nanotechnology and nanomaterials>> (Bukovel, Ukraine, 2013);
IV міжнародна конференція <<Сучасні проблеми фізики конденсованого стану>> (Київ, Україна, 2015);
ІІ Всеукраїнська науково--практична конференція МЕІСS--2017 (Дніпро, Україна, 2017).

{\publicationsTXT}
За отриманими результатами опубліковано 25 наукових праць,
з них 24 статті у фахових журналах і 1 у матеріалах наукової конференції.



{\structureTXT}
Дисертація складається із вступу, шести розділів, загальних висновків та списку використаних джерел.
Загальних обсяг дисертації складає
%% на случай ошибок оставляю исходный кусок на месте, закомментированным
%\ref*{TotPages}~сторінки з~\totalfigures{}~рисунками та~\totaltables{}~таблицями.
%Список використаних джерел містить \total{citenum}~найменувань.
%
\formbytotal{TotPages}{сторінк}{у}{и}{ок}, включаючи
\formbytotal{totalcount@figure}{рисун}{ок}{ки}{ків} та
\formbytotal{totalcount@table}{таблиц}{ю}{і}{ь}.
%Список використаних джерел містить
%\formbytotal{citenum}{найменуван}{ня}{ь}{ь}.



%
%%\publications\ Основные результаты по теме диссертации изложены в ХХ печатных изданиях~\cite{Sokolov,Gaidaenko,Lermontov,Management},
%%Х из которых изданы в журналах, рекомендованных ВАК~\cite{Sokolov,Gaidaenko},
%%ХХ --- в тезисах докладов~\cite{Lermontov,Management}.
%
%\ifnumequal{\value{bibliosel}}{0}{% Встроенная реализация с загрузкой файла через движок bibtex8
%    \publications\ Основные результаты по теме диссертации изложены в XX печатных изданиях,
%    X из которых изданы в журналах, рекомендованных ВАК,
%    X "--- в тезисах докладов.%
%}{% Реализация пакетом biblatex через движок biber
%%Сделана отдельная секция, чтобы не отображались в списке цитированных материалов
%    \begin{refsection}[vak,papers,conf]% Подсчет и нумерация авторских работ. Засчитываются только те, которые были прописаны внутри \nocite{}.
%        %Чтобы сменить порядок разделов в сгрупированном списке литературы необходимо перетасовать следующие три строчки, а также команды в разделе \newcommand*{\insertbiblioauthorgrouped} в файле biblio/biblatex.tex
%        \printbibliography[heading=countauthorvak, env=countauthorvak, keyword=biblioauthorvak, section=1]%
%        \printbibliography[heading=countauthorconf, env=countauthorconf, keyword=biblioauthorconf, section=1]%
%        \printbibliography[heading=countauthornotvak, env=countauthornotvak, keyword=biblioauthornotvak, section=1]%
%        \printbibliography[heading=countauthor, env=countauthor, keyword=biblioauthor, section=1]%
%        \nocite{%Порядок перечисления в этом блоке определяет порядок вывода в списке публикаций автора
%                vakbib1,vakbib2,%
%                confbib1,confbib2,%
%                bib1,bib2,%
%        }%
%        \publications\ Основные результаты по теме диссертации изложены в~\arabic{citeauthor}~печатных изданиях,
%        \arabic{citeauthorvak} из которых изданы в журналах, рекомендованных ВАК,
%        \arabic{citeauthorconf} "--- в~тезисах докладов.
%    \end{refsection}
%    \begin{refsection}[vak,papers,conf]%Блок, позволяющий отобрать из всех работ автора наиболее значимые, и только их вывести в автореферате, но считать в блоке выше общее число работ
%        \printbibliography[heading=countauthorvak, env=countauthorvak, keyword=biblioauthorvak, section=2]%
%        \printbibliography[heading=countauthornotvak, env=countauthornotvak, keyword=biblioauthornotvak, section=2]%
%        \printbibliography[heading=countauthorconf, env=countauthorconf, keyword=biblioauthorconf, section=2]%
%        \printbibliography[heading=countauthor, env=countauthor, keyword=biblioauthor, section=2]%
%        \nocite{vakbib2}%vak
%        \nocite{bib1}%notvak
%        \nocite{confbib1}%conf
%    \end{refsection}
%}
%При использовании пакета \verb!biblatex! для автоматического подсчёта
%количества публикаций автора по теме диссертации, необходимо
%их~здесь перечислить с использованием команды \verb!\nocite!.
