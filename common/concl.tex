%% Согласно ГОСТ Р 7.0.11-2011:
%% 5.3.3 В заключении диссертации излагают итоги выполненного исследования, рекомендации, перспективы дальнейшей разработки темы.
%% 9.2.3 В заключении автореферата диссертации излагают итоги данного исследования, рекомендации и перспективы дальнейшей разработки темы.
\begin{enumerate}[leftmargin=0cm,itemindent=3em]
  \item Вперше експериментально досліджено вплив ультразвукового навантаження на параметри монокристалічних кремнієвих сонячних елементів у діапазоні температур $290\div340$~К
  та виявлена оборотна акустоіндукована деградація фотоелектричних властивостей, пов'язана зі зменшенням часу життя носіїв заряду в акустичному полі.
  Виявлено, що в умовах акустичного навантаження збільшується внесок у рекомбінаційні процеси більш мілких рівнів.
  Встановлено, що кисневмісні преципітати ефективно впливають на процеси рекомбінації та беруть участь у акусто--дефектній взаємодії.
  Запропоновано модель акустоактивного комплексного дефекту для пояснення особливостей акустоіндукованих ефектів.
 Виявлено ефект акустоіндукованого зменшення шунтуючого опору та запропоновано його пояснення із залученням моделі дислокаційно--індукованого імпедансу.

\item Вперше досліджено вплив ультразвукового навантаження на параметри кремнієвих структур з $p$-$n$ переходом, які були опромінені реакторними нейтронами та $\gamma$--квантами $^{60}$Co.
      Виявлено, що в опромінених структурах, порівняно з неопроміненими, спостерігається підвищення ефективності акустоіндукованого зменшення шунтуючого опору та часу життя неосновних носіїв заряду в базі діода.
      З'ясовано, що акустоіндуковані оборотні зміни фактора неідеальності та часу життя носіїв в області просторового заряду   мають різний знак в опромінених та неопромінених структурах.
      Встановлено, що нейтронно--опромінених діодах основними акустоактивними центрами є дивакансії,
      а в $\gamma$--опромінених --- комплекс вакансії та міжвузольного кисню
%      виявлені ефекти в нейтронно--опромінених діодах пов'язані зі впливом ультразвуку на стан дивакансій,  тоді як у гамма--опромінених діодах основним акустоактивним центром є комплекс вакансії та міжвузольного кисню.
     Виявлено, що комплекс з міжвузольного вуглецю та міжвузольного кисню практично не приймає участі в акусто--дефектній взаємодії.

\item  Проведено порівняльний аналіз та тестування 16 основних методів визначення параметрів діодів Шотткі з вольт--амперних характеристик.
         Спираючись на результати тестування методів на експериментальних та синтезованих  ВАХ,
         запропоновано шляхи оптимізації методів Nord, Bohlin та Mikhelashvili з метою збільшення точності розрахунку.
      Запропоновано адаптивну процедуру оптимізації вибору діапазону ВАХ, який використовується для побудови допоміжних функцій при застосуванні аналітичних методів визначення параметрів структур метал--напівпровідник.
       Показано, що така процедура дозволяє суттєво (приблизно на порядок при кімнатних температурах у випадку низького рівня похибок вимірювання) підвищити точність визначення параметрів.

   \item Встановлено, що найбільш ефективними методами з точки зору точності визначення параметрів та швидкості розрахунків є еволюційні алгоритми, метод Gromov з адаптивною процедурою та метод Lee.
    Показано, що використання функції Ламберта при чисельному визначенні параметрів діодів Шотткі дозволяє зменшити похибки.
    Визначено залежності точності визначення послідовного опору, висоти бар'єру Шотткі та фактора неідельності від величин параметрів та рівня випадкових помилок вимірювання вольт--амперних характеристик.

   \item
Встановлено, що при прямому зміщенні перенесення заряду в структурах Al$-n-n^+$--Si---Al з бар'єром Шотткі у діапазоні температур $130\div330$~К відбувається внаслідок термоелектронної емісії через неоднорідний контакт.
%Виконано експериментальне дослідження прямих і зворотних вольт--амперних характеристик структур Al$-n-n^+$--Si---Al з бар'єром Шотткі в діапазоні температур $130\div330$~К.
%Виявлено, що при підвищенні температури спостерігається збільшення висоти бар'єру та зменшення фактора неідеальності та встановлено механізм перенесення заряду із залученням моделі термоелектронної емісії через неоднорідний контакт у всьому діапазоні температур.
%       Показано, що отримані результати можна пояснити у рамках моделі термоелектронної емісії через неоднорідний контакт у всьому діапазоні температур.
        Показано, що при низьких температурах ($T<220$~К) суттєвим стає проходження заряду через області зі зниженим бар'єром і визначено середнє значення висоти бар'єру Шотткі в цих областях.
%          --- $54\pm4$~мВ.
     Виявлено, що при зворотному зміщенні в структурах Al---$n$--$n^+$--Si---Al перенесення заряду відбувається як внаслідок термоелектронної емісії через неоднорідний бар'єр, так і завдяки процесам тунелювання через глибокий центр (міжвузольний атом вуглецю).

\item
%Проведено експериментальне дослідження впливу $\gamma$--ви\-про\-мі\-ню\-ван\-ня $^{60}$Co на електрофізичні параметри структур Al$-n-n^+$--Si---Al.
     Показано, що опромінення $\gamma$-квантами $^{60}$Co структур Al$-n-n^+$--Si---Al суттєво підсилює процеси тунелювання носіїв заряду як при прямому зміщенні, так і при зворотному.
     Встановлено, що при прямому зміщенні тунельний механізм перенесення заряду стає основним в низькотемпературній області ($T<250$~К),
а при зворотному --- з'являється компонента струму, пов'язана з багатофононним тунелюванням.
 Виявлено, що висота бар'єру, фактор неідеальності та величина зворотного струму немонотонно змінюються при збільшенні поглинутої дози.
З'ясовано, що у випадку поглинутої дози $10^6$~рад зміна
електрофізичних
параметрів відбувається внаслідок накопичення дефектів акцепторного типу на границі метал--напівпровідник та укрупнення патчів, викликаного радіаційно підсиленим дислокаційним ковзанням.
При $10^7$~рад
визначальними механізмами змін властивостей діодів Шотткі є інтенсифікація процесів тунелювання внаслідок утворення значної кількості радіаційних дефектів та гетерування останніх в областях зі зниженим бар'єром.
Встановлено взаємозв'язок характеру дозової немонотонності зміни висоти бар'єру Шотткі та ступеню неоднорідності контакту.
%Показано, що характер дозової немонотонності зміни висоти бар'єру Шотки різний для   однорідних областей та для всього діоду загалом.


\item
Вперше досліджено вплив ультразвукового навантаження у динамічному режимі при кімнатній температурі на параметри кремнієвих діодів Шотткі Al$-n-n^+$--Si---Al.
Виявлено, що при поширенні акустичних хвиль спостерігаються оборотні зменшення висоти бар'єру,
збільшення зворотного струму та струму насичення, тоді як фактор неідеальності практично не змінюється.
З'ясовано, що ультразвукове навантаження практично не впливає на процеси прямого тунелювання та багатофононного тунелювання.
Встановлено, що вплив акустичного навантаження на термоемісійну складову струму структур пояснюється іонізацією дефектів на межі метал--напівпровідник
  внаслідок взаємодії ультразвуку з дислокаціями та радіаційними точковими порушеннями періодичності в неопромінених та опромінених структурах, відповідно.

\item
Вперше експериментально досліджено динамічний вплив ультразвукового навантаження в діапазоні частот $8\div28$~МГц на електричні властивості структур Mo/$n$--$n^{+}$--Si з бар'єром Шотткі в діапазоні температур $130\div330$~К.
 Виявлено акустоіндуковані оборотні зміни фактора неідеальності та висоти бар'єру Шотткі, причому зміни немонотонно залежать від температури і найбільш ефективний вплив ультразвуку спостерігається поблизу 200~K.
  Показано, що зі збільшенням частоти ультразвуку  спостерігається як загальне підвищення ефективності акустичного впливу на параметри кремнієвих діодів Шотткі,
так і зростання температури максимуму ефективності.
 Використовуючи модель неоднорідного контакту встановлено, що за умов ультразвукового навантаження відбувається збільшення висоти бар'єру як в області розташування патчів, так і за їх межами, а також уширюється розподіл параметрів патчів та збільшується їх ефективна густина.
З'ясовано, що механізм акустоіндукованих змін параметрів структур Mo/$n$--$n^{+}$--Si пов'язаний з рухом дислокаційних перегинів.
% Показано, що частотні та температурні особливості акустоіндукованих змін параметрів структур Mo/$n$--$n^{+}$--Si можуть бути пояснені в рамках
% моделі поглинання ультразвуку внаслідок руху дислокаційних перегинів.

\item Виявлено ефект оборотного збільшення зворотного струму структур Mo/$n$--$n^{+}$--Si за умов акустичного навантаження.
Встановлено, що ефект послаблюється при збільшенні температури та зміщення та посилюється при зростанні частоти ультразвуку.
Показано, що основними механізмами зворотного струму є термоелектронна емісія та тунелювання, стимульоване фононами;
в умовах поширення акустичних хвиль відбувається зменшення енергії активації рівнів, що беруть участь у тунелюванні,
густини заповнених інтерфейсних станів та коефіцієнта Пула--Френкеля.

\item Виявлено вплив мікрохвильового опромінення на параметри точкових дефектів у монокристалах $n$--6$H$--SiC, $n$--GaAs та епітаксійних структурах на основі арсеніду галію.
Встановлено, що причинами радіаційностимульованих змін поперечного перерізу захоплення електронів та розташування енергетичних рівнів пасток у забороненій зоні є
збільшення кількості міжвузольних атомів у приповерхневому шарі.
Показано, що викликані високочастотним опроміненням процеси перетворення дефектних комплексів інтенсифікуються за наявності механічних напруг.

\item Вперше експериментально досліджено вплив ультразвукової обробки на параметри структури Au--TiB$_x$--$n$--$n^+$--GaAs з контактом Шотткі
 залежно від частоти та потужності акустичних хвиль.
 Встановлено, що при допороговій (менше 2,5~Вт/см$^2$) інтенсивності ультразвуку відбувається збільшення однорідності параметрів арсенід галієвих діодів Шотткі, створених в єдиному технологічному процесі, пов'язане з
 акустостимульованою дифузією точкових дефектів.



\item Встановлено, що ультразвукова обробка викликає зменшення концентрації та звуження енергетичного спектра радіаційноіндукованих пасток  на інтерфейсі системи   Si--SiO$_2$.

%  На основе анализа \ldots
%  \item Численные исследования показали, что \ldots
%  \item Математическое моделирование показало \ldots
%  \item Для выполнения поставленных задач был создан \ldots
\end{enumerate}

