%% Согласно ГОСТ Р 7.0.11-2011:
%% 5.3.3 В заключении диссертации излагают итоги выполненного исследования, рекомендации, перспективы дальнейшей разработки темы.
%% 9.2.3 В заключении автореферата диссертации излагают итоги данного исследования, рекомендации и перспективы дальнейшей разработки темы.
\begin{enumerate}
  \item Вперше експериментально досліджено вплив ультразвукового навантаження на параметри монокристалічних кремнієвих сонячних елементів у діапазоні температур $290\div340$~К
  та виявлена оборотна акустоіндукована деградація фотоелектричних властивостей, пов'язана зі зменшенням часу життя носіїв заряду в акустичному полі.
  Виявлено, що в умовах акустичного навантаження збільшується внесок у рекомбінаційні процеси більш мілких рівнів, а ефективність взаємодії ультразвукових хвиль з точковими дефектами зростає з підвищенням частоти пружних коливань.
  Запропонована якісна модель акусто-активного комплексного дефекту, в рамках якої пояснено особливості акустоіндукованих ефектів.
  Досліджено можливу роль різних комплексів у визначенні властивостей структур та показано, що саме кисневмісні преципітати ефективно впливають на процеси рекомбінації та беруть участь у акусто--дефектній взаємодії.
 Виявлено ефект акустоіндукованого зменшення шунтуючого опору та запропоновано його пояснення в межах моделі дислокаційно-індукованого імпедансу.

\item Вперше досліджено вплив ультразвукового навантаження на параметри кремнієвих структур з $p$-$n$ переходом, які були опромінені реакторними нейтронами та $\gamma$--квантами $^{60}$Co.
      Виявлено, що в опромінених структурах, порівняно з неопроміненими, спостерігається підвищення ефективності акустоіндукованого зменшення шунтуючого опору та часу життя неосновних носіїв заряду в базі діоду.
      З'ясовано, що акустоіндуковані оборотні зміни фактору неідеальності та часу життя носіїв в області просторового заряду   мають різний знак в опромінених та неопромінених структурах.
      Встановлено, що виявлені ефекти в нейтронно--опромінених діодах пов'язані зі впливом ультразвуку на стан дивакансій,  тоді як в гамма--опромінених діодах основним акустоактивним центром є комплекс вакансії та міжвузольного кисню.
     Отримані результати свідчать, що ультразвукове навантаженні викликає перебудову комплексу VO$_i$.
     Водночас виявлено, що комплекс з міжвузольного вуглецю та міжвузольного кисню практично не приймає участі в акусто--дефектній взаємодії.

\item  Проведено порівняльний аналіз та тестування 16 основних відомих методів визначення параметрів діодів Шотки з вольт--амперних характеристик.
         Спираючись на результати тестування методів на експериментальних та синтезованих  ВАХ,
         запропоновано шляхи оптимізації методів Nord, Bohlin та Mikhelashvili з метою збільшення точності розрахунку.
      Запропоновано адаптивну процедуру для оптимізації вибору діапазону ВАХ, який використовується для побудови допоміжних функцій при застосуванні аналітичних методів визначення параметрів структур метал--напівпровідник та показано, що вона дозволяє суттєво (приблизно на порядок при кімнатних температурах у випадку низького рівня похибок вимірювання) підвищити точність визначення параметрів і не викликає критичного збільшення часу, необхідного для розрахунків.

   \item Показано, що найбільш придатними методами з точки зору точності визначення параметрів є еволюційні алгоритми (особливо MABC завдяки найменшому часу розрахунку), метод Gromov з адаптивною процедурою та метод Lee.
    Показано, що використання функції Ламберта при чисельному визначенні параметрів діодів Шотки дозволяє зменшити похибки визначення та вплив на них інших чинників; з іншого боку, час роботи алгоритму зростає.
    Проаналізовано залежності точностей визначення послідовного опору, висоти бар'єру Шотки та фактора неідельності від величин параметрів та рівня випадкових помилок вимірювання вольт--амперних характеристик.

   \item
Проведено експериментальне дослідження прямих і зворотних вольт--амперних характеристик структур Al$-n-n^+$--Si---Al з бар'єром Шотки в діапазоні температур $130\div330$~К.
Виявлено, що при підвищенні температури спостерігається збільшення висоти бар'єру та зменшення фактору неідеальності.
       Показано, що отримані результати можна пояснити у рамках моделі термоелектронної емісії через неоднорідний контакт у всьому діапазоні температур.
         Показано, що при низьких температурах ($T<220$~К) суттєвим стає проходження заряду через області зі зниженим бар'єром і визначено середнє значення висоти бар'єру Шотки в цих областях --- $54\pm4$~мВ.
     Виявлено, що при зворотному зміщенні в структурах Al$-n-n^+$--Si---Al перенесення заряду відбувається як внаслідок термоелектронної емісії через неоднорідний бар'єр, так і завдяки процесам прямого тунелювання через глибокий центр,
          яким, імовірно, є міжвузольний атом вуглецю.

\item Проведено експериментальне дослідження впливу $^{60}$Co $\gamma$--ви\-про\-мі\-ню\-ван\-ня $^{60}$Co на електрофізичні параметри структур Al$-n-n^+$--Si---Al.
     Показано, що радіаційне опромінення суттєво підсилює процеси тунелювання носіїв заряду як при прямому зміщенні, так і при зворотному.
     Встановлено, що при прямому зміщенні тунельний механізм перенесення струму стає основним в низькотемпературній області ($T<250$~К),
а при зворотному --- з'являється компонента струму, пов'язана з багатофононним тунелюванням.
 Виявлено, що висота бар'єру, фактор неідеальності та величина зворотного струму немонотонно змінюються при збільшенні поглинутої дози.
Встановлено, що для низьких значень поглинутої дози зміна електрофізичних параметрів $\gamma$--оп\-ро\-мі\-не\-них структур
відбувається внаслідок накопичення дефектів акцепторного типу на границі розділу
метал--напівпровідник та укрупнення патчів внаслідок радіаційно підсиленного дислокаційного ковзання, тоді як при високих значеннях поглинутої дози
цей ефект маскується інтенсифікацією процесів тунелювання внаслідок утворення значної кількості радіаційних дефектів.
Показано, що характер дозової немонотонності зміни висоти бар'єру Шотки різний для   однорідних областей та для всього діоду загалом.

\item
Вперше досліджено вплив ультразвукового навантаження у динамічному режимі при кімнатній температурі на параметри кремнієвих діодів Шотки Al$-n-n^+$--Si---Al.
Виявлено, що при поширенні акустичних хвиль спостерігаються оборотні зменшення висота бар'єру,
збільшення зворотного струму та струму насичення, в той час як фактор неідеальності практично не змінюється.
Встановлено, що ультразвукове навантаження практично не впливає на процеси прямого тунелювання та багатофононного тунелювання
Показано, що вплив акустичного навантаження на термоемісійну складову струму структур можна пояснити іонізацією дефектів на межі металл--напівпровідник
  внаслідок взаємодії ультразвуку з дислокаціями та радіаційними точковими порушеннями періодичності в неопромінених та опромінених структурах, відповідно.

\item
Вперше експериментально досліджено динамічний вплив ультразвукового навантаження в діапазоні частот $8\div28$~МГц на електричні властивості структур Mo/$n$-$n^{+}$--Si з бар'єром Шотки в діапазоні температур $130\div330$~К.
 Виявлено акустоіндуковані оборотні зміни фактору неідеальності та висоти бар'єру Шотки, причому зміни немонотонно залежать від температури і найбільш ефективний вплив УЗ спостерігається поблизу 200~K.
  Показано, що зі збільшенням частоти УЗ  спостерігається як підвищення ефективності акустичного впливу на параметри кремнієвих діодів Шотки,
так і зростання температури максимуму ефективності.
 Застосування моделі неоднорідного контакту показало, що за умов ультразвукового навантаження відбувається збільшення висоти бар'єру як в області знаходження патчів, так і за їх межами, а також уширюється розподіл параметрів патчів та збільшується їх ефективна густина.
Показано, що частотні та температурні особливості акустоіндукованих змін параметрів структур Mo/$n$--$n^{+}$--Si можуть бути пояснені в рамках
 моделі поглинання ультразвуку внаслідок руху дислокаційних перегинів.

\item Вперше виявлено ефект оборотного збільшення зворотного струму структур Mo/$n$--$n^{+}$--Si за умов їх акустичного навантаження;
показано, що ефект послаблюється при збільшенні температури та зміщення та посилюється при зростанні частоти ультразвуку.
Показано, що основними механізмами зворотного струму є термоелектронна емісія та тунелювання, стимульоване фононами;
в умовах поширення акустичних хвиль відбувається зменшення енергії активації рівнів, що беруть участь у тунелюванні,
густини заповнених інтерфейсних станів та коефіцієнта Пула--Френкеля.

\item Експериментально досліджено вплив мікрохвильового опромінення на параметри точкових дефектів в монокристалах $n$--6$H$--SiC, $n$--GaAs та епітаксійних структурах на основі арсеніду галію.
Показано, що причинами радіаційностимульованих змін поперечного перерізу захоплення електронів та розташування енергетичних рівнів пасток у забороненій зоні є
збільшення кількості міжвузольних атомів у приповерхневому шарі та релаксація внутрішніх механічних напруг.
Показано, що наявність стискуючих напруг у приповерхневому шарі прискорює процеси перетворення дефектних комплексів внаслідок високочастотного опромінення.

\item Вперше експериментально досліджено вплив ультразвукової обробки на параметри структури Au--TiB$_x$--$n$--$n^+$--GaAs з контактом Шотки
 в залежності від частоти та потужності акустичної обробки.
Встановлено, що при малій інтенсивності акустичної обробки (менше 2,5~Вт/см$^2$) характер УЗ впливу за величину зворотного струму залежить від механізму перенесення заряду:
  якщо домінуючим механізмом є тунельний, то ультразвукова обробка викликає збільшення зворотного струму, якщо термоемісійний --- зменшення.
  Показано, що причиною виявлених ефектів може бути акусто--стимульована дифузія точкових дефектів.



\item Показано, що ультразвукова обробка здатна викликати гомогенізацію як параметрів кремнієвих діодів Шотки, створених в єдиному технологічному процесі, так і енергетичного спектру радіаційноіндукованих пасток
    на інтерфейсі системи    Si--SiO$_2$.

%  На основе анализа \ldots
%  \item Численные исследования показали, что \ldots
%  \item Математическое моделирование показало \ldots
%  \item Для выполнения поставленных задач был создан \ldots
\end{enumerate}

