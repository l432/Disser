\begin{center}
\section*{\MakeUppercase{анотація}}
\end{center}
\textbf{\thesisAuthorFIO~\thesisTitle.} --- Рукопис.

\abstractBegin

Досліджено вплив ультразвукового навантаження на проходження струму в кремнієвих сонячних елементах,
 у тому числі опромінених.
 Виявлено оборотні процеси зменшення ефективності фотоелектричного перетворення,
 пов'язані з акустоіндукованою перебудовою точкових рекомбінаційних центрів.
 Запропонована модель акустоактивного комплексного дефекту.
% Показано, що акустоактивними радіаційними дефектами в кремнії є дивакансія та А--центр.
 Проведено порівняльний аналіз аналітичних, чисельних та еволюційних методів визначення параметрів діодів Шотки.
 Показано взаємозв'язок між характером немонотонності дозової залежності зміни висоти бар'єру Шотки та ступенем неоднорідності контакту.
 Виявлено оборотній вплив ультразвуку на властивості структур кремній--метал, встановлено його закономірності
 та показано, що він пов'язаний з рухом дислокаційних перегинів.
 Показано, що вплив мікрохвильового опромінення на дефектну структуру приповерхневого шару монокристалів GaAs і SiC та епітаксійних структур GaAs
 пов'язаний зі зростанням концентрації міжвузольних атомів.
 Виявлено, що ультразвукова обробка викликає гомогенізацію параметрів арсенід галієвих діодів Шотки та енергетичного спектру радіаційноіндукованих пасток  на інтерфейсі системи  Si--SiO$_2$.

\keywords


\begin{center}
{\section*{\MakeUppercase{АННОТАЦИЯ}}}
\end{center}
\textbf{\thesisAuthorFIOru~\thesisTitleRu.} --- Рукопись.

\abstractBeginRu

В диссертационной работе проведено исследование влияния ультразвукового нагрузки на процессы переноса заряда в кремниевых солнечных элементах,
 в том числе и радиационно облученных.
 Выявлено обратимые процессы уменьшения эффективности фотоэлектрического преобразования,
 связанные с акустоиндукованной перестройкой точечных рекомбинационных центров.
  Предложенная модель комлексного акустоактивного дефекта, для которого при распространении акустической волны изменяется расстояние между компонентами.
 Обнаружен эффект обратимого увеличения шунтирующего сопротивления при ультразвуковом нагружении;
используя модель дислокационного импеданса показано, что эффект связан с увеличением эффективности захвата электронов линейными дефектами.
 Показано, что основными акустоактивнымы радиационными дефектами в кремнии являются дивакансия и А--центр.
 Проведено сравнительный анализ аналитических, численных и эволюционных методов определения параметров диодов Шоттки.
 Показано, что наиболее приемлемыми методами с точки зрения точности и быстродействия являются эволюционные алгоритмы и метод Ли.
 Показано, что использование функции Ламберта в численных методах позволяет увеличить точность определения параметров.
 Определены механизмы переноса заряда и обратимого возрастания тока при ультразвуковом нагружении
 в структурах Al--$n$--$n^+$--Si--Al и их модификация вследствие $\gamma$--облучения.
 Обнаружены и исследованы эффект обратимого акустоиндуцированного влияния на свойства структур Mo/$n$--$n^{+}$--Si в широком диапазоне температур
 и показано, что они связаны с движением дислокационных перегибов и изменением размеров дефектных кластеров.
 Исследовано влияние микроволнового облучения на дефектную структуру приповерхностного слоя монокристаллов GaAs и SiC и эпитаксиальных структур GaAs;
 показано,  что ее изменения связаны с ростом концентрации междоузельных атомов.
 Экспериментально показано, что ультразвуковая обработка способна вызвать гомогенизацию как параметров арсенид галлиевых диодов Шоттки, созданных в едином технологическом процессе, так и энергетического спектра радиацийноиндуцированных ловушек на интерфейсе системы Si--SiO$_2$.


\keywordsRu


\begin{center}
{\section*{\MakeUppercase{ABSTRACT}}}
\end{center}

\textbf{\thesisAuthorFIOen~\thesisTitleEn.} ---  Manuscript.

\abstractBeginEn

The thesis concerns the research of ultrasonic loading influence on current in silicon solar cells,
 including irradiated.
 It has been revealed the reversible decreasing in the photoelectric transformation efficiency,
 which is related to acoustically induced rebuilding of point recombination centers.
 A model of acoustic complex defect is proposed.
% It is shown that the main acoustically active radiation defects are divacancy and A--center.
 A comparative analysis of analytical, numerical and evolutionary methods for Schottky diode parameters determination  has been carried out.
 The relationship between the type of the non--monotonicity of the Schottky barrier height dose dependence and the contact inhomogeneity degree is revealed.
 The reversible influence of ultrasound on the properties of silicon--metal structures is reveled and its features are established.
It is shown that this effect is associated with the dislocation overhangs movement and the defective clusters size change.
 The influence of microwave irradiation on the defect structure of the near--surface layer of  GaAs and SiC single crystals as well as GaAs epitaxial structures has been investigated.
 It is shown that defect structure changes are deals with the increase of interstitial atoms concentration.
 It has been found that ultrasonic treatment can causes increase in the homogeneity of both the parameters of GaAs Schottky diodes and the energy spectrum of radiation induced interface traps in Si--SiO$_2$.

\keywordsEn
