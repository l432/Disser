
\vspace{1.5em}
\begin{center}
\section*{\MakeUppercase{анотація}}
\end{center}

\vspace{-1.0em}

\textbf{\thesisAuthorFIO~\thesisTitle.} --- Рукопис.

\abstractBegin

Дисертація присвячена дослідженню впливу ультразвукового навантаження та опромінення ($\gamma$--кванти, нейтрони)
на протікання струму в структурах із $p$---$n$--переходом (Si) та контактом Шотткі (Si, GaAs).
%та дефектну структуру монокристалів  (GaAs, SiC) та епітаксійних структур (GaAs).
У кремнієвих сонячних елементах, у тому числі опромінених,
 виявлено акусто--індуковане зменшення часу життя носіїв заряду, оборотне при кімнатних температурах.
 Для пояснення виявлених ефектів запропоновано модель акустоактивного комплексного точкового рекомбінаційного центру.
% Показано, що акустоактивними радіаційними дефектами в кремнії є дивакансія та А--центр.
 Проведено порівняльний аналіз аналітичних, чисельних та еволюційних методів визначення параметрів діодів Шотткі
 та визначено найоптимальніші з них з погляду точності та швидкодії.
 В структурах Al---$n$--$n^+$--Si встановлено взаємозв'язок між характером зміни висоти бар'єру Шотткі при збільшенні дози $\gamma$--квантів та ступенем неоднорідності контакту.
 В структурах кремній---метал виявлено оборотній вплив ультразвука на висоту бар'єру та величину зворотного струму та показано, що він зумовлений рухом дислокаційних перегинів і зміною розмірів кластерів дефектів.
 Встановлено, що вплив мікрохвильового опромінення на дефектну структуру приповерхневого шару монокристалів GaAs і SiC та епітаксійних структур GaAs
 викликаний зростанням концентрації міжвузольних атомів.
 Виявлено, що ультразвукова обробка здатна підвищувати однорідність параметрів арсенід--ґалієвих діодів Шотткі та модифікувати концентрацію і енергетичний спектр радіаційно--індукованих пасток  на інтерфейсі системи  Si--SiO$_2$.

\keywords

\vspace{2.5em}

\begin{center}
{\section*{\MakeUppercase{АННОТАЦИЯ}}}
\end{center}

\vspace{-1.3em}

\textbf{\thesisAuthorFIOru~\thesisTitleRu.} --- Рукопись.

\abstractBeginRu

В диссертационной работе проведено исследование влияния ультразвуковой нагрузки (частота 2$\div$30~МГц, интенсивность до 1~Вт/см$^2$)
и радиационного облучения ($\gamma$--кванты, нейтроны) на протекание тока в структурах с $p$---$n$--переходом (Si) и контактом Шоттки (Si, GaAs),
а также структуру дефектов монокристаллов  (GaAs, SiC) и эпитаксиальных структур (GaAs).
В кремниевых солнечных элементах, в том числе и радиационно модифицированных, во время распространения ультразвука обнаружено
уменьшение времени жизни носителей заряда и изменение величины фактора неидеальности,
причем знак последнего эффекта противоположен в исходных и облученных образцах.
Эффекты обратимы при комнатных температурах.
Предложена модель акустоактивного комплексного точечного рекомбинационного центра, для которого в процессе ультразвуковой нагрузки изменяется расстояние между компонентами.
В рамках модели объяснены обнаруженные эффекты и показано, что наиболее эффективное взаимодействие акустической волны з точечным дефектом ожидается в случае, если последний состоит из компонент вакансионного и междоузельного типов.
Показано, что основными акустоактивными радиационными дефектами в кремнии являются дивакансия и А--центр.
 Обнаружен эффект обратимого увеличения шунтирующего сопротивления при ультразвуковой нагрузке;
используя модель дислокационного импеданса показано, что причиной эффекта является увеличение эффективности захвата электронов линейными дефектами.
 Проведен сравнительный анализ 16 аналитических, численных и эволюционных методов определения параметров диодов Шоттки из вольт--амперных характеристик.
 Показано, что наиболее оптимальными с точки зрения точности и быстродействия являются эволюционные алгоритмы и метод Ли,
 а использование функции Ламберта в численных методах позволяет увеличить точность определения параметров.
  Установлено, что протекание тока в структурах Al--$n$--$n^+$--Si--Al при температурах 130$\div$330 связано с процессами термоэлектронной эмиссии через неоднородный барьер и туннелирования.
 Облучение структур $\gamma$--квантами $^{60}$Co приводит к усилению туннельной компоненты тока, в частности, вследствие возникновения процессов многофононного туннелирования,
 а также к изменению высоты барьера, причиной которого является накопление дефектов акцепторного типа на границе металл--полупроводник.
 Показано, что наличие  локальных неоднородностей на границе раздела вызывает уменьшение эффективной  величины высоты барьера при облучении с дозой $10^6$~рад,
 тогда как за пределами патчей при этом барьер Шоттки увеличивается.
 Впервые обнаружен эффект обратимого возрастания тока в процессе ультразвуковой нагрузки при 305~К;
 установлено, что в исходных структурах он связан с взаимодействием акустической волны с линейными дефектами, тогда как в облученных --- с точечными радиационными.
  Обнаружен и исследован эффект динамического акусто--индуцированного влияния на высоту барьера Шоттки и величину обратного тока структур Mo---$n$--$n^+$--Si
 в диапазоне температур 130$\div$330;
 показано, что он обусловлен движением дислокационных перегибов и изменением размеров кластеров дефектов.
 При помощи метода акустоэлектрической релаксационной спектроскопии установлено,
 что влияние микроволнового облучения на дефектную структуру приповерхностного слоя монокристаллов GaAs и SiC и эпитаксиальных структур GaAs вызвано увеличением концентрации междоузельных атомов.
 Экспериментально показано, что ультразвуковая обработка способна уменьшать как разброс параметров арсенид галлиевых диодов Шоттки, созданных в едином технологическом процессе, так и концентрацию, и ширину энергетического спектра радиационно--индуцированных ловушек на интерфейсе системы Si--SiO$_2$.


\keywordsRu


\begin{center}
{\section*{\MakeUppercase{ABSTRACT}}}
\end{center}

\vspace{-1.0em}

\textbf{\thesisAuthorFIOen~\thesisTitleEn.} ---  Manuscript.

\abstractBeginEn


The thesis is devote to the study of the ultrasonic loading as well as irradiation ($\gamma$--rays, neutrons) influence
on current in the $p$--$n$ structures (Si) and Schottky diode (Si, GaAs).
It is revealed the acoustically induced reversible decrease in the carrier lifetime in the silicon solar cells,
 including irradiated.
 A model of acoustically active complex point recombination centers is proposed for the purpose of interpretation of revealed effects.
% It is shown that the main acoustically active radiation defects are divacancy and A--center.
 A comparative analysis of analytical, numerical and evolutionary methods for Schottky diode parameters determination  has been carried out and
 the most optimal ones are defined in terms of accuracy and speed.
 The relationship between the type of change in the Schottky barrier height of Al---$n$--$n^+$--Si structure with the increasing of $\gamma$--rays dose
 and the degree of contact inhomogeneity is revealed.
 The reversible influence of ultrasound on the both barrier height and reverse current in the silicon--metal structures is revealed and attributed to
the dislocation kinks movement and the defect clusters size changing.
 It is discovered that the effect of microwave irradiation on the defective structure of the near-surface layer of GaAs and SiC single crystals and as well as GaAs epitaxial structures
 is caused by increase in concentration of interstitial atoms.
 It has been found that ultrasonic treatment can causes both  increase in the homogeneity of the parameters of GaAs Schottky diodes and the modification in concentration as well as energy spectrum of radiation induced interface traps in Si--SiO$_2$.

\keywordsEn
