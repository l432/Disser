
\begin{center}
\section*{\MakeUppercase{анотація}}
\end{center}

\textbf{\thesisAuthorFIO~\thesisTitle.} --- Рукопис.

\abstractBegin

У дисертаційній роботі проведено дослідження впливу ультразвукового навантаження на процеси перенесення заряду в кремнієвих сонячних елементах,
 у тому числі і радіаційно опромінених.
 Виявлено оборотні процеси зменшення ефективності фотоелектричного перетворення,
 пов'язані з акустоіндукованою перебудовою точкових рекомбінаційних центрів.
 Запропонована модель акустоактивного комплексного дефекту.
 Показано, що основними акустоактивними радіаційними дефектами в кремнії є дивакансія та А--центр.
 Проведено порівняльний аналіз аналітичних, чисельних та еволюційних методів визначення параметрів діодів Шотки.
 Визначено механізми перенесення заряду та оборотного зростання струму при ультразвуковому навантаженні
 в структурах Al$-n-n^+$--Si---Al та їх модифікацію внаслідок $\gamma$--опромінення.
 Виявлено та досліджено ефект оборотного акустоіндукованого  впливу на властивості структур Mo/$n$--$n^{+}$--Si в широкому діапазоні температур
 та показано, що вони пов'язані з рухом дислокаційних перегинів та зміною розмірів дефектних кластерів.
 Досліджено вплив мікрохвильового опромінення на дефектну структуру приповерхневого шару монокристалів GaAs і SiC та епітаксійних структур GaAs; 
 показано,  що її зміни пов'язані зі зростанням концентрації міжвузольних атомів.
 Експериментально показано, що ультразвукова обробка здатна викликати гомогенізацію як параметрів арсенід галієвих діодів Шотки, створених в єдиному технологічному процесі, так і енергетичного спектру радіаційноіндукованих пасток  на інтерфейсі системи  Si--SiO$_2$.

\keywords


\begin{center}
{\section*{\MakeUppercase{АННОТАЦИЯ}}}
\end{center}

\textbf{\thesisAuthorFIOru~\thesisTitleRu.} --- Рукопись.

\abstractBeginRu

В диссертационной работе проведено исследование влияния ультразвукового нагрузки на процессы переноса заряда в кремниевых солнечных элементах,
 в том числе и радиационно облученных.
 Выявлено обратимые процессы уменьшения эффективности фотоэлектрического преобразования,
 связанные с акустоиндукованной перестройкой точечных рекомбинационных центров.
 Предложенная модель акустоактивного комплексного дефекта.
 Показано, что основными акустоактивнимы радиационными дефектами в кремнии являются дивакансии и А--центр.
 Проведен сравнительный анализ аналитических, численных и эволюционных методов определения параметров диодов Шоттки.
 Определены механизмы переноса заряда и обратимого возрастания тока при ультразвуковом нагружении
 в структурах Al$-n-n^+$--Si---Al и их модификация вследствие $\gamma$--облучения.
 Обнаружены и исследованы эффект обратимого акустоиндуцированного влияния на свойства структур Mo/$n$--$n^{+}$--Si в широком диапазоне температур
 и показано, что они связаны с движением дислокационных перегибов и изменением размеров дефектных кластеров.
 Исследовано влияние микроволнового облучения на дефектную структуру приповерхностного слоя монокристаллов GaAs и SiC и эпитаксиальных структур GaAs; 
 показано,  что ее изменения связаны с ростом концентрации междоузельных атомов.
 Экспериментально показано, что ультразвуковая обработка способна вызвать гомогенизацию как параметров арсенид галлиевых диодов Шоттки, созданных в едином технологическом процессе, так и энергетического спектра радиацийноиндуцированніх ловушек на интерфейсе системы Si--SiO$_2$.


\keywordsRu


\begin{center}
{\section*{\MakeUppercase{ABSTRACT}}}
\end{center}

\textbf{\thesisAuthorFIOen~\thesisTitleEn.} ---  Manuscript.

\abstractBeginEn

The thesis concerns the research of ultrasonic loading influence on charge transfer processes in silicon solar cells,
 including radiation irradiated.
 It has been revealed the reversible decreasing in the photoelectric transformation efficiency,
 which is related to acoustically induced rebuilding of point recombination centers.
 A model of acoustic complex defect is proposed.
 It is shown that the main acoustically active radiation defects are divacancy and A--center.
 A comparative analysis of analytical, numerical and evolutionary methods for Schottky diode parameters determination  has been carried out.
 The mechanisms of charge transfer and reversible current arising under ultrasound loading are determined 
 for the Al$-n-n^+$--Si---Al structures;
 the  mechanisms modification after $\gamma$--irradiation was examined as well.
 The effect of reversible acoustically induced influence on the properties of Mo/$n$--$n^{+}$--Si structures  was discovered and investigated in a wide temperature range;
 and it is shown that this effect is associated with the dislocation overhangs movement and the defective clusters size change.
 The influence of microwave irradiation on the defect structure of the near--surface layer of  GaAs and SiC single crystals as well as GaAs epitaxial structures has been investigated.
 It is shown that defect structure changes are deals with the increase of interstitial atoms concentration.
 It has been shown experimentally that ultrasonic treatment can causes increase in the homogeneity of both the parameters of GaAs Schottky diodes, manufactured in
the same technological cycle, and the energy spectrum of radiation induced interface traps in Si--SiO$_2$.


\keywordsEn
